
\chapter{概率}
\label{chap:probability}

\section{37法则}
\label{sec:37-rule}

\begin{example}[秘书问题,Secretary Problem]
  公司要要聘请一名秘书,共收到$n$个应聘者投递的简历。人事HR准备安排面试,每次面试一人,面试后就要及时决定是否聘他,如果当时决定不聘他,他便不会回来。面试后总能清楚了解应聘者的合适程度,并能和之前曾经面试过的每个人做比较。问有什么策略,可以使从这$n$个应聘者中选中其中最合适应聘者的概率最大。
\end{example}

在统计学中,此类问题的变种有很多,比如相亲问题、止步问题、见好就收问题、苏丹的嫁妆问题、挑剔的求婚者问题等等。这种样本数量固定且每个样本只出现一次的情况下选择其中最优样本的问题,在统计学中有一个37法则,即把样本总量的前37\%的样本做为参考,以这37\%样本中最优的那个作为参考点,如果在剩下63\%的样本中,出现比参考点好的样本,就果断选择它;如果剩下的63\%的样本中都没有比参考点更好的样本,就取最后一个样本。

\begin{proof}[37法则的证明]
  37法则采用的策略是前$k$个样本作为训练集并取训练集中最佳的样本作为参考样本,后面的$n-k$个样本只要出现比参考样本还要好的样本就取之作为候选样本。若全部$n$个样本中的最佳样本出现在前$k$个训练样本中,则后面$n-k$个样本中取不到比最佳样本还要好的样本,按策略只能取最后一个样本作为候选样本。这种策略取出的候选样本是最佳样本的概率为:
  \begin{align*}
    P(k) & = && \sum_{i=1}^n P(\small\text{第$i$个样本是最佳样本且选了第$i$个样本为候选样本})\\
         & = && \left(\sum_{i=1}^k + \sum_{i=k+1}^n\right) P(\small\text{第$i$个样本是最佳样本且选了第$i$个样本为候选样本})\\
         & = && \sum_{i=k+1}^n P(\small\text{第$i$个样本是最佳样本且选了第$i$个样本为候选样本})\\ % &\quad\text{前$k$个永远不是候选样本}
             & = && \sum_{i=k+1}^n P(\small\text{第$i$个样本是最佳样本且前$i-1$个样本中的最佳样本在前$k$个样本中})\\
         & = && \sum_{i=k+1}^n \frac{k}{i-1} \cdot \frac1n = \frac{k}{n}\sum_{i=k+1}^n\frac1{i-1}
  \end{align*}
  对于$k=0$的情况,候选样本就是第一个样本,其为最佳样本的概率为$1/n$,从而$P(0)=1/n$。

  上式的右边是个离散函数,可以考虑其连续形式来取得其极值点。记$x\equiv k/n$,并令$n\to+\infty$,则有
  \begin{align*}
    \mathcal{P}(x) \equiv \lim_{n\to+\infty}P(k) = \lim_{n\to+\infty} \frac{k}{n}\cdot \sum_{i=k+1}^n \frac{n}{i-1}\cdot \frac{1}{n} = x\int_{x}^1 \frac1t\mathrm{d}t = -x\mathrm{ln}(x)
  \end{align*}
  上式是连续函数的情况,令其导数为零,可知其极点为$x=1/e\approx 0.367879\approx 37\%$,这就是37法则名字的来源。对于离散情况即$n$为有限值时,当$k=n/e\approx 0.37n$时是最佳分割。
\end{proof}

\begin{example}
  一楼到十楼的每层电梯门口都放着一颗钻石,钻石大小不一,随机分布。一个人乘坐电梯从一楼到十楼,每层楼电梯门都会打开一次。若允许这人在从一楼到十楼的过程中拿一次钻石,即不能观察完十层楼所有的钻石后再坐一次电梯拿钻石,也不能拿了一颗钻石后再更换为另一颗钻石,问这个人如何做才能使他尽可能地拿到最大的钻石?
\end{example}
\begin{proof}[提示]
  这个问题也许是没有标准答案的,其中的一种分析方法是应用37法则,按37法则,$10\times 0.37\approx 4$,取前四层的钻石作为参考,从第五层开始,只要出现比前四层钻石都要大的钻石就选此层的钻石,若一直走到最后的十楼都没有发现有比前四层楼中钻石都要大的钻石,就取最后一层第十层的钻石。
\end{proof}