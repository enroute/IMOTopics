
\chapter{圆}
\label{chap:circle}

到一点的距离是某个固定常数(即半径)的点的集合,在三维空间中就是球面,而在平面中就是圆。圆在生活中应用非常广泛,比如下水道井盖为什么是圆的,其中一个原因就是不管井盖怎么放都不会掉到井里去。

\section{圆与三角形}
\label{sec:circle-and-triangle}

\begin{example}
  以固定长度$a$为斜边作一个直角三角形,则这些三角形中,该斜边上的高的最大值是多少?
\end{example}
\begin{proof}[提示]\mbox{}\par
  \begin{center}
    \begin{tikzpicture}[scale=1.0]
      \foreach \r/\theta in{1.5/55}{
        \draw(0,0)circle(1.5);
        \coordinate(A)at(-1.5,0);\coordinate(B)at(1.5,0);\coordinate(C)at(\theta:1.5);
        \coordinate(D)at($(B)!(C)!(A)$);
        \draw(C)--(A)--(B)node[midway,below]{$a$}--(C)--(D)  node[midway, left]{$h$};
        \tkzMarkRightAngle(C,D,B)
      }
    \end{tikzpicture}
  \end{center}
  另外一个顶点都是落在图中的圆上,显然高的最大值是圆的半径。
\end{proof}

\begin{theorem}[圆内接四边形]
  平面上四边形$ABCD$,以下几个命题等价:
  \begin{enumerate}
  \item $A,B,C,D$共圆;
  \item $\angle A + \angle C = \pi$;
  \item $\angle ACB = \angle ADB$。
  \end{enumerate}
\end{theorem}
\begin{proof}[提示]
  $(1)\implies(2)$及$(1)\implies(3)$都是显然的。

  $(2)\implies(1)$。等价于固定$A,B,C$,平面上不含点$B$的直线$AC$的另一侧上所有使$\angle AD'C + \angle ABC=\pi$的动点$D'$都落在$\triangle ABC$的外接圆上。分动点在圆内及圆外两种情况考虑即可。

  \begin{center}
    \begin{tikzpicture}[scale=1.0]
      \begin{scope}
        \foreach \r/\a/\b/\c/\d/\dd in{1.5/210/320/40/120/150}{
          \coordinate(O)at(0,0);
          \coordinate[label=left:$A$](A)at(\a:\r);
          \coordinate[label=above:$C$](C)at(\c:\r);
          \coordinate[label=above:$D$](D)at(\d:\r);
          \coordinate[label=above:$D''$](D'')at(\dd:\r);
          % \tkzInterLC(C,D')(O,C)\tkzGetPoints{D''}{}
          \coordinate[label=above:$D'$](D')at($(D'')+.3*(D'')-.3*(C)$);
          \coordinate(A')at($(A)+.3*(A)-.3*(C)$);
          \coordinate(C')at($(C)+.3*(C)-.3*(A)$);
          \coordinate(X)at(0,-1.1*\r);
          \fill[color=blue!30](A')--(A' |- X)--(C' |- X)--(C');
          \coordinate[label=right:$B$](B)at(\b:\r);
          \draw(0,0)circle(\r);
          \draw(A)--(B)--(C)--(D)--cycle (A)--(C) (B)--(D);
          \draw[dashed](C)--(D')--(A)--(D'');
          \tkzDrawPoints(A,B,C,D,D',D'')
          \draw[dashed](A')--(C');
        }
      \end{scope}
      \begin{scope}[shift={(6,0)}]
        \foreach \r/\a/\b/\c/\d/\dd in{1.5/210/320/40/120/170}{
          \coordinate(O)at(0,0);
          \coordinate[label=left:$A$](A)at(\a:\r);
          \coordinate[label=above:$C$](C)at(\c:\r);
          \coordinate(A')at($(A)+.3*(A)-.3*(C)$);
          \coordinate(C')at($(C)+.3*(C)-.3*(A)$);
          \fill[color=blue!30](A')--(A' |- X)--(C' |- X)--(C');

          \coordinate[label=above:$D$](D)at(\d:\r);
          \coordinate[label=above:$D''$](D'')at(\dd:\r);
          \coordinate[label=right:$B$](B)at(\b:\r);
          \coordinate[label=above:$D'$](D')at($(D'')+.3*(D'')-.3*(B)$);
          \coordinate(X)at(0,-1.1*\r);
          \draw(0,0)circle(\r);
          \draw(A)--(B)--(C)--(D)--cycle (A)--(C) (B)--(D);
          \draw[dashed](B)--(D')--(A)--(D'');
          \tkzDrawPoints(A,B,C,D,D',D'')
          \draw[dashed](A')--(C');
        }
      \end{scope}
    \end{tikzpicture}
  \end{center}

  $(3)\implies(1)$。类似的,分半平面内圆内与圆外两种情况分别考虑动点$D'$,若$\angle AD'B=\angle ACB$,则$D'$必落在$\triangle ABC$的外接圆上\footnote{但外接圆上的任一点$P$不一定有$\angle APB=\angle ACB$,除非$A,B$是圆上一直径的两端点。},这可以分动点在圆内及圆外两种情况分别考虑得到。
\end{proof}

\begin{theorem}[割线定理]
  平面上线段$AB$与$CD$相交于$E$,则以下命题等价:
  \begin{enumerate}
  \item $A,B,C,D$共圆;
  \item $AE\cdot EB=CE\cdot ED$。
  \end{enumerate}
\end{theorem}
\begin{proof}[提示]
  $(1)\implies(2)$。连接$AC$及$BD$(或者连接$AD$及$BC$),由三角形相似显然可得。

  \begin{center}
    \begin{tikzpicture}
      \foreach \r/\a/\b/\c/\d in{1.5/210/50/320/120}{
        \coordinate(O)at(0,0);
        \coordinate[label=left:$A$](A)at(\a:\r);
        \coordinate[label=above:$B$](B)at(\b:\r);
        \coordinate[label=right:$C$](C)at(\c:\r);
        \coordinate[label=above:$D$](D)at(\d:\r);
        \tkzInterLL(A,B)(C,D)\tkzGetPoint{E}
        \draw(O)circle(\r);
        \draw(A)--(B) (C)--(D);
        \draw[dashed](A)--(D) (B)--(C);
        \tkzDrawPoints(A,B,C,D,E)
        \node[above]at(E){$E$};
      }
    \end{tikzpicture}
  \end{center}

  $(2)\implies(1)$。由$AE\cdot EB=CE\cdot ED$,可知$\triangle AED\sim\triangle CEB$,从而$\angle A=\angle C$,四点共圆。
\end{proof}