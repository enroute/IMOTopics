
\chapter{历法}
\label{chap:calendar}

\epigraph{清明时节雨纷纷,路上行人欲断魂。\\借问酒家何处有,牧童遥指杏花村。}{唐·杜牧《清明》}

清明是中国从古流传至今的二十四节气之一,而公历是近代才传入中国的,为什么每年的清明不是公历4月4日就是4月5日呢?看完这节,希望你能想明白,一年中在2月29日之前的节气,每年都是固定日期的,而在2月29日之后的节气,每年都是差不多是固定日期的,最多相差一天。

\section{太阳历}
\label{sec:solar-calendars}

太阳历又称为阳历,是以地球绕太阳公转的运动周期为基础而制定的历法。

\begin{definition}[回归年,太阳年,Tropical Year,Solar Year]
  太阳年是指在地球上观察到太阳回到同一个地方所需要的时间,即地球绕太阳旋转一周所需要的时间,大约是$365.2422$天。
\end{definition}

日历年是日历中经过一年所包含的时间。历史上制定日历,除了记数之外,另外一个很重要的作用是指导农耕生产,因此自然是希望日历与太阳历越是一致越好,这样看日历就知道当天是处于什么季节。

为了记数方便,日历年都是取的整数,一般取为太阳年最接近的整数即365天。若日历年直接采用365天而不作其它修正的话,那么一年下来日历年与太阳年的误差为0.2422天,如下图。若不作修正的话,每个日历年上不同年份的同一个日期(比如1月1日),地球所在的位置就会不一样。这样的话,日历就只剩余记数的作用,而失去了指导生活生产(比如农业种植)的作用,因为在这样的日历下,有些年份的1月1日甚至会是在炎热的夏季。

置闰的方法可以修正这个误差,即通过在日历年中插入天数来调整日历年与太阳年的误差。由于$0.2422\times4\approx1$,儒略历(Julian Calendar)就采用每4年添加一日(闰日)的方式来修正,即儒略历的平年有365天,每隔4年就要添加一日(闰日),该年就是有366天的闰年。

\begin{center}
  \begin{tikzpicture}[scale=1.0]
    \begin{scope}
      % the sun
      \draw(0,0)node{\tiny Sun}circle(.3);
      \foreach \t in{0,30,60,90,120,150,180,210,240,270,300,330}{%
        \draw(\t:.5)--(\t:.7);
      }
      \draw(0,0)circle(2);
      \draw[fill=white](0:2)node[left]{\tiny Earth}circle(.1);
      \draw[-{>[scale=2.5,length=2,width=3]}](0:2.4)arc(0:45:2.4);
      \node[below]at(0,-2.5){开始位置};
    \end{scope}
    \begin{scope}[shift={(6,0)}]
      % the sun
      \draw(0,0)circle(.3);
      \foreach \t in{0,30,60,90,120,150,180,210,240,270,300,330}{%
        \draw(\t:.5)--(\t:.7);
      }
      \draw(0,0)circle(2);
      \draw[dashed,help lines,fill=white](0:2)circle(.1);
      \draw[fill=white](350:2)circle(.1);
      \draw[dashed,help lines,-{>[scale=2.5,length=2,width=3]}](0:2.4)arc(0:350:2.4);
      \draw[dashed,help lines,<->](350:2.8)arc(350:360:2.4);
      \draw[dashed,help lines](3,0)--(0,0)--(350:3);
      \path (355:2)--(355:3)node[sloped,right]{0.2422天};
      \node[below]at(0,-2.5){一个日历年之后};
    \end{scope}
  \end{tikzpicture}
\end{center}

按儒略历每4年一闰日,日历上过了4年为$1+365\times4$天,而实际的4个太阳年共有$365.2422\times4$天,4年下来其误差为$1+365\times4-365.2422\times4=1-0.2422\times4=0.0312$天,再扩大一下,日历上过了400年,则与实际的400个太阳年误差为3.12天,即比太阳年多跑了3.12天。为解决这个误差,格里历(Gregorian Calendar),即现在通用的公历,在每400年中去掉3个闰年。格里历的置闰规则如下:
\begin{quotation}
  四年一闰,百年不闰,四百年再闰。
\end{quotation}

剩余的每400年产生的$3.12-3=0.12$天的误差,就需要后人在用到时继续修正了。

\begin{definition}[闰年,Leap Year]
  2月份有28天的称为平年,2月份有29天的称为闰年。确定平年闰年的方法是:
  \begin{enumerate}
  \item 能被400整除的年份定为闰年;
  \item 否则能被100整除的年份定为平年;
  \item 否则能被4整除的年份定为闰年;
  \item 其余年份定为平年。
  \end{enumerate}
  即不能被4整除的都是平年;能被400整除的是闰年;不能被400整除但能被100整除的是平年;不能被100整除但能被4整除的是闰年。
\end{definition}

闰年置闰日的方法是在当年2月份增加一天,从原来的28天加到29天。

\begin{table}[htbp]
  \centering
  \caption{每月天数}
  \label{tab:days-of-months}
  \begin{minipage}{\textwidth}  %for footnote in tabular
  \centering
  \begin{tabular}{ccccccccccccc}
    \toprule
    月份 & 1  & 2          & 3  & 4  & 5  & 6  & 7  & 8  & 9  & 10 & 11 & 12\\\midrule
    天数 & 31 & 28\footnote{平年28天,闰年29天} & 31 & 30 & 31 & 30 & 31 & 31 & 30 & 31 & 30 & 31\\
    \bottomrule
  \end{tabular}
  \end{minipage}
\end{table}

\begin{example}
  公元1900年是平年,因为$400\notdivides 1900$,但$100\mid 1900$。公元2012年是闰年,因为$4\mid 2012$但$100\notdivides 2012$。
\end{example}

\begin{example}
  小明已经上小学4年级了,可是他却说:“我妈妈说我每个生日都没有落下,可是我到现在为止总共才给我过了3个生日。”小明说的有可能吗?
\end{example}
\begin{proof}[提示]
  比如闰年闰日过生日。
\end{proof}

\begin{example}[10年3650天]
  陈奕迅有首歌叫《十年》,吕珊有首歌叫《3650夜》。那么,十年到底有可能是几天?
\end{example}
\begin{proof}[提示]
  普通情况下,四年一闰。考虑到百年不闰的情况,十年有可能有一个闰年、两个闰年或三个闰年。
  {\small
  \begin{align*}
    \underline{2008},\ 2009,\ 2010,\ 2011,\ \underline{2012},\ 2013,\ 2014,\ 2015,\ \underline{2016},\ 2017,\ 2018,\ 2019,\ \underline{2020}
  \end{align*}
  }
  从2008--2017这十年包含了3个闰年(下划线部分),从2009--2018这十年包含了两个闰年。考虑百年不闰的1900年,以其为中心向两边扩散,则十年的范围有可能只包含一个闰年。
  {
  \begin{align*}
    1897,\ 1898,\ 1899,\ 1900,\ 1901,\ 1902,\ 1903,\ \underline{1904},\ 1905,\ 1906&\qedhere
  \end{align*}
  }
\end{proof}

\begin{example}
  小明是2013年入学的小学生,今年全家去旅行,过了三天后回家。到家后小明一连撕掉了3张日历。姨妈打电话过来问起小明什么时候去的旅行,小明说不记得了,只记得刚刚撕掉的3张日历的数字和是32。那么你知道小明旅行是在哪一年吗?
\end{example}
\begin{proof}[提示]
  撕掉3张日历,暗示了这3张正好对应小明旅行的3天日期。两种情况,一是3天都是同一月份,二是3天在不同的月份。

  如果3天是同一月份,则连续3天的日期之和应能被3整除,然而32并不能被3整除,排除这种情况。

  3天不在同一月份,则必须有一天是上月月末,一天是当月第一天即1号,另外一天可能是当月2号,也可能是上月倒数第二天。如果是上月月末一天外加当月1号2号两天,则上月月末日期数字为$32-1-2=29$。如果是当月1号外加上月最后两天,则上月最后两天的日期数字之和为$32-1=31$,不可能,因为每月最后两天日期数字之和最小的也有$27+28>31$。所以被撕掉的3天日期的唯一情况是$29,1,2$。

  从而当年是闰年。从2013年开始的闰年按顺序依次为:2016,2020,2024,$\cdots\cdots$。而小明是2013年上的小学,题目中说明全家旅行时小明仍然是小学生。如果是2020年的2月份去的旅行,那么小明在旅行时应该已经是初中二年级上学期了,所以题目中说的今年只能是2016年。
\end{proof}

\begin{example}
  柳如是在给岳怜花表演魔术。柳如是请岳怜花在一张纸上写上她的生日的月份,将这个数字乘以4,加13,然后再乘以25。之后减200,然后再加上她生日的日期(即生日年月日里的“日”),然后乘以2,再减40,再乘以50,然后再加上她出生年份的后两位数。柳如是请岳怜花将最后的结果说出来,岳怜花说是71512。柳如是想了一下,说:“那我知道你是什么时候出生的了,2012年6月10日。”岳怜花说:“太神奇了!可是你早就知道我的生日了呀。”
\end{example}
\begin{proof}[提示]
  这个问题其实与日历关系不大,只是个普通的数字游戏。记岳怜花的出生年(后两位)月日分别是$y$,$m$和$d$,最终结果为$s$,则
  \begin{align*}
    s ={}& \left\{ \left[ (4m + 13)\times 25 - 200 + d \right]\times 2 - 40 \right\} \times 50 + y \\
      ={}&       \left[ (100m + 325 - 200 + d) \times 2 - 40 \right] \times 50 + y\\
      ={}& (200m + 210 + 2d) \times 50 + y\\
      ={}& 10000m + 10500 + 100d + y
  \end{align*}
  所以$(s - 10500$这个数字,万位以上的数字就是$m$(月份),千位与百倍两数字组成的就是$d$(日),最后两位就是年。

  回到柳如是的魔术,$71512 - 1050 = 61012$,最后面两位$12$是年,紧接着两位$10$是日,剩余的$6$就是月。

  从后往前反过来推,也可以设计不同的数字魔法。比如若化简到最后的结果为
  \begin{align*}
    s = 10000y + 100m + d + 1234
  \end{align*}
  那么最后的结果减去1234之后,最后两位就是日,紧接着两位就是月,剩余最前面的就是年。
\end{proof}



\section{太阴历}
\label{sec:tai-yin-calendar}



\section{大明历}
\label{sec:da-ming-calendar}

祖冲之在公元465(也有人说是公元462年,即刘宋大明六年)制定了大明历。祖冲之在此之前就测得了地球围绕太阳旋转一周的天数大约为365.24281481天。由于该数不是正整数,古人制定历法,很重要的一件事情就是对历法调整安排,使得制定的历法与季节的循环相匹配。

\section{节气}
\label{sec:jie-qi}

古人把$360^\circ$黄经划分24等分,每隔$15^\circ$为一节气,其中$0^\circ$为春分,夏至是$90^\circ$,秋分是$180^\circ$,冬至是$270^\circ$。而太阳黄经在$15^\circ$时则定为清明,太阳直射北回归线之日即中国北方最热之时定为夏至,如图~\ref{fig:24-jie-qi}所示。

公历(即格里历)是根据太阳年来确定,其日期是与黄经基本是一一对应的,误差在一日之内(闰日),所以黄经每个角度对应的公历日期基本是固定的,所以节气在公历上的体现也基本是固定的,误差也在一日之内。

\begin{figure}[htbp]
  \centering
  \begin{tikzpicture}[scale=1.0]
    \draw(0,0)circle(5);
    \foreach \a/\v in {0/春分,15/清明,30/谷雨,45/立夏,60/小满,75/芒种,90/夏至,
      105/小暑,120/大暑,135/立秋,150/处暑,165/白露,180/秋分,
      195/寒露,210/霜降,225/立冬,240/小雪,255/大雪,270/冬至,
      285/小寒,300/大寒,315/立春,330/雨水,345/惊蛰
    }{
      \draw(0,0)--(90-\a:5.2)node[pos=1.1]{\v};
    }
    \foreach \a/\r/\v in {0/4.5/$0^\circ$, 90/4.3/$90^\circ$, 180/4.5/$180^\circ$, 270/4.2/$270^\circ$}{
      \draw[very thick](0,0)--(90-\a:5.2);
      \node[fill=white] at(90-\a:\r) {\v};
    }
    \foreach \a/\r/\v in {45/4.3/$45^\circ$, 135/4.3/$135^\circ$, 225/4.2/$225^\circ$, 315/4.2/$315^\circ$}{
      % \draw[very thick](0,0)--(90-\a:5.2);
      \node[fill=white] at(90-\a:\r) {\v};
    }
  \end{tikzpicture}
  \caption{二十四节气}
  \label{fig:24-jie-qi}
\end{figure}

节气是古人用于指导工作的,如春耕、播种等,与太阳与地球的相对位置密切相关的。节气实际上是一种阳历。即中国古代所使用的历法既包括阴历,也包括阳历(节气),所以实际上是一种阴阳历。


\section{天干地支}
\label{sec:gan-zhi}

天干地支是中国古代的记数方法。

\begin{definition}[天干]
  十天干是指:\nopagebreak

  \centering
  甲、乙、丙、丁、戊、己、庚、辛、壬、癸
\end{definition}

\begin{definition}[地支]
  十二地支是指:\nopagebreak

  \centering
  子、丑、寅、卯、辰、巳、午、未、申、酉、戌、亥
\end{definition}

天干和地支组合就是以「甲子」为首的六十干支循环,如表~\ref{tab:gan-zhi-loop}。这是一种循环配对,并不是天干任取一个,地支任取一个总数为$10\times12=120$种的配对。容易观察到,由于天干与地支的个数都是偶数,这种循环配对只有奇数配奇数,偶数配偶数的配法,如排第1的天干“甲”,便只能与排1、3、5、7、9、11的“子、寅、辰、午、申、戌”配对。

\begin{table}[htbp]
  \centering
  \caption{干支循环}
  \label{tab:gan-zhi-loop}
  \begin{tikzpicture}[scale=1.0]
    \setcounter{X}{0}
    \newcommand{\dizhi}{品}
    \foreach \y in{0,1,2,3,4,5}{%,6,7,8,9,10,11}{%
      \foreach \x/\tiangan in{0/甲,1/乙,2/丙,3/丁,4/戊,5/己,6/庚,7/辛,8/壬,9/癸}{%
        % \foreach \y in{0/子,1/丑,2/寅,3/卯,4/辰,5/巳,6/午,7/未,8/申,9/酉,10/戌,11/亥}{%
        % FIXME: \underline cause box raised!!
        \ifnum0=\theX      {\node at(\x,-\y*.8){\raisebox{-6pt}{\underline{\tiangan{}子}}};}
        \else\ifnum\theX=1 {\node at(\x,-\y*.8){\tiangan{}丑};}
        \else\ifnum\theX=2 {\node at(\x,-\y*.8){\tiangan{}寅};}
        \else\ifnum\theX=3 {\node at(\x,-\y*.8){\tiangan{}卯};}
        \else\ifnum\theX=4 {\node at(\x,-\y*.8){\tiangan{}辰};}
        \else\ifnum\theX=5 {\node at(\x,-\y*.8){\tiangan{}巳};}
        \else\ifnum\theX=6 {\node at(\x,-\y*.8){\tiangan{}午};}
        \else\ifnum\theX=7 {\node at(\x,-\y*.8){\tiangan{}未};}
        \else\ifnum\theX=8 {\node at(\x,-\y*.8){\tiangan{}申};}
        \else\ifnum\theX=9 {\node at(\x,-\y*.8){\tiangan{}酉};}
        \else\ifnum\theX=10{\node at(\x,-\y*.8){\tiangan{}戌};}
        \else\ifnum\theX=11{\node at(\x,-\y*.8){\tiangan{}亥};}
        \fi\fi\fi\fi\fi\fi\fi\fi\fi\fi\fi\fi
        \stepcounter{X}
        \ifnum\theX=12\setcounter{X}{0}\fi
      }
    }
  \end{tikzpicture}
\end{table}