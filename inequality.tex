
\chapter{不等式}
\label{chap:inequality}

\section{排序不等式}
\label{sec:rearrangement-inequality}

\begin{theorem}[排序不等式,Rearrangement Inequality]
  对任意两个实数序列$\{x_i|i=1,2,\cdots,n\}$及$\{y_i|i=1,2,\cdots,n\}$,若两序列都是递增排序的,即有
  \begin{align*}
    x_1\le x_2 \le x_3 \le \cdots \le x_n\\
    y_1\le y_2 \le y_3 \le \cdots \le y_n
  \end{align*}
  则对集合$\{x_i\}$的任意一个排列组成的序列$\{x_i'|i=1,2,\cdots,n\}$,有
  \begin{align*}
    x_ny_1 + x_{n-1}y_2 + x_{n-2}y_3 + \cdots x_1y_n 
    \le&
         x_1'y_1 + x_2'y_2 + x_3'y_3 + \cdots + x_n'y_n \\
    \le&
    x_1y_1 + x_2y_2 + x_3y_3 + \cdots + x_ny_n
  \end{align*}
  即“$\text{反序和}\le \text{乱序和} \le \text{顺序和}$”。
\end{theorem}
\begin{proof}
  只需证明“乱序和$\le$顺序和”即顺序和取得最大值,然后对序列
  \begin{align*}
    -x_n\le -x_{n-1} \le -x_{n-2} \le \cdots \le -x_1,\quad\quad
    y_1 \le y_2 \le y_3\cdots y_n
  \end{align*}
  应用顺序和取得最大值即可得反序和取得最小值。

  % 可用反证法,或者对$n$用数学归纳法。但数学归纳法到最后还是要用到反证。

  % 用反证法证明和取到最大值的重排是递增的排列。

  假设存在某个重排$\{x_i'\}$,使得其与$\{y_i\}$的乘积和取得最大值。下面分析可以将$\{x_i'\}$重新排列成递增的数列使得新数列与$\{y_i\}$的乘积和也是最大值。

  对任意$1\le s < t\le n$,由$\{y_i\}$的递增性,有$y_s\le y_t$。分$y_s=y_t$与$y_s<y_t$两种情况考虑。

  若$y_s<y_t$,则必有$x_s'\le x_t'$。否则若$x_s'>x_t'$,交换$x_s'$与$x_t'$后得到的新序列
  \begin{center}
    \begin{tikzpicture}[scale=1.0]
      \foreach \x/\i in {0/$x_1'$, 1/$x_2'$, 2/$x_3'$, 3/$\cdots$, 4/$x_s'$, 5/$\cdots$, 6/$x_t'$, 7/$\cdots$, 8/$x_n'$}{%
        \ifnum4=\x
          %\node[draw,circle](N0\x) at (\x, 0) {\i};
          \draw(\x,0)circle(.3);
        \else\ifnum6=\x
          %\node[draw,circle](N0\x) at (\x, 0) {\i};
          \draw(\x,0)circle(.3);
        \fi\fi
        \node(N0\x) at (\x, 0) {\i};
      }
      \foreach \x/\i in {0/$x_1'$, 1/$x_2'$, 2/$x_3'$, 3/$\cdots$, 4/$x_t'$, 5/$\cdots$, 6/$x_s'$, 7/$\cdots$, 8/$x_n'$}{%
        \ifnum4=\x
          %\node[draw,circle](N0\x) at (\x, 0) {\i};
          \draw(\x,1.5)circle(.3);
        \else\ifnum6=\x
          %\node[draw,circle](N0\x) at (\x, 0) {\i};
          \draw(\x,1.5)circle(.3);
        \fi\fi
        \node(N1\x) at (\x, 1.5) {\i};
      }
      \node[left] at (-1,0) {原重排:};
      \node[left] at (-1,1.5) {交换两数:};
      \draw[<->](N04)--(N16);
      \draw[<->](N06)--(N14);
    \end{tikzpicture}
  \end{center}
  则此两数列与$\{y_i\}$的乘积和的差别在于
  \begin{align*}
    x_t'y_s+x_s'y_t - (x_s'y_s+x_t'y_t)& = (x_t'-x_s')y_s + (x_s'-x_t')y_t\\
                                      & = (x_s'-x_t')(y_t-y_s) > 0
  \end{align*}
  即新构造的重排数列与$\{y_i\}$的乘积和更大,这与$\{x_i'\}$与$\{y_i\}$的乘积和为最大值矛盾。从而若$y_s<y_t$,则必有$x_s'\le x_t'$。

  若$y_s=y_t$,则由$\{y_i\}$的递增性,必有$y_s=y_{s+1}=\cdots=y_t$。此时可以对$x_s', x_{s+1}', \cdots, x_{t}'$作递增重排代替原来位置上的数,得到的数列与$\{y_i\}$的乘积和不变,同样为最大值。
  \begin{center}
    \begin{tikzpicture}[scale=1.0]
      \foreach \x/\i in {0/$x_1'$, 1/$x_2'$, 2/$x_3'$, 3/$\cdots$, 4/$x_s'$, 5/$\cdots$, 6/$\cdots$, 7/$x_t'$, 8/$\cdots$, 9/$x_n'$}{%
        \ifnum4=\x
          %\node[draw,circle](N0\x) at (\x, 0) {\i};
          \draw(\x,0)circle(.3);
        \else\ifnum5=\x
          %\node[draw,circle](N0\x) at (\x, 0) {\i};
          \draw(\x,0)circle(.3);
        \else\ifnum6=\x
          %\node[draw,circle](N0\x) at (\x, 0) {\i};
          \draw(\x,0)circle(.3);
        \else\ifnum7=\x
          %\node[draw,circle](N0\x) at (\x, 0) {\i};
          \draw(\x,0)circle(.3);
        \fi\fi\fi\fi
        \node(N0\x) at (\x, 0) {\i};
      }
      \draw[decorate,decoration={brace,amplitude=5pt,mirror}](N04.south)--(N07.south);
      \node[below] at (5.5,-.5) {此部分递增重排};
    \end{tikzpicture}
  \end{center}

  综合起来,对相等的$y_i$对应的$x_i'$重新作递增排列后可得到一递增的$\{x_i\}$的重排列,其与$\{y_i\}$的乘积和取到最大值。  
\end{proof}

\begin{theorem}[排序不等式,Rearrangement Inequality]
  类似的,若两实数数列是严格递增的,即
  \begin{align*}
    x_1< x_2 < x_3 < \cdots < x_n\\
    y_1< y_2 < y_3 < \cdots < y_n
  \end{align*}
  则同样有“反序和$<$乱序和$<$顺序和”。
\end{theorem}

\begin{corollary}
  对任意正数数列$(a_1,a_2,\cdots, a_n)$,记$(a_1',a_2',\cdots,a_n')$是其任一重排列,则有
  \begin{align*}
    \frac{a_1}{a_1'} + \frac{a_2}{a_2'} + \cdots + \frac{a_n}{a_n'}\ge n
  \end{align*}
\end{corollary}
\begin{proof}[提示]
  不妨设$a_1,a_2,\cdots,a_n$是递增。考虑两递增数列:
  \begin{align*}
    \begin{array}{ccccccccc}
      a_1&\le& a_2&\le& a_3&\le&\cdots&\le& a_n\\
      \dfrac1{a_n}&\le& \dfrac1{a_{n-1}}&\le& \dfrac1{a_{n-2}}&\le&\cdots&\le& \dfrac1{a_1}
    \end{array}
  \end{align*}
  应用“反序和$\le$乱序和”可得。
\end{proof}

\begin{example}[Nesbitt's Inequality]对任意正数$a,b,c$,有
  \begin{align*}
    \frac{a}{b+c}+\frac{b}{c+a}+\frac{c}{a+b}\ge \frac32
  \end{align*}
\end{example}
\begin{proof}[提示]
  该不等式有多种证明方法。此处使用排序不等式。由对称性,不妨设$a\le b\le c$,从而有$a+b\le a+c\le b+c$。对两个递增数列
  \begin{align*}
    a \le b \le c, \quad \quad \frac1{b+c}\le \frac1{a+c}\le \frac1{a+b}
  \end{align*}
  应用排序不等式,有
  \begin{align*}
    \frac{a}{b+c}+\frac{b}{a+c}+\frac{c}{a+b}\ge \frac{b}{b+c}+\frac{c}{a+c}+\frac{a}{a+b}\\
    \frac{a}{b+c}+\frac{b}{a+c}+\frac{c}{a+b}\ge \frac{c}{b+c}+\frac{a}{a+c}+\frac{b}{a+b}
  \end{align*}
  其中上面两式的右边都是乱序和。两式相加,可得。
\end{proof}

% \begin{example}
%   任意正数$a,b,c,d$,有
%   \begin{align*}
%     \frac{a}{b+c}+\frac{b}{c+d}+\frac{c}{d+a}+\frac{d}{a+b}
%   \end{align*}
% \end{example}
% \begin{proof}[提示]
%   先简化分母,令
%   \begin{align*}
%     u\equiv b+c,\quad v\equiv c+d, \quad x\equiv d+a,\quad y\equiv a+b
%   \end{align*}
%   则有
%   \begin{align*}
    
%   \end{align*}
% \end{proof}

\begin{example}[IMO 1995]
  $a,b,c$是正数,且$abc=1$,则
  \begin{align*}
    \frac1{a^3(b+c)} + \frac1{b^3(c+a)} + \frac1{c^3(a+b)}\ge\frac32
  \end{align*}
\end{example}
\begin{proof}
  要证明的不等式等价于
  \begin{align*}
    \frac{b^3c^3}{(b+c)} + \frac{c^3a^3}{(c+a)} + \frac{a^3b^3}{(a+b)}\ge\frac32
  \end{align*}

  由对称性,不妨设$a\le b\le c$,从而有
  \begin{align*}
    \begin{array}{*5{>{\displaystyle}c}}
      ab & \le & ca & \le & bc  \\[5pt]
      \frac1{b+c} & \le & \frac1{c+a} & \le & \frac1{a+b}
    \end{array}
  \end{align*}
  如果要应用排序不等式的话,原不等式的左边对应于“反序和”,是最小的,得不到$\ge$的结果。此路不通。

  关键点:做变换代换。令
  \begin{align*}
    x\equiv\frac1a,\quad y\equiv\frac1b,\quad z\equiv\frac1c
  \end{align*}
  则有$xyz=1$,且
  \begin{align*}
    \frac1{a^3(b+c)} + \frac1{b^3(c+a)} + \frac1{c^3(a+b)} =&
    \frac{x^3}{\frac1y+\frac1z} + \frac{y^3}{\frac1z+\frac1x} + \frac{z^3}{\frac1x+\frac1y}\\
    =&\frac{x^2}{y+z}+\frac{y^2}{z+x}+\frac{z^2}{x+y}
  \end{align*}
  由对称性,不妨设$x\le y\le z$(此处已经抛弃了前面$a\le b\le c$的假设,若仍采用的话,则应有$z\le y\le x$),则有
  \begin{align*}
    \begin{array}{*5{>{\displaystyle}c}}
      x & \le & y & \le & z\\
      \frac{x}{y+z} & \le & \frac{y}{z+x} & \le & \frac{z}{x+y}
    \end{array}
  \end{align*}
  应用排序不等式,有
  \begin{align*}
    \frac{x^2}{y+z}+\frac{y^2}{z+x}+\frac{z^2}{x+y} \ge & \frac{xy}{y+z}+\frac{yz}{z+x}+\frac{zx}{x+y}\\
    \frac{x^2}{y+z}+\frac{y^2}{z+x}+\frac{z^2}{x+y} \ge & \frac{xz}{y+z}+\frac{yx}{z+x}+\frac{zy}{x+y}
  \end{align*}
  两式相加,有
  \begin{align*}
    \frac{x^2}{y+z}+\frac{y^2}{z+x}+\frac{z^2}{x+y} \ge \frac{x+y+z}{2}\ge\frac{\sqrt[3]{xyz}}{2}=\frac32&\qedhere
  \end{align*}
\end{proof}

\begin{example}[切比雪夫不等式,Chebyshev's Sum Inequality]
  若$a_i$与$b_i$是两递增数列,则
  \begin{align*}
    \frac1n\sum_{i=1}^n a_ib_i\ge
    \left( \frac1n\sum_{i=1}^n a_i \right)
    \left( \frac1n\sum_{i=1}^n b_i \right)
  \end{align*}
  当$a_1=a_2=\cdots a_n$或者$b_1=b_2=\cdots b_n$时等号成立。
\end{example}
\begin{proof}[提示]
  应用$n$次排序不等式,有
  \begin{align*}
    a_1b_1 + \cdots + a_nb_n = &\,   a_1b_1 + a_2b_2 + a_3b_3 + \cdots + a_nb_n\\
    a_1b_1 + \cdots + a_nb_n \ge&\,  a_1b_2 + a_2b_3 + a_3b_4 + \cdots + a_nb_1\\
    a_1b_1 + \cdots + a_nb_n \ge&\,  a_1b_3 + a_2b_4 + a_3b_5 + \cdots + a_nb_2\\
    \vdots\\
    a_1b_1 + \cdots + a_nb_n \ge&\,  a_1b_n + a_2b_1 + a_3b_2 + \cdots + a_nb_{n-1}
  \end{align*}
  将此$n$个式子相加可得。
\end{proof}

\begin{example}[切比雪夫不等式的积分形式]
  若$f,g$同为单调增函数或同为单调减函数,则有
  \begin{align*}
    \int f(x)g(x)\mathrm{d}x\ge\int f(x)\dx\int g(x)\dx
  \end{align*}
\end{example}