
\chapter{数论}
\label{chap:number-theory}

\epigraph{Die ganzen Zahlen hat der liebe Gott gemacht, alles andere ist Menschenwerk.\\
  上帝创造了自然数,其余的都是人的工作。}{Leopold Kronecker(利奥波德·克罗内克)}

Kronecker是德国数学家与逻辑学家,主要研究代数和数论,特别是在椭圆函数理论上有突出贡献。以克罗内克命名的数学理论包括克罗内克$\delta$函数、克罗内克积等。

\section{自然数}
\label{sec:what-is-natural-number}

关于自然数(Natural Number)所指,目前并没有定论,有时是指正整数$1,2,3,\cdots$,有时是指非负整数$0,1,2,3,\cdots$,取决于主观意愿。在这里,为避免混淆,只采用正整数、非负整数等无疑义的说法。

\section{质数}
\label{sec:prime-number}

\begin{definition}[质数,Prime Number]
  一个大于$1$的正整数,若除了$1$和它本身之外没有其它的因子,则称该数为质数,也叫素数。不是质数的正整数称为合数。
\end{definition}

\begin{table}[htbp]
  \centering
  \begin{tabular}{cccccccccc}
    \hline
    2  & 3  & 5  & 7  & 11 & 13 & 17 & 19 & 23 & 29\\
    31 & 37 & 41 & 43 & 47 & 53 & 59 & 61 & 67 & 71\\
    73 & 79 & 83 & 89 & 97 &    &    &    &    &   \\
    \hline
  \end{tabular}
  \caption{100以内的质数}
  \label{tab:prime-numbers<100}
\end{table}

\begin{property}
  $2$是唯一的一个偶数质数。
\end{property}

\begin{theorem}
  有无穷多个质数。
\end{theorem}
\begin{proof}
  下面是欧拉提出的方法,用反证。假设只有有限个质数$p_1,p_2,\cdots,p_n$,那么对于正整数$p_{n+1}=p_1p_2p_3\cdots p_n + 1$,显然有$p_{n+1}$不被$p_1,p_2,\cdots,p_n$整除,从而找到一个$p_1,p_2,\cdots,p_n$之外且比它们都大的质数$p_{n+1}$,矛盾。
\end{proof}

可以用欧拉的反证法构造质数数列。从质数数列$p_1,p_2,\cdots,p_n$出发,则$p_1p_2\cdots p_n+1$要不是质数,要不就含有一个不同于$p_1,p_2,\cdots,p_n$的质因子,从而可找到下一个质数$p_{n+1}$。

\begin{example}
  从$41$出发,按欧拉方法找出$5$个不同的质数。
\end{example}
\begin{proof}[解]千万不要用纸笔算,除非你想锻炼计算能力,请利用计算机编程验证,让合适的人做合适的事。
  \begin{align*}
    p_1 & = 41 && p_1 + 1 = 42 = 2\times 3\times 7&&\text{随便取一个,比如取2}\\
    p_2 & = 2  && p_1p_2 + 1 = 2\times41 + 1=83   &&\text{是质数}\\
    p_3 & = 83 && p_1p_2p_3 + 1=6807=3\times2269  &&\text{为方便起见,取3}\\
    p_4 & = 3  && p_1p_2p_3p_4+1=20419=7\times2917&&\text{取7}\\
    p_5 & = 7  &&                                 &&\qedhere
  \end{align*}
\end{proof}

下面是除了著名的哥德巴赫猜想(参考例~\ref{ex:Goldbach-conjecture})之外的又一个关于质数的悬而未决的猜想。
\begin{example}[成对质数的猜想]
  以$p$与$p+2$形式成对出现的质数有无穷多个。
\end{example}
\begin{proof}[示例]
  如$(3,5)$、$(11,13)$、$(17,19)$、$(29,31)$和$(41,43)$等等都是这种成对形式的质数。
\end{proof}

\section{整除}
\label{sec:divisible}

\begin{definition}
  若整数$a$是整数$b$的整数倍,即存在某个整数$q$,使得$a=bq$,则称$b$整除$a$,也称$a$被$b$整除。记为$b\mid a$。$b$不能整除$a$则记为$b\notdivides a$。
\end{definition}

\begin{example}
  \begin{align*}
    17293\mid 0,\quad 4\mid 28,\quad 5\notdivides 12
  \end{align*}
\end{example}

\subsection{最大公约数}
\label{sec:gcd}

\begin{definition}[最大公约数,Greatest Common Divisor(gcd)]
  两个不同时为零的整数$a,b$的公约数中最大的一个称为$a,b$的最大公约数,记为$\gcd(a,b)$。
\end{definition}

\begin{example}
  \begin{align*}
    \gcd(0,8)=8,\quad 
    \gcd(-9,-12)=\gcd(9,12)=3,\quad
    \gcd(7,22)=1
  \end{align*}
\end{example}

\begin{definition}[互质,Coprime]
  若$\gcd(a,b)=1$,则称$a,b$互质。
\end{definition}
$\gcd(a,b)=1$是指$a,b$除了$1$之外没有其余的正的公约数。

\begin{theorem}
  若$a,b$互质,且质数$p\mid ab$,则必有$p\mid a$或者$p\mid b$。
\end{theorem}
\begin{corollary}
  若$a_1,a_2,\cdots,a_n$两两互质,且质数$p\mid a_1a_2\cdots a_n$,则$p$必能整除$a_1,a_2,\cdots, a_n$中的某一个。
\end{corollary}

\begin{example}
  $4,5,9$两两互质,其乘积为$4\times5\times9=180$,作质因式分解,有
  \begin{align*}
    180=2^2\times 3^2\times 5
  \end{align*}
  从而能整除$180$的质数只有$2,3,5$,这三个数都能整除$4,5,9$中的某一个。
\end{example}

\begin{theorem}
  若整数$a,b$互质,且$b\mid ac$,则必有$b\mid c.$
\end{theorem}

\begin{definition}[欧拉函数,Euler Function]\label{def:Euler-function}
  对于任意正整数$n$,欧拉函数$\varphi(n)$表示从$1$到$n$中与$n$互质的整数的个数。
\end{definition}

\begin{table}[htbp]
  \centering
  \begin{tabular}{cccccccccccccccc}
    \toprule
    $n$          & 1 & 2 & 3 & 4 & 5 & 6 & 7 & 8 & 9 & 10 & 11 & 12 & 13 & 14 & $\cdots$\\\midrule
    $\varphi(n)$ & 1 & 1 & 2 & 2 & 4 & 2 & 6 & 4 & 6 & 4  & 10 & 4  & 12 & 6  & $\cdots$\\
    \bottomrule
  \end{tabular}
  \caption{欧拉函数表}
  \label{tab:Euler-function-values}
\end{table}

\begin{theorem}
  若正整数$n$是质数,则$\varphi(n)=n-1$,若正整数$n$是合数,且其质因数分解式为
  \begin{align*}
    n=p_1^{a_1} p_2^{a_2} p_3^{a_3} \cdots p_k^{a_k}
  \end{align*}
  其中$p_1,p_2,p_3,\cdots,p_k$是互不相等的质数,则
  \begin{align*}
    \varphi(n)=n\left(1-\frac1{p_1}\right) \left(1-\frac1{p_2}\right) \left(1-\frac1{p_3}\right) \cdots \left(1-\frac1{p_k}\right)
  \end{align*}
\end{theorem}

\begin{example}[求$\varphi(15)$]
  由于$15=3\times 5$,在$1,2,3,\cdots,15$中,不是$3$的倍数的占$1-1/3=2/3$,不是$5$的倍数的占$1-1/5=4/5$。
  \begin{align*}\setlength\arraycolsep{3pt}\renewcommand*{\arraystretch}{.9}
    \begin{array}{cccccccccccccccc}
                         & 1 & 2 & 3 & 4 & 5 & 6 & 7 & 8 & 9 & 10 & 11 & 12 & 13 & 14 & 15\\
    \text{排除$3x$}\quad & 1 & 2 & \cancel3 & 4 & 5 & \cancel6 & 7 & 8 & \cancel9 & 10 & 11 & \cancel{12} & 13 & 14 & \cancel{15}\\
    \text{排除$5x$}\quad & 1 & 2 & 3 & 4 & \xcancel5 & 6 & 7 & 8 & 9 & \xcancel{10} & 11 & 12 & 13 & 14 & \xcancel{15}
    \end{array}
  \end{align*}
  从而
  \begin{align*}
    \varphi(15)=15\times\left(1-\frac13\right)\times\left(1-\frac15\right)
    =15\times\frac23\times\frac45=8
  \end{align*}
\end{example}

\begin{definition}[非负整数系数线性组合]
  给定$x_1, x_2, \cdots, x_n$,对任意非负整数$a_1,a_2,\cdots, a_n$,称
  \begin{align*}
    a_1 x_1 + a_2 x_2 + \cdots + a_3 x_n
  \end{align*}
  为$x_1, x_2, \cdots, x_n$的非负整数系数线性组合。
\end{definition}

\begin{example}[硬币问题,Frobenius coin problem]
  给定若干种面额的硬币,每种面额的硬币数量足够多,那么用这些硬币所不能表达的最大的整数价值是多少?如用$2$元和$5$元的硬币,所不能表示的整数价值有$1$元、$3$元;偶数都可以用若干$2$元的表示,大于或等于$5$的奇数都可以用一个$5$以及若干个$2$表示,从而$3$就是$2,5$的非负整数系数线性组合里不包含的最大整数。

  Frobenius硬币问题的数学提法如下:

  给定正整数$a_1, a_2, \cdots, a_n$且$\gcd(a_1, a_2, \cdots, a_n)=1$,求这些正整数的非负整数系数线性组合所不包含的整数的最大值。这个最大整数就称为$a_1, a_2,\cdots, a_n$的Frobenius Number,通常记为$g(a_1,a_2,\cdots, a_n)$。
\end{example}

\begin{example}[麦乐鸡鸡块问题,McNugget Numbers]
  $80$年代时,英国的麦当劳售卖三种规格的麦乐鸡块盒,每种盒子里分别有$6,9,20$块鸡块。有些数量的鸡块能通过购买三种鸡块盒得到,有些数量则不可以。问不能通过购买这三种鸡块盒得到的鸡块数量的最大值是什么?
\end{example}

\begin{theorem}
  若正整数$m,n$互质,则$m,n$的非负整数系数线性组合所不包含的正整数的个数为
  \begin{align*}
    \dfrac{(m-1)(n-1)}2
  \end{align*}
  且其Frobenius数$g(m,n)=mn-m-n$。
\end{theorem}
\begin{proof}下面简称能被$m,n$的非负整数系数线性组合所表示的正整数为可表示的,否则称为不可表示的。
  
  首先证明任意整数$k\ge mn-m-n+1=(m-1)(n-1)$是可表示的。可由定理~\ref{th:Skupien}的证明可得。亦可利用贝祖定理~\ref{th:Bezout}及二元一次不定方程的理论得到,引用的地方比较多,看看就好。由$m,n$互质及贝祖定理,存在整数$a,b$使得$ma+nb=1$。从而$m\times(ak)+n\times(bk)=k$。从而对于关于$x,y$的二元一次不定方程$mx+ny=k$的所有解为$x=ak+tn, y=bk-tm$,其中$t$是任意整数。选择$t$,使得$x\in[0,n-1)$,则对于此$t$,有
  \begin{align*}
    &mx + ny = k > mn -m -n \\
    \implies\quad& n(y+1)>m(n-1-x)>0\\
    % \overset{x\in[0,n-1)}{\implies}\quad&
                                            \implies\quad&y+1>0\\
    \implies\quad& y\ge0    
  \end{align*}
  从而此$t$使得$x,y$均非负且满足$mx+ny=k$,即$k$可表示。


  其次证明$mn-m-n\equiv (m-1)(n-1)-1$是不可表示的。反证。假设存在非负整数$x,y$,使得
  \begin{align*}
    mn-m-n=mx+ny
  \end{align*}
  两边分别对$m,n$取模,则有
  \begin{align*}
    \begin{cases}
      -n&\equiv ny\pmod m\\
      -m&\equiv mx\pmod n
    \end{cases}
    \underset{m,n\text{互质}}{\implies}
    \begin{cases}
      y\equiv -1\equiv m-1 \pmod m\\
      x\equiv -1\equiv n-1 \pmod n
    \end{cases}
  \end{align*}
  再由$x,y$的非负性,有$y\ge m-1, x\ge n-1$,从而
  \begin{align*}
    mx + ny\ge m(n-1) + n(m-1) = 2mn -m -n > mn - m -n
  \end{align*}
  矛盾。从而$mn-m-n$不可表示。

  最后证明不可表示的正整数个数为$(m-1)(n-1)/2$。这需要用到以下事实:对于任意的整数$k\in[0,g(m,n)]$,$(k, g(m,n)-k)$这两个数中总是只有一个能被表示。由于两个可表示的数之和是可表示的,而$k + g(m,n)-k = g(m,n)$是不可表示的,从而$(k,g(m,n)-k)$这两数中最多只有一个是可表示的,从而至少有一个是不可表示的。设$k$是不可表示的,剩余需要证明$g(m,n)-k$是可表示的,从而两个只恰好有一个是可表示的一个是不可表示的。由前面证明所述,关于$x,y$的二元一次不定方程$mx+ny=k$的所有解都可以表示为$x=a+tn,y=b-tm$,选择$t$,使得$x\in[0,n-1)$。由于$k$是不可表示的,从而此$t$对应的$y<0\implies y\le-1$,否则$x,y$非负,$k=mx+ny$可表示。因此有
  \begin{align*}
    g(m,n)-k=mn-m-n-mx-my=m(n-1-x)+n(-y-1)
  \end{align*}
  由$x\in[0,n-1)$及$y\le -1$可知$n-1-x\ge0, -y-1\ge0$,从而$g(m,n)-k$可表示。

  结合起来,有$g(m,n)=(m-1)(n-1)$,且不可被表示的正整数有$k/2=(m-1)(n-1)/2$个。。
\end{proof}

\begin{theorem}[Skupi\'en的推广]\label{th:Skupien}
  若正整数$m,n$互质,则对任意正整数$k \ge (m-1)(n-1)$,存在唯一的非负整数$\alpha,\beta$,使得$\alpha < n$且$k =\alpha m + \beta n$。
\end{theorem}
\begin{proof}
  存在性。由定理~\ref{th:coprime-modular},$\{im: 0\le i< n\}=\{0,m,2m,\cdots,(n-1)m\}$这$n$个数模$n$两两不同余,故对任意整数$k\ge (m-1)(n-1)$,$k,k-m,k-2m,k-3m,\cdots,k-(n-1)m$模$n$两两不同余,从而其中某一个模$n$为$0$,不妨记$k-\alpha m\equiv 0\pmod n$,其中$0\le \alpha \le n-1$,从而存在整数$\beta$,使得$k-\alpha m = \beta n$,即$k=\alpha m + \beta n$。

  非负性。
  由上面构造方法,已有$0\le\alpha<n$,且$\beta n = k-\alpha m\ge (m-1)(n-1)-(n-1)m=1-n$,从而$(\beta+1)n\ge1\implies\beta\ge0$。

  唯一性。给定$k\ge (m-1)(n-1)$,若存在非负整数$\alpha<n,a<n$及非负整数$\beta, b$,使得$k=\alpha m + \beta n = am + bn$,则$(\alpha -a)m=(b-\beta)n$,由$m,n$互质,可知$m\mid b-\beta, n\mid\alpha -a$。但$\alpha<n, a<n$,从而$-n<\alpha-a<n$,从而只有$\alpha = a$才能有$n\mid \alpha -a$,继而$\beta = b$。
\end{proof}

\subsection{最小公倍数}
\label{sec:lcm}

\begin{definition}[最小公倍数,Least Common Multiple(lcm)]
  两个非零整数$a,b$的正的公倍数中最小的一个称为$a,b$的最小公倍数,记为$\lcm(a,b)$。
\end{definition}

\begin{example}
  \begin{align*}
    \lcm(3,7)=21,\quad \lcm(4,6)=12,\quad \lcm(-6,9)=\lcm(6,9)=18
  \end{align*}
\end{example}

\begin{theorem}
  任意两个非零整数$a,b$,有$\gcd(a,b)\cdot\lcm(a,b)=ab.$
\end{theorem}
\begin{proof}[提示]
  记$s\equiv ab/\gcd(a,b)$,则首先$s$是$a,b$的公约数,其次用反证法证明$s$是$a,b$所有正公约数中最小的一个,从而$s=\lcm(a,b)$。
\end{proof}

\section{同余}
\label{sec:modular}

\begin{example}[模5同余]
  一个整数被$5$除时,余数只能是$0,1,2,3,4$这$5$种之一。可以把所有的整数按除$5$的余数分类,则可分为$5$类:
  \begin{align*}\renewcommand*{\arraystretch}{.9}\setlength\arraycolsep{4pt}
    \begin{array}{cccccccccc}
      \text{第1类,余数为0:} & \quad \cdots, & -15, & -10, & -5, & 0, & 5, & 10, & 15, & \cdots\\
      \text{第2类,余数为1:} & \quad \cdots, & -14, & -9,  & -4, & 1, & 6, & 11, & 16, & \cdots\\
      \text{第3类,余数为2:} & \quad \cdots, & -13, & -8,  & -3, & 2, & 7, & 12, & 17, & \cdots\\
      \text{第4类,余数为3:} & \quad \cdots, & -12, & -7,  & -2, & 3, & 8, & 13, & 18, & \cdots\\
      \text{第5类,余数为4:} & \quad \cdots, & -11, & -6,  & -1, & 4, & 9, & 14, & 19, & \cdots
    \end{array}
  \end{align*}
  每一类中任意两数都称为模$5$同余。
\end{example}

类似的,对任意整数$d$,可得到模$d$同余的概念。容易得到,以下关于两整数$a$和$b$模$d$同余的定义是等价的:
\begin{enumerate}
\item $a$、$b$模$d$的余数相同;
\item $a-b$能被$d$整除;
\item 存在某个整数$n$使得$a=b+nd$。
\end{enumerate}

一般使用高斯所创记法表示同余关系,即用
\begin{align*}
  a \equiv b \pmod d \quad\quad a\not\equiv b\pmod d
\end{align*}
分别表示$a$和$b$模$d$同余、$a$和$b$模$d$不同余。

\subsection{整除的性质}
\begin{example}[被$7$整除]\label{ex:divided-by-7}
  任意一个整数$n$,可以唯一地用若干个正整数$a_i$表示为
  \begin{align*}
    n=\sum_{i=0}^{k} a_i\times 10^i
  \end{align*}
  其中$0\le a_i<10$,如
  \begin{align*}
    32568 = 8 + 6\times 10 + 5\times 10^2 + 2\times 10^3 + 3\times10^4
  \end{align*}
  观察$10^i$模$7$的余数,可得表~\ref{tab:10^i-modular-7}。

  \begin{table}[htbp]
    \centering
    \begin{tabular}{cccccccccc}
      \toprule
                & $10^0$ & $10^1$ & $10^2$ & $10^3$ & $10^4$ & $10^5$ & $10^6$ & $10^7$ & $\cdots$ \\ \midrule
      模$7$余数 & $1$    & $3$    & $2$    & $6$    & $4$    & $5$    & $1$    & $3$    & $\cdots$ \\ 
      \bottomrule
    \end{tabular}
    \caption{$10^i$模$7$余数}
    \label{tab:10^i-modular-7}
  \end{table}

  由于$10\not\equiv0\pmod7$,所以$10^i$模$7$的余数永不为零,表格中只会出现$1,2,3,4,5,6$这$6$个余数,且每$6$个一循环。从而
  \begin{align*}
    n\equiv\,&\sum_{i=0}^{k} a_i\times10^i\tag*{$\pmod7$}\\
     \equiv\,& a_0\times10^0 + a_1\times10^1 + a_2\times10^2 + a_3\times10^3 + a_4\times10^4 + a_5\times10^5 +\\
             & a_6\times10^6 + a_7\times10^7 + a_8\times10^8 + \cdots \tag*{$\pmod7$}\\
     \equiv\,& a_0 + 3a_1 + 2a_2 + 6a_3 + 4a_4 + 5a_5 +\\
             & a_6 + 3a_7 + 2a_8 + 6a_9 + \cdots\tag*{$\pmod7$}
  \end{align*}
  所以若$n=\sum\limits_{i=0}a_i\times10^i$能被$7$整除,当且仅当
  $a_0 + 3a_1 + 2a_2 + 6a_3 + 4a_4 + 5a_5 + a_6 + 3a_7 + 2a_8 + 6a_9 + \cdots$
  能被$7$整除。
\end{example}

\begin{example}
  整数$782370456$能否被$7$整除?
\end{example}
\begin{proof}[提示]
  将大于$7$的数字减去$7$,原数能否被$7$整除等价于$012300456=12300456$能否被$7$整除。按下面表格
  \begin{align*}\renewcommand{\arraystretch}{.9}
    \begin{array}{ccccccccc}
      \toprule
      i                    & 0 & 1 & 2 & 3 & 4 & 5 & 6 & 7\\ \midrule
      \text{数字倒序}      & 6 & 5 & 4 & 0 & 0 & 3 & 2 & 1\\
      \text{$10^i$模7余数} & 1 & 3 & 2 & 6 & 4 & 5 & 1 & 3\\
      \bottomrule
    \end{array}
  \end{align*}
  将上下两行对应数字相乘并求和,则原数能否被$7$整除等价于$6\times1 + 5\times3 + 4\times2 + 0\times6 +0\times4 + 3\times5 + 2\times1 + 1\times3 = 49$能否被$7$整除。
\end{proof}

\begin{question}[被3或9整除]
  用类似的方式,证明一个正整数若能被$3$(或$9$)整除,等价于其数字之和能被$3$(或$9$)整除。
\end{question}
\begin{proof}[提示]
  任意非负整数$i$,有$10^i\equiv 1\pmod3$,$10^i\equiv1\pmod9$。
\end{proof}

\begin{theorem}\label{th:coprime-modular}
  若正整数$a,b$互质,则$0,a,2a,3a,\cdots,(b-1)a$模$b$两两不同余。
\end{theorem}
\begin{proof}
  反证。否则存在$0\le i<j$,使得$ia\equiv ja\pmod b$,从而$b\mid (j-i)a > 0$,而$j-i<b$,从而有$j-i$与$b$互质,因而有$b\mid a$,这与$a,b$互质矛盾。
\end{proof}

\subsection{同余的运算}
\label{sec:op-of-modular}

在例~\ref{ex:divided-by-7}中用到了同余式的以下性质。若$a\equiv\alpha\pmod d, b\equiv\beta\pmod d$,则对任意整数$u,v$,有
\begin{enumerate}
\item $au+bv\equiv \alpha u+\beta v\pmod d$;
\item $ab\equiv \alpha\beta\pmod d$;
% \item $u^a\equiv u^{\alpha}\pmod d$;
\item $a^u\equiv \alpha^{u}\pmod d$。
\end{enumerate}

%% 其实由第2条性质可以直接得到第3条,取$b=a,\beta=\alpha$代入,有$a^2\equiv\alpha^2\pmod d$,然后取$a\to a^2,\alpha\to\alpha^2,b=a,\beta=\alpha$代入,又有$a^3\equiv\alpha^3\pmod d$,依次类推。

注意,$u^a\equiv u^{\alpha}\pmod d$则不一定成立,如$2\equiv9\pmod7$,但取$u=3$,则$3^2\equiv2\not\equiv6\equiv 3^9\pmod7$。

\begin{question}
  求$12\times88\times43\times1988$模$7$的余数。
\end{question}
\begin{proof}[提示]
  \begin{align*}\renewcommand{\arraystretch}{.9}\setlength\arraycolsep{2pt}
    \begin{array}{ccccccccl}
             & 12 & \times & 88 & \times & 43 & \times & 1989 &\\
      \equiv & 5  & \times & 11 & \times & 1  & \times & 1212 &\quad\pmod7\\
      \equiv & 5  & \times & 4  & \times & 1  & \times & 512  &\quad\pmod7\\
      \equiv & 5  & \times & 4  & \times & 1  & \times & 22   &\quad\pmod7\\
      \equiv & 5  & \times & 4  & \times & 1  & \times & 1    &\quad\pmod7
    \end{array}
  \end{align*}
  上述变化中,$88\to 11, 1989\to1212$是将每个大于7的数字减7;$11\to4$是整体减7;$1212\to512,512\to22$是将最前面的几位(这里是两位)减去7的倍数,分别是$12-7=5, 51-49=2$。
\end{proof}

\subsection{费马小定理}
\label{sec:Fermat-theorem}

\begin{theorem}[费马小定理,Fermat's Little Theorem]
  给定质数$p$,则对任意的正整数$a$,有$a^p\equiv a\pmod p$;若$p\notdivides a$,则同时有$a^{p-1}\equiv1\pmod p$。
\end{theorem}
\begin{proof}
  若$p|a$,即$a\equiv0\pmod p$,则显然有$a^p\equiv a\equiv0\pmod p$。

  对于$a\notdivides p$,考虑$a,2a,3a,\cdots,(p-1)a$这$p-1$个数。可以证明其中任意两个不模$p$不同余,否则两个的差$(s-t)a$能被$p$整除,其中$1\le s<t\le p-1$不能被$p$整除,再由$p$是质数从而必须有$a$被$p$整除,矛盾。于是在模$p$同余下,这$p-1$个数必与$1,2,3,\cdots,p-1$重新排列后一一对应,从而有
  \begin{align*}
    & a \times 2a \times 3a \times \cdots \times (p-1)a
    \equiv 1 \times 2 \times 3 \times \cdots \times (p-1) \pmod p\\
    \implies & (1\times 2\times 3\times\cdots\times (p-1))(a^{p-1}-1)\equiv0\pmod p
  \end{align*}
  由于$p$不能整除$1,2,3,\cdots,p-1$中任意一个,且$p$为质数,从而$p$整除$a^{p-1}-1$,即$a^{p-1}\equiv1\pmod p$,从而也有$a^p\equiv a\pmod p$。
\end{proof}

\begin{corollary}
  若正整数$a,b$互质,则$a^{\varphi(b)}\equiv1\pmod b$,其中$\varphi$是\ref{def:Euler-function}中的欧拉函数。
\end{corollary}


\subsection{辗转相除法}
\label{sec:Euclidean-algorithm}

\begin{theorem}\label{th:euclidean-algorithm-theorem}
  若整数$a,b,q,r$满足$a=bq+r$,即$a\equiv r\pmod b$,则$\gcd(a,b)=\gcd(b,r).$
\end{theorem}
\begin{proof}[提示]
  任意同时整除$a,b$的数同时也能整除$r$;反之,任意同时整除$b,r$的数同时也能整除$a$。
\end{proof}

\begin{definition}[辗转相除法]
  对任意两正整数$a,b$,求$\gcd(a,b)$。若$a=b$,则$\gcd(a,b)=\gcd(a,a)=a$。下面不妨设$a>b$,从而由定理~\ref{th:euclidean-algorithm-theorem},有
  \begin{align*}
    a&=  \,bq_1 + r_1,   &&\quad(0\le r_1 < b)   & \gcd(a,b)    &=\,\gcd(b,r)\\
    b&=  \,r_1q_2 + r_2, &&\quad(0\le r_2 < r_1) & \gcd(b,r_1)  &=\,\gcd(r_1, r_2)\\ 
    r_1&=\,r_2q_3 + r_3, &&\quad(0\le r_3 < r_2) & \gcd(r_1,r_2)&=\,\gcd(r_2, r_3)\\
    r_2&=\,r_3q_4 + r_4, &&\quad(0\le r_4 < r_3) & \gcd(r_2,r_3)&=\,\gcd(r_3, r_4)\\
    \multicolumn{7}{c}{$\cdots$}\\
    r_{n-1}&=\,r_nq_{n+1} + 0, &&\quad(0\le r_n < r_{n-1}) & \gcd(r_{n-1},r_n)&=\,\gcd(r_{n}, 0)=r_{n}
  \end{align*}
  由于$r_i$每次总要减小,所以在有限个步骤后总有某个$r_{n+1}=0$,此时可得$\gcd(a,b)=r_n.$
\end{definition}

\begin{example}
  求$\gcd(1234,678).$
\end{example}
\begin{proof}[提示]在辗转相除法求$\gcd$中,商$q_i$是没有用的,利用计算机中的取模运算\%,可以方便的得到
  \begin{align*}\renewcommand{\arraystretch}{.9}\setlength{\arraycolsep}{2pt}
    \begin{array}{ccccc}
      1234 & \% & 678 & = & 556\\
      678  & \% & 556 & = & 122\\
      556  & \% & 122 & = & 68\\
      122  & \% & 68  & = & 54\\
      68   & \% & 54  & = & 14\\
      54   & \% & 14  & = & 12\\
      14   & \% & 12  & = & 2\\
      12   & \% & \textcircled{2}   & = & 0
    \end{array}
  \end{align*}
  所以$\gcd(1234,678)=2$。
\end{proof}

\section{整数分类}
\label{sec:category}

按不同的分类方法,整数可以分为不同的集合。按同余分类是一种比较常见的分类方法,如按模$2$的余数分类,可分为奇数和偶数。再比如在研究幻方中则通常将幻方(参考\ref{chap:magic-square})的阶数按$2n+1, 4n, 4n+2$的形式来分类。

\begin{example}
  将正整数分成若干子集,使得任意正整数$x$与$2x$不在同一个子集内。则最少只需要两个子集。
\end{example}
\begin{proof}
  按大小顺序依次安放各数,奇数可以随便放,因其不是任意整数的两倍。放偶数$n$时,只要放入$n/2$所在的另一个子集即可。

  另一种方法,任意正整数可以唯一地写为$2^km$的形式,其中$m$是正奇数,$k$是非负整数。若$n_1=2^{k_1}m_1 = 2n_2=2\cdot 2^{k_2}m_2$,则$k_1=k_2+1$其奇偶性不同。将正整数按上述分解中$k$的奇偶性分成两类,每类作一个子集即可。
\end{proof}

\section{重要定理}
\label{sec:important-thorems-of-number-theory}

