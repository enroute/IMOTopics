
\chapter{不动点}
\label{chap:fixed-point}

本节内容需要拓扑与泛函分析的知识,不需深究。

\begin{example}
  两张内容一样的地图,把小地图放在大地图上,使小地图完全在大地图内,则必有一点,使两张地图上对应的点代表同一个物理位置。
\end{example}

此问题也有另外一种描述方式:把一张地图放在地上,则地图上必有一点正好在它所代表的物理位置的正上方。在这种描述里,大地图对应于所在城市/国家/甚至地球,小地图则对应于地上的地图。

要回答这问题,需要用到\term{不动点}理论。点$x$称为是映射$f$的不动点,若$f(x)=x$,即对不动点映射过后的像不动,还是落在自身。不动点定理表明在某些条件下,映射至少有一个不动点。

\begin{definition}[压缩映射]
  距离空间$\mathcal{D}$上的映射$f:\mathcal{D}\to\mathcal{D}$称为是压缩映射,若存在$q\in[0,1)$,使得任意$x,y$有
  \begin{align}
    d(f(x),f(y))\le q d(x,y)
  \end{align}
\end{definition}

\begin{theorem}[巴拿赫不动点定理,压缩不动点定理,Banach fixed-point theorem]
  完备距离空间上压缩映射存在唯一的不动点$x^*$,且由任意一点$x_0$通过以下迭代:
  \begin{align*}
    x_{n}=f(x_{n-1}),\quad\quad n=1,2,3,\cdots
  \end{align*}
  有$x_n\to x^*$。
\end{theorem}
\begin{proof}
  需要用到拓扑与泛函分析的知识,此处略过。
\end{proof}

回到大小地图的问题,在大小地图所在平面上建立坐标系,考虑将大地图上的点对应的坐标$x$映射到小地图上的点对应的坐标$x'$的映射$f$,其中$x$与$x'$在大小地图上表示同一个物理位置,则容易知道$f$是一个压缩映射,因为每两个点的距离都被压缩了。从而由巴拿赫定理,存在唯一的不动点。