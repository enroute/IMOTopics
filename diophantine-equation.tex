
\chapter{不定方程}
\label{chap:diophantine-equation}

\epigraph{三人同行七十稀,五树梅花廿一支,\\七子团圆正半月,除百零五便得知。}{明·程大位《算法统宗》}

一般来说,求解$n$个未知变量,需要$n$个条件。
不定方程是指约束个数少于未知数个数的方程(组)。由于约束少,不定方程通常有无穷多的解。

这里只讨论整数系数不定方程的整数解问题。早在三世纪初,古希腊数学家丢番图(Diophantus)就研究过若干此类方程,所以不定方程也称为丢番图方程(Diophantine equation)。

\section{线性不定方程}
\label{sec:linear-diophantine-equation}

\begin{theorem}
  若$\gcd(a,b)\notdivides c$,则不定方程$ax+by=c$无整数解。
\end{theorem}
\begin{proof}
  反证。假设存在整数解$x=x_0, y=y_0$,由于存在整数$a_0,b_0$满足$a=a_0\cdot\gcd(a,b), b=b_0\cdot\gcd(a,b)$,从而有
  \begin{align*}
    &c=a_0\cdot\gcd(a,b)x_0 + b_0\cdot\gcd(a,b)y_0=\gcd(a,b)(a_0x_0+b_0y_0)
  \end{align*}
  从而有$\gcd(a,b)\mid c$,矛盾。
\end{proof}

\begin{theorem}
  若不定方程$ax+by=c$有一个整数解$x_0, y_0$,则不定方程的通解(指所有整数解)为
  \begin{align*}
    \begin{cases}
      x=x_0+k\cdot\gcd(a,b)\\
      y=y_0-k\cdot\gcd(a,b)
    \end{cases}, \quad k\in\mathcal{Z}
  \end{align*}
\end{theorem}
\begin{proof}
  首先容易验证$x=x_0+kb, y=y_0-ka$是不定方程的整数解。其次,对不定方程的任一整数解$x=x', y=y'$,由
  \begin{align*}
    ax_0 + by_0 =c, \quad ax' + by' =c
  \end{align*}
  两式相减,则有$a(x_0-x')=b(y'-y_0)$。记$a_0=a/\gcd(a,b), b_0=b/\gcd(a,b)$,则$a_0,b_0$互质,且有$a_0(x_0-x')=b_0(y'-y_0)$,从而$a_0\mid y'-y_0$,从而存在整数$k$,使得$y'-y_0=ka_0$,代入可分别求得$x'=x_0-kb_0, y'=y_0+ka_0$。

  由此定理,要求不定方程的通解,只需找到一组特解即可。
\end{proof}

\begin{example}
  求不定方程$4x+6y=14$的所有整数解。
\end{example}
\begin{proof}[提示]
  首先原不定方程化简为$2x+3y=7$,其系数为$2,3$,由$\gcd(2,3)=1\mid 7$,原不定方程有整数解。将方程变形,有
  \begin{align*}
    2x = 7-3y,\quad\implies\quad x=\frac{7-3y}{2}=3-y+\frac{1-y}{2}
  \end{align*}
  为使$x$是整数,取$y=1$,则$x=2$,此即为特解。其通解按公式可得。
\end{proof}

\begin{example}
  求不定方程$7x+19y=283$的所有整数解。
\end{example}
\begin{proof}[提示]
  首先$\gcd(7,19)=1$,所以有整数解。变换形式如下
  \begin{align*}
    x=\frac{283-19y}{7}=40+\frac{3-19y}{7}=40+\frac{3-21y+2y}{7}=40-3y+\frac{3+2y}{7}
  \end{align*}
  用$y=1,2,3,\cdots$枚举,试出$7\mid 3+2y$的某个$y$。
  \begin{center}
    \begin{tabular}{cccccccccc}
      \toprule
      $y$    & 1 & 2 & 3 & 4  & 5  & 6  & 7  & 8  & 9 \\
      $3+2y$ & 5 & \textcircled{7} & 9 & 11 & 13 & 15 & 17 & 19 & \textcircled{21}\\
      \bottomrule
    \end{tabular}
  \end{center}
  取$y=2$,使$x=40-3*2+1=35$,此即为特解。  
\end{proof}

\section{猴子分果}
\label{sec:monkeys-dividing-fruits}

\begin{example}
  5只猴子去采果子,采完后把果子放到一起,由于太累了便都呼呼大睡了。

  后来,1只猴子先醒来,把果子平均分成5份后发现多了一个,于是它把多的一个扔了,从剩下的5份里拿了一份走了。

  第2只猴子醒了,把剩余的果子平均分成了5份,发现多了2个,于是它把多的2个也扔了,从剩下的5份里拿了一份走了。

  第3只猴子醒了,把剩余的果子平均分成了5份,发现多了3个,于是它把多的3个也扔了,从剩下的5份里拿了一份走了。

  第4只猴子醒了,把剩余的果子平均分成了5份,发现多了4个,于是它把多的4个也扔了,从剩下的5份里拿了一份走了。

  第5只猴子醒了,把剩余的果子平均分成了5份,发现不多不少,它从剩下的5份里拿了一份走了。

  问这些猴子共采了多少个果子?
\end{example}
\begin{proof}
  可以列出不定方程组。记第$i$个猴子拿走了$x_i$个果子,则有
  \begin{align*}
    4x_1 &= 5x_2 + 2, & 4x_2 &= 5x_3 + 3\\
    4x_3 &= 5x_4 + 4, & 4x_4 &= 5x_5
  \end{align*}
  于是总共采到的果子数$s$为
  \begin{align*}
    s&=5x_1 + 1 \\
     &=5\times\frac14 (5x_2+2) + 1 = \frac{25}{4}x_2 + \frac72\\
     &=\frac{25}{4}\times \frac14\times (5x_3+3) + \frac72 = \frac{125}{16}x_3 + \frac{131}{16}\\
     &=\frac{125}{16}\times \frac14 (5x_4+4) + \frac{131}{16} = \frac{625}{64}x_4 + 16\\
     &=\frac{625}{64}\times \frac14\times5x_5 + 16 = \frac{5^5x_5}{4^4} + 16
  \end{align*}
  由于$\gcd(5,4)=1$,从而要使$s$是整数,必有整数$k$,使得$x_5=4^4k$,即$s=5^5k + 16$,且此时有
  \begin{align*}
    x_5& = 4^4 k,&\quad x_4&=\frac54\times 4^4k = 5\times4^3k \\
    x_3&=\frac14(5x_4+4) = 5^2\times4^2k + 1,&\quad x_2 &= \frac14(5x_3+3) = 5^3\times 4k + 2\\
    x_1&=\frac14(5x_2 + 2) = 5^4k + 3
  \end{align*}
  从而猴子们采摘的果子总数可能为
  \begin{align*}
    k&=0 \implies\  s= 5^5\times0 + 16=16\\
    k&=1 \implies\  s= 5^5\times1 + 16=3141\\
    k&=2 \implies\  s= 5^5\times2 + 16=6266&&\qedhere
  \end{align*}
\end{proof}

\begin{example}
  这是丢番图提的一个问题:求三个数,使它们的和及其中任意两数之和都是完全平方数。
\end{example}
\begin{proof}
  丢番图给出过一个解:$(80,320,41)$。实际上这个问题是有无穷多的整数解的。
\end{proof}


\section{的士数}
\label{sec:taxicab-number}

1919年,哈代(Godfrey Harold Hardy)前往看望病重的拉马努金(Srinivasa Ramanujan)。哈代说:“我乘的士来,车牌号码是1729。这个数可真没趣,希望不是不祥之兆。”拉马努金回答说:“不,那是个有趣得很的数。可以用两个立方数之和来表达而且有两种表达方式的数中,1729是最小的。”

\begin{definition}[的士数,Taxicab number]
  能以$n$个不同的方法表示成两个正立方数之和的最小的正整数称为第$n$个的士数,通常记为$\mathrm{Ta}(n)$。
\end{definition}

对于$\mathrm{Ta}(2)$,它实际上是不定方程$x^3+y^3=u^3+v^3$的一个正整数解的立方和。到上前为止,只找到前6个的士数。前三个如表~\ref{tab:taxicab-number}所示。$\mathrm{Ta}(4)$、$\mathrm{Ta}(5)$和$\mathrm{Ta}(6)$过于复杂,此处不列出。
\begin{table}[htbp]
  \centering
  \begin{tabular}{cl}
    \toprule
    $n$              & \multicolumn{1}{c}{$\mathrm{Ta}(n)$}\\\midrule
    1                & $2 = 1^3 + 1^3$\\
    2                & $1729=1^3+12^3=9^3+10^3$\\
    3                & $87539319=167^3+436^3=228^3+423^3=255^3+414^3$\\
    \bottomrule
  \end{tabular}
  \caption{的士数}
  \label{tab:taxicab-number}
\end{table}

\section{中国剩余定理}
\label{sec:Chinese-remainder-theorem}

\begin{definition}[模反元素,Modular multiplicative inverse]
  整数$a,b,m$若满足$ab\equiv1\pmod m$,则称$a,b$是模$m$的模反元素。
\end{definition}

\begin{theorem}[模反的存在性]
  若$a,m$互质,则存在$a$模$m$的模反元素。
\end{theorem}
\begin{proof}[提示]
  考虑$a,2a,3a,\cdots,ma$这$m$个数,由于$a,m$互质,它们模$m$是两两不同余的,否则两个同余的相减会导出$m\mid a$的矛盾。而模$m$的同余类只有$m$个,从而其中必有一个模$m$等于1。
\end{proof}

\begin{example}
  南北朝·孙子《孙子算经》:今有物不知其数,三三数之余二,五五数之余三,七七数之余二。问物有几何?
\end{example}
\begin{proof}[提示]
  与二元线性不方程类似,只要找到一组特解$x$,则所有的整数解都可以表达为$x+t\cdot\lcm(3,5,7)=x+105t$。

  三人同行七十稀。对$3,5,7$,先找一个数$a$,满足$a\equiv 1\pmod3, a\equiv0\pmod5, a\equiv0\pmod7$,由后两个,有$a=5\cdot7\cdot x$,取$x$为$5\cdot 7$为模$3$的模反元素,即$35x\equiv1\pmod3$,可取$x=2$,此时$a=70$。

  五树梅花廿一枝。对$3,5,7$,找一个数$b$,满足$b\equiv0\pmod3, b\equiv1\pmod5,b\equiv0\pmod7$,由模3模7为0,有$b=21y$,取$y$为21模5的模反元素,即$21y\equiv1\pmod5$,可取$y=1$,此时$b=21$。

  七子团圆正半月。对$3,5,7$,找一个数$c$,满足$b\equiv0\pmod3, b\equiv0\pmod5,b\equiv1\pmod7$,由模3模5为0,有$c=15z$,取$z$为15模7的模反元素,即$15z\equiv1\pmod7$,可取$z=1$,此时$c=15$。

  由此可得到一个特解$2\times 70+3\times 21+2\times 15=233$。其所有正整数解为$23,128,233,338,\cdots$。
\end{proof}

\begin{theorem}[孙子定理,中国剩余定理,Chinese remaider theorem]
  大于$1$的正整数$n_1,n_2,\cdots, n_k$两两互质,则对任意整数$a_1,a_2,\cdots,a_k$,方程组
  \begin{align*}
    x\equiv&\ a_1\pmod{n_1}\\
    x\equiv&\ a_2\pmod{n_2}\\
    &\cdots\\
    x\equiv&\ a_k\pmod{n_k}
  \end{align*}
  有正整数解。  
\end{theorem}
\begin{proof}
  用构造方法找出一组特解。取定下标$i$,构造数$x_i$,使得
  \begin{align*}
    \begin{cases}
      x_i\equiv1\pmod{n_i}\\
      x_i\equiv0\pmod{n_j}, \quad\forall j\ne i
    \end{cases}
  \end{align*}
  由后一个条件,$x_i$为如下形式
  \begin{align*}
    x_i=t_i\cdot \underbrace{n_1n_2\cdots n_{i-1}n_{i+1}\cdots n_k}_{\text{没有}n_i}=t_i\cdot \frac{n_1n_2 \cdots n_k}{n_i}
  \end{align*}
  记$M_i\equiv\dfrac{n_1n_2\cdots n_k}{n_i}$,取$t_i$为$M_i$模$n_i$的模反元素(由于$n_i,M_i$互质,$t_i$是存在的),则$x_i=t_iM_i$满足条件。

  对所有下标$i$,构造出$x_i$后,可得一个特解
  \begin{align*}
    x=a_1x_1+a_2x_2+\cdots+a_kx_k&\qedhere
  \end{align*}
\end{proof}

\begin{theorem}
  记$x$是剩余问题的一个特解,则其整数通解为$x+t\cdot n_1n_2\cdots n_k$.
\end{theorem}
\begin{proof}
  设$y$是剩余问题的任一整数解,考虑$y-x$,则对任意下标$i$,有$y-x\equiv0\pmod{n_i}$,而$n_1,n_2,\cdots,n_k$两两互质,从而存在整数$t$,使得$y-x=t\cdot n_1n_2\cdots n_k$。
\end{proof}

