
\chapter{抽屉原理}
\label{chap:pigeonhole-principle}

\section{原理}
\label{sec:pigeonhole-principle-theory}

\begin{theorem}[抽屉原理,鸽笼原理,鸽巢原理,Pigeonhole principle]
  若有$n$个笼子和$kn+1$只鸽子,所有的鸽子都被关在鸽笼里,那么至少有一个笼子有至少$k+1$只鸽子。
\end{theorem}

\begin{proof}
  反证法,若每个笼子最多有不超过$k$个鸽子,那么所有笼子鸽子总数不超过$kn$个,与原来有$kn+1$个鸽子矛盾。
\end{proof}

当$k=1$时,是抽屉原理最直观的情况:若有$n$个笼子和$n+1$只鸽子,所有的鸽子都被关在鸽笼里,那么至少有一个笼子有至少$2$只鸽子,即装有不少于两个鸽子的笼子数最少是$1$。

\think 若上面$n+1$只鸽子换成$n+1,n+2,\cdots$是否成立?对于$2n-1$、$2n$及$2n+1$只鸽子,又可以得到什么样的结论?

\begin{example}
  普遍认为成人的头发数量约在10万左右,那么对于深圳的常住人口(按2018年2180万算),基本可以肯定有人有相同的头发数。
\end{example}

\begin{proof}
  按头发数做鸽笼,按统计数据成人头发平均约为10万左右,假设大部分深圳常住人口(比如2180万的80\%>1600万)的头发数量都小于等于1000万根。按抽屉原理可得。
\end{proof}

\begin{example}
  在$1,2,\cdots,100$里任意选51个数,总能在这选出来的数中找到两个互质数。
\end{example}

\begin{proof}
  将100个数按$(1,2), (3,4), \cdots, (99,100)$分为50组,则选51个数总有两个落到同一组中,而同一组的两数互质,从而得证。
\end{proof}

\begin{question}
  在$1,2,\cdots,100$里任意选51个数,总能在这选出来的数中找到两个,使得其中一个被另一个整除。
\end{question}
\hints $\forall a\in\mathcal{Z}^+$,存在奇数$k$及整数$n$,使得$a=k\cdot 2^n$。对所有整数按$k$分类。

\begin{example}
  在任意9个两两不等的实数中,总能选出两个,设其为$a,b$,满足以下条件
  \begin{align*}
    0 < \frac{a-b}{1+ab} < \sqrt2 -1
  \end{align*}
\end{example}

\begin{proof}[提示]
  联想公式
  \begin{align*}
    \tan(\alpha-\beta)=\frac{\tan\alpha-\tan\beta}{1+\tan\alpha\tan\beta}
  \end{align*}

  而$\tan(x)$的周期是$(-\pi/2, \pi/2]$,将区间8等份(让9个数总有两个落入其中一份)为以下8个区间
  \begin{align*}
    (-\frac\pi2, -\frac{3\pi}8], (-\frac{3\pi}8, -\frac\pi4], \cdots, (\frac{3\pi}8, \frac\pi2]
  \end{align*}

  $\forall a, \exists\hat a\in(-\pi/2, \pi/2], s.t. \tan\hat a = a$,从而任意9个实数,总存在两个,记为$a,b$,存在$\hat a, \hat b$落在上述8个区间中的同一个,且有$a=\tan\hat a, b=\tan\hat b$。不妨设$a>b$,从而$0<\hat a-\hat b<\pi/8$,由$\tan(x)$在上述每个区间中均为严格单调递增,有
  \begin{align*}
             & 0 < \tan(\hat a - \hat b) < \tan\frac\pi8\\
    \implies & 0 < \frac{a - b}{1+ab} < \sqrt2 - 1&\qedhere
  \end{align*}
\end{proof}


\begin{question}[XLI Mathematical Olympiad in Poland]
  一个三角形可以放置在一个单位正方形内,且存在一种放置方法使得正方形的中心不在三角形内。那么该三角形有一条边长小于1。
\end{question}

\hints 将正方形划分几份,使得三角形至少有两个顶点落在某份中。

%%%%%%%%%%%%%%%%%%%%
%%% Questions
%%%%%%%%%%%%%%%%%%%%
\begin{question}
  从$1,2,\cdots,100$中任意取11个数字,那么这11个数字组成的集合中,总存在两个不相交的非空子集,其元素之和相等。
\end{question}
\begin{proof}[提示}
11个元素组成的集合,其非空真子集总共有$2^{11}-2=2046$个。
% 其中两两互补,即任意一个总有能找到一个与之互补。从而将11个元素分成两份且每份都至少有一个元素的方法共有$2046\div2=1023$个
而1到100取10个数字(\think 为什么取10个而不是11个?取11个其和已经最大了,没有真子集的和能与其相比了。),其和最大值为$91+92+\cdots+100=955$。从而由抽屉原理,这2045个子集中,总有两个的和是相等的,且显然这两个和相等的子集,没有一个是另一个的真子集(否则其和就不相等了),在这两个子集中去掉相同的元素,就可以得到两个不相交、非空且和相等的集合。
\end{proof}

\begin{question}
  在三维空间中任意选取9个坐标都是整数的点。证明这连接9个点而形成的线段中至少包含另外一个坐标为整数的点
\end{question}

\begin{question}
  任意六个人中,以下两个结论总有一个成立:
  \begin{enumerate}
  \item 其中有三人互相认识。
  \item 其中有三人互相不认识。
  \end{enumerate}
\end{question}

\hints 相当于正六边形的六个顶点用两种颜色的线段两两连接起来,则这些线段组合的三角形中,总存在一个同色的三角形。

\begin{question}
  任意一个16位的正整数的数字所组成的数字串中,总存在一个子串,其数字的乘积是一个完全平方数。
\end{question}

\hints 若16位数中含有数字$0,1,4,9$,那么取这个数字本身组成的子串即可,从而只需考虑由数字$2,3,5,6,7,8$组成的16位数。由于
\begin{align*}
  6 = 2\times 3,\quad 8=2^3
\end{align*}
从而任意一个子串各数字的乘积可以写成如下形式
\begin{align*}
  2^a \times 3^b \times 5^c \times 7^d
\end{align*}
若$a,b,c,d$都是偶数,那么这个子串数字的乘积就是完全平方数。而$a,b,c,d$的奇偶组合数总共有$2^4=16$种,即
\begin{align*}
  \text{(奇奇奇奇)},\quad \text{(奇奇奇偶)},\quad \text{(奇奇偶奇)},\quad \text{(奇奇偶偶)}\\
  \text{(奇偶奇奇)},\quad \text{(奇偶奇偶)},\quad \text{(奇偶偶奇)},\quad \text{(奇偶偶偶)}\\
  \text{(偶奇奇奇)},\quad \text{(偶奇奇偶)},\quad \text{(偶奇偶奇)},\quad \text{(偶奇偶偶)}\\
  \text{(偶偶奇奇)},\quad \text{(偶偶奇偶)},\quad \text{(偶偶偶奇)},\quad \text{(偶偶偶偶)}
\end{align*}

从而至少需要构造17种选取方式。记16位数各数字按顺序分别为$x_1,x_2,\cdots,x_{16}$,令
\begin{align*}
  F(k) =
  \begin{cases}
    1 = 2^0 \times 3^0 \times 5^0 \times 7^0 & k=0\\
    \prod_{i=1}^k x_i = 2^{a_k} \times 3^{b_k} \times 5^{c_k} \times 7^{d_k},&k=1,2,\cdots,16
  \end{cases}
\end{align*}
则$F(0),F(1),\cdots,F(16)$这17个数中至少有两个,其$a,b,c,d$的奇偶性组合相同,不妨记为$F(k_1),F(k_2)$,且$k_1<k_2$,从而$F(k_2)/F(k_1)$是完全平方数,从而两个数对应的子串,长串中拿掉短串部分,其数字乘积即为完全平方数。
