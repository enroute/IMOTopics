
\chapter{有些坑不能跳}
\label{chap:traps}

\epigraph{好读书,不求甚解,每有会意,便欣然忘食。}{
  {晋·陶潜《五柳先生传》}
}

世间诸多事物,了解即可,权当开阔视野,无需沉迷,否则跳脱不出,只能自误。读书,有时需要五柳先生的“不求甚解”。

\section{猜想}
\label{sec:guess}

\begin{example}[Collatz猜想,也叫做$3n + 1$问题]这可能是数学中最为世人所知的未解之谜。它是如此初等,连小学生都能听懂它的内容;但解决它却如此之难,以至于 Paul Erd\"os 曾说:“或许现在的数学还没准备好去解决这样的问题。”Collatz 猜想的叙述如下:

  任意取一个正整数$n$。如果$n$是奇数,则把$n$变为$3n + 1$ ;如果$n$是偶数,则把$n$变为$n/2$。不断重复操作,则最终一定会得到$1$。
\end{example}

比如取$n=26$,则有
\begin{align*}
  26\to 13\to 40\to 20\to 10\to 5\to 16\to 8\to 4\to 2\to 1
\end{align*}

这个问题看起来是如此简单,以至于无数的数学家都掉进了这个坑里。光从这个问题的众多别名,便能看出这个问题害人不浅: Collatz 猜想又叫做 Ulam 猜想、 Kakutani 问题、 Thwaites 猜想、 Hasse 算法、 Syracuse 问题$\cdots\cdots$研究这个问题的人很多,解决这个问题的人却一个没有。后来,人们干脆把它叫做 $3n + 1$ 问题,让哪个数学家也不沾光。

这个问题有多难呢?我们可以从下面的这个例子中略见一斑。上面从 26 出发只消 10 步就能变成 1 ,但若换一个数,比如 27 ,情况就大不一样了:

$
27 \to 82 \to 41 \to 124 \to 62 \to 31 \to 94 \to 47 \to 142 \to 71 \to 214 \to 107 \to 322 \to 161 \to 484 \to 242 \to 121 \to 364 \to 182 \to 91 \to 274 \to 137 \to 412 \to 206 \to 103 \to 310 \to 155 \to 466 \to 233 \to 700 \to 350 \to 175 \to 526 \to 263 \to 790 \to 395 \to 1186 \to 593 \to 1780 \to 890 \to 445 \to 1336 \to 668 \to 334 \to 167 \to 502 \to 251 \to 754 \to 377 \to 1132 \to 566 \to 283 \to 850 \to 425 \to 1276 \to 638 \to 319 \to 958 \to 479 \to 1438 \to 719 \to 2158 \to 1079 \to 3238 \to 1619 \to 4858 \to 2429 \to 7288 \to 3644 \to 1822 \to 911 \to 2734 \to 1367 \to 4102 \to 2051 \to 6154 \to 3077 \to 9232 \to 4616 \to 2308 \to 1154 \to 577 \to 1732 \to 866 \to 433 \to 1300 \to 650 \to 325 \to 976 \to 488 \to 244 \to 122 \to 61 \to 184 \to 92 \to 46 \to 23 \to 70 \to 35 \to 106 \to 53 \to 160 \to 80 \to 40 \to 20 \to 10 \to 5 \to 16 \to 8 \to 4 \to 2 \to 1
$

可见,当$n$的值不同时,从$n$变到$1$的路子是很没规律的。

在决定挑战证明这个猜想之前,请再想一想为什么众多前辈都在该问题上折戟沉沙。


\begin{example}[蜂窝猜想,Honeycomb Conjecture]
  六角蜂巢猜想(蜂窝猜想)阐述了正六边形网格(蜂巢)是使用最少的总周长将该表面划分成面积相等的区域的最佳方法。
\end{example}

关于蜂窝猜想的第一次记录可以追溯到36BC的马库斯·特伦提乌斯·瓦罗,但总是被认为是帕普斯(c.290 – c.350)提出的。该猜想于1999年被美国数学家托马斯·黑尔斯(Thomas C. Hales)证明,黑尔斯在他的研究中提到,有理由相信这个猜想可能已经出现在比瓦罗较早的数学家的思想中了。

\subsection{曾经的三大数学猜想}
\label{sec:3-math-conjectures}

哥德巴赫猜想、费马猜想和四色猜想曾经被称为世界三大数学猜想。这三个问题的共同点是题面简单易懂,内涵深邃无比。至目前为止,仅剩余哥德巴赫猜想未被解决。

\begin{example}[哥德巴赫猜想,Goldbach Conjecture]\label{ex:Goldbach-conjecture}
  1742年6月7日,哥德巴赫写信给当时的大数学家欧拉,提出了以下想法:任何一个大于等于6的偶数,都可以表示成两个奇质数之和;任何一个大于等于9的奇数,都可以表示成三个奇质数之和。
\end{example}

哥德巴赫猜想至今仍没人能证明,最接近成功的是陈景润的证明。

\begin{example}[费马猜想,费马大定理,Fermat's Last Theorem]
  当整数$n>2$时,关于$x,y,z$的方程
  \begin{align*}
    x^n+y^n=z^n
  \end{align*}
  没有正整数解。
\end{example}
1637年,费马在阅读丢番图(Diophantus)《算术》拉丁文译本时,曾在第11卷第8命题旁写道:

\begin{quotation}
  将一个立方数分成两个立方数之和,或一个四次幂分成两个四次幂之和,或者一般地将一个高于二次的幂分成两个同次幂之和,这是不可能的。关于此,我确信我发现了一种美妙的证法,可惜这里的空白处太小,写不下。
\end{quotation}

尽管费马表明他已找到一个精妙的证明,但长时间以来都没有人能真正证明该问题,直到1995年英国数学家安德鲁·怀尔斯(Andrew John Wiles)及其学生理查·泰勒(Richard Taylor)将他们的证明出版后,该猜想才称为“费马大定理”(又称为:费马最后定理)。


\begin{definition}[外飞地,Exclave]
  外飞地通常指某国家(或国家以下的某级地方行政单位)拥有一块与本国(地区)分离开来的领土,若该领土被其他国家(地区)包围,则该领土称为某国(地区)的外飞地。
\end{definition}

美国的阿拉斯加是世界上面积最大的飞地,它与美国本土被加拿大分隔。

\begin{example}[四色猜想,四色定理,Four Color Theorem]
  每个无外飞地的地图都可以用不多于四种颜色来染色,而且不会有两个邻接的区域颜色相同。
\end{example}

“是否只用四种颜色就能为所有地图染色?”的问题最早是由英国数学家法兰西斯·古德里在1852年提出的,被称为“四色问题”或“四色猜想”。人们发现,要证明宽松一点的“五色定理”(即“只用五种颜色就能为所有地图染色”)很容易,但四色问题却出人意料地异常困难。曾经有许多人发表四色问题的证明或反例,但都被证实是错误的。

1976年,数学家凯尼斯·阿佩尔和沃夫冈·哈肯借助电子计算机首次得到一个完全的证明,四色问题也终于成为四色定理。这是首个主要借助计算机证明的定理。这个证明一开始并不为许多数学家接受,因为不少人认为这个证明无法用人手直接验证。尽管随着计算机的普及,数学界对计算机辅助证明更能接受,但仍有数学家希望能够找到更简洁或不借助计算机的证明。

\section{看起来很简单}
\label{sec:looks-simple}

\begin{example}
  找出下面关于$a,b,c$的方程的正整数解:
  \begin{align*}
    \frac{a}{b+c}+\frac{b}{c+a}+\frac{c}{a+b}=4
  \end{align*}
\end{example}
这个总是决不像它看起来那么简单。容易看出来的是,这是一个齐次方程,即若$(a,b,c)$是一个解,那么对任意非零整数$k$,$(ka, kb, kc)$也是一个解。

其次,去分母化简后可知,这是一个三次丢番图方程。一次丢番图方程很简单,二次丢番图方程也已经被研究得很透彻,三次就很难了,充满了高深的理论,四次,难难难。

齐次是可以降维的。引入新变量,经过复杂的变换之后,可以得到另一个新方程:
\begin{align*}
  y^2=x^3+109x^2+224x
\end{align*}
至于这个变换是怎么来的,篇幅就大了,如果是立志要解决此类问题的数学家,可以去找找椭圆曲线的相关书籍查查,如果没听说过椭圆曲线的相关理论,那还是不要深究这个问题了,一笑而过就好。据说有人找出来的一个解为:
{\tiny
\begin{align*}
  a={}&4373612677928697257861252602371390152816537558161613618621437993378423467772036\\
  b={}&36875131794129999827197811565225474825492979968971970996283137471637224634055579\\
  c={}&154476802108746166441951315019919837485664325669565431700026634898253202035277999  
\end{align*}
}
至于是不是一个正确的解,就留待有兴趣的人去做吧。

