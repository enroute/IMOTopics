
\begin{example}[拼装矩形]
  若大矩形由若干个小矩形拼装,每个小矩形至少有一条边是整数长,则大矩形也至少有一条边是整数长。
\end{example}
\begin{proof}
  如图,令$T$是平面上的矩形。
  \begin{center}
    \begin{tikzpicture}[scale=.5]
      \filldraw[pattern=north west lines,pattern color=red!20](2,1)rectangle(5,3)node[midway,fill=white]{$T$};
      \draw[->](-1,0)--(6,0)node[right]{$x$};
      \draw[->](0,-1)--(0,4)node[above]{$y$};
      \draw[dashed](2,1)--(2,0)node[below]{$a$}
                   (5,1)--(5,0)node[below]{$b$}
                   (2,1)--(0,1)node[left]{$c$}
                   (2,3)--(0,3)node[left]{$d$};
      \node[below left]at(0,0){$O$};
      \begin{scope}[shift={(9,1)}]
        \draw(0,0)rectangle(3,2);
        \draw(1,0)--(1,1.5) (1,1)--(3,1) (0,1.5)--(2.5,1.5) (1.2,1.5)--(1.2,2) (2.5,1)--(2.5,2);
        \foreach \x/\y/\v in{.5/.75/1, 2/.5/2}{
          \node at(\x,\y){\tiny $T_{\v}$};
        }
      \end{scope}
    \end{tikzpicture}
  \end{center}
  考虑$T$上的二重积分:
  \begin{align*}
    \int_c^d\int_a^b e^{2\pi i(x+y)} \dx\dy
    =\int_a^b e^{2\pi ix} \dx \cdot \int_c^d e^{2\pi iy} \dy
  \end{align*}
  容易证明
  \begin{align*}
    \int_a^b e^{2\pi ix} \dx = 0 \iff b-a\text{是整数},\qquad
    \int_c^d e^{2\pi iy} \dy = 0 \iff d-c\text{是整数}
  \end{align*}
  根据拼装要求,对每个小矩形$T_i$,都有$\int\int_{T_i} = 0$,根据积分的可加性,$\int\int_T = 0$,从而至少其中一边的长度为整数。
\end{proof}


\begin{example}[费马平方和定理]
  每个形如$p=4m+1$的素数都是两个平方数之和。
\end{example}
\begin{proof}
  
\end{proof}
