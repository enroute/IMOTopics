
\section{平面几何五大模型}
\label{sec:five-models-in-geometry}

\subsection{等积模型}
\label{sec:same-volumne-model}

\begin{theorem}
  两个三角形若等底等高,则其面积相等。
\end{theorem}

\begin{theorem}
  两个三角形若底相同,则其面积比等于高的比;两个三角形若高相同,则其面积比等于对应底的比。
\end{theorem}

这两个结论由三角形面积公式(底乘高除以2)显然可得。


\subsection{鸟头模型}
\label{sec:common-angle-theorem}

\begin{definition}[共角三角形]
  两三角形若有一个角相等或互补,则称这两个三角形为共同三角形。
\end{definition}

\begin{theorem}[共角定理,鸟头定理]
  共角三角形的面积比等于相等或互补的角的两夹边的乘积比。
\end{theorem}
\begin{proof}
  由三角形的另一面积公式$S=\dfrac12ab\sin C$,结论显然成立。
\end{proof}

\begin{example}
  如图,其中线段长度如标示,求阴影部分与非阴影部分的面积比。

  \centering
  \begin{tikzpicture}[scale=1.0]
    \begin{scope}[shift={(0,0)}]
      \coordinate (A) at (2,2);
      \coordinate (B) at (0,0);
      \coordinate (C) at (3,0);
      \coordinate (D) at ($.75*(B) + .25*(A)$);
      \coordinate (E) at ($2/3*(A) + 1/3*(C)$);
      \draw(E)--(D)
              --(B)node[pos=.15,left]{$1$}
              --(C)
              --(E)node[pos=.65,right]{$2$}
              --(A)node[pos=.65,right]{$1$}
              --(D)node[pos=.35,left]{$3$};
      \fill[pattern=north west lines,pattern color=blue!30](A)--(D)--(E)--cycle;
    \end{scope}
  \end{tikzpicture}
\end{example}
\begin{proof}[解]
  由共角定理,阴影部分与大三角形面积比为
  \begin{align*}
    \frac{3\times 1}{(3+1)\times(1+2)}=\frac{3}{3\times4}=\frac14
  \end{align*}
  从而在大三角形中,阴影部分面积占$\dfrac14$,非阴影部分点$\dfrac34$,阴影与非阴影面积之比为$1:3$。
\end{proof}

\begin{question}
  以下图形的阴影与非阴影面积之间的比例关系同样可应用鸟头定理,请自行思考。

  \centering
  \begin{tikzpicture}[scale=1.0]
    \begin{scope}[shift={(0,0)}]
      \coordinate(A) at (2,2);
      \coordinate(B) at (0,0);
      \coordinate(C) at (3,0);
      \coordinate(D) at ($.4*(C)+.6*(A)$);
      \coordinate(E) at ($1.4*(A)$);
      \draw[pattern=north west lines,pattern color=blue!30](A)--(D)--(E)--cycle;
      \draw(D)--(C)--(B)--(A);
    \end{scope}

    \begin{scope}[shift={(5,0)}]
      \coordinate(A) at (2,2);
      \coordinate(B) at (0,0);
      \coordinate(C) at (3,0);
      \coordinate(D) at ($1.6*(C)-.6*(A)$);
      \coordinate(E) at ($.4*(A)$);
      \draw[pattern=north west lines,pattern color=blue!30](A)--(D)--(E)--cycle;
      \draw(A)--(C)--(B)--(E);
    \end{scope}

    \begin{scope}[shift={(8,0)}]
      
    \end{scope}
  \end{tikzpicture}
\end{question}


\subsection{蝴蝶模型}
\label{sec:butterfly-theorem}

\begin{theorem}[蝴蝶定理]
  如图,任意凸四边形的对角线将四边形分成$4$个小三角形,则其面积$S_1,S_2,S_3,S_4$满足以下关系
  \begin{align*}
    S_1S_3=S_2S_4
  \end{align*}

  \centering
  \begin{tikzpicture}[scale=1.0]
    \coordinate(A)at(0,0);
    \coordinate(B)at(4,0);
    \coordinate(C)at(3,2.5);
    \coordinate(D)at(1,3);
    \draw(A)--(B)--(C)--(D)--(A)--(C) (B)--(D);
    \tkzInterLL(A,C)(B,D)\tkzGetPoint{O}
    \fill[pattern=north west lines,pattern color=blue!30](A)--(O)--(B)--cycle;
    \fill[pattern=north west lines,pattern color=blue!30](C)--(O)--(D)--cycle;
    \node at ($1/3*(D)+1/3*(A)+1/3*(O)$) {$S_2$};
    \node at ($1/3*(A)+1/3*(B)+1/3*(O)$) {$S_3$};
    \node at ($1/3*(B)+1/3*(C)+1/3*(O)$) {$S_4$};
    \node at ($1/3*(C)+1/3*(D)+1/3*(O)$) {$S_1$};
  \end{tikzpicture}
\end{theorem}
\begin{proof}
  找三角形的同底或等高关系,可得$\dfrac{S_1}{S_2}=\dfrac{S_4}{S_3}$,变换一下即得证。
\end{proof}

\begin{example}[梯形中的蝴蝶定理]
  如图,梯形中的上下底边长分别为$a$和$b$,则对角线分成的$4$个小三角形的面积有如下关系:
  \begin{align*}
    S_1:S_2:S_3:S_4=a^2:ab:b^2:ab
  \end{align*}

  \centering
  \begin{tikzpicture}[scale=1.0]
    \coordinate(A)at(0,0);
    \coordinate(B)at(5,0);
    \coordinate(C)at(4,3);
    \coordinate(D)at(2,3);
    \draw(A)--(B) node[midway,below]{$b$}--(C)--(D) node[midway,above]{$a$}--(A)--(C) (B)--(D);
    \tkzInterLL(A,C)(B,D)\tkzGetPoint{O}
    \fill[pattern=north west lines,pattern color=blue!30](A)--(O)--(B)--cycle;
    \fill[pattern=north west lines,pattern color=blue!30](C)--(O)--(D)--cycle;
    \node at ($1/3*(D)+1/3*(A)+1/3*(O)$) {$S_2$};
    \node at ($1/3*(A)+1/3*(B)+1/3*(O)$) {$S_3$};
    \node at ($1/3*(B)+1/3*(C)+1/3*(O)$) {$S_4$};
    \node at ($1/3*(C)+1/3*(D)+1/3*(O)$) {$S_1$};
  \end{tikzpicture}
\end{example}
\begin{proof}
  首先由三角形相似性,有$S_1:S_3=a^2:b^2$。其次$S_2+S_3$与$S_4+S_3$同底等高从而面积相等,也就是$S_2=S_4$。再由蝴蝶定理,并令$S_1=a^2$,从而有$S_3=b^2$,并且
  \begin{align*}
    S_1\cdot S_3=S_2\cdot S_4\implies a^2 \cdot b^2 = S_2^2\implies S_2=ab&\qedhere
  \end{align*}
\end{proof}


\subsection{燕尾模型(swallowtaile-theore??????)}
\label{sec:swallowtail-theore}


\begin{theorem}[燕尾定理]
  如图,$D,E,F$分别是三角形$ABC$三边上一点,且$AD$、$BE$与$CF$有共同的交点$O$,则
  \begin{align*}
    S_{\triangle ABO}:S_{\triangle ACO} = S_{\triangle OBD}:S_{\triangle OCD} = BD:CD
  \end{align*}
  其余三角形有类似结论。
  
  \centering
  \begin{tikzpicture}[scale=1.0]
    \begin{scope}[shift={(0,0)}]
      \coordinate(A)at(1,3);
      \coordinate(B)at(0,0);
      \coordinate(C)at(4,0);
      \coordinate(O)at(1.5,1.5);
      \tkzInterLL(A,O)(B,C)\tkzGetPoint{D}
      \tkzInterLL(B,O)(C,A)\tkzGetPoint{E}
      \tkzInterLL(C,O)(A,B)\tkzGetPoint{F}
      \fill[color=blue!30](A)--(B)--(O)--(C)--(A);
      \draw(A)--(B)--(C)--cycle (A)--(D) (B)--(E) (C)--(F);
      \tkzLabelPoints[above](A)
      \tkzLabelPoints[below left](B)
      \tkzLabelPoints[below right](C)
      \tkzLabelPoints[below](D)
      \tkzLabelPoints[right](E)
      \tkzLabelPoints[left](F)
      \node at($(O) + (-.1,-.4)$) {$O$};
    \end{scope}

    \begin{scope}[shift={(7,0)}]
      \coordinate(A)at(1,3);
      \coordinate(B)at(0,0);
      \coordinate(C)at(4,0);
      \coordinate(O)at(1.5,1.5);
      \tkzInterLL(A,O)(B,C)\tkzGetPoint{D}
      \tkzInterLL(B,O)(C,A)\tkzGetPoint{E}
      \tkzInterLL(C,O)(A,B)\tkzGetPoint{F}
      \fill[pattern=north west lines,pattern color=blue!30](A)--(B)--(O);
      \fill[pattern=north east lines,pattern color=red!30](A)--(C)--(O);
      \draw(A)--(B)--(C)--cycle (A)--(D) (B)--(O) (C)--(O);
      \tkzLabelPoints[above](A)
      \tkzLabelPoints[below left](B)
      \tkzLabelPoints[below right](C)
      \tkzLabelPoints[below](D)
      % \tkzLabelPoints[right](E)
      % \tkzLabelPoints[left](F)
      \node at($(O) + (-.1,-.4)$) {$O$};
    \end{scope}
  \end{tikzpicture}
\end{theorem}
\begin{proof}[说明]
  如第二图的最小模型,若只考虑一个燕尾,可在$AD$上随意取点$O$即有上述结论,即无需知道$E,F$两点。
\end{proof}

\begin{example}
  在燕尾定理的图中,燕尾$ABOC$的面积与$\triangle ABC$的面积比等于$AO:AD$。
\end{example}
\begin{proof}[提示]
  将燕尾沿$AD$切成两边分别考虑,有
  \begin{align*}
    \frac{S_{\triangle AOB}}{S_{\triangle ADB}} = \frac{AO}{AD},\quad
    \frac{S_{\triangle AOC}}{S_{\triangle ADC}} = \frac{AO}{AD}
  \end{align*}
  两式分子分母分别相加可得\footnote{这里用到了这样一个结论:若$a_1/b_1=a_2/b_2$,则$(a_1+a_2)/(b_1+b_2)=a_1/b_1$。当然,这里$b_1$和$b_2$需要满足一定条件,请自行思考。}。
\end{proof}

\begin{example}
  在燕尾定理的图中,证明
  \begin{align*}
    \frac{BD}{DC}\times\frac{CE}{EA}\times\frac{AF}{FB}=1
  \end{align*}
\end{example}
\begin{proof}
  由燕尾定理,有
  \begin{align*}
    S_{\triangle ABO}:S_{\triangle ACO} &= BD:DC\\
    S_{\triangle ACO}:S_{\triangle CBO} &= AF:FB\\
    S_{\triangle CBO}:S_{\triangle ABO} &= CE:EA
  \end{align*}
  三式相乘可得。
\end{proof}


\begin{example}
  已知$\triangle ABC$中,$AE:EF:FB=1:3:2$,$D$是$BC$的中点,且图中阴影部分$S_{EFHG}=51$,求$\triangle ABC$的面积$S_{\triangle ABC}$。
  \begin{center}
    \begin{tikzpicture}[scale=1.0]
      \coordinate(A)at(0,0);\coordinate(E)at(1,0);\coordinate(F)at(4,0);\coordinate(B)at(6,0);
      \coordinate(C)at(4,4);\coordinate(D)at(5,2);
      \tkzInterLL(A,D)(C,E)\tkzGetPoint{G}
      \tkzInterLL(A,D)(C,F)\tkzGetPoint{H}
      \fill[color=red!20](E)--(F)--(H)--(G)--cycle;
      \draw(A)--(B)--(C)--(A)--(D) (E)--(C)--(F);
      \tkzDrawPoints(A,B,C,D,E,F,G,H);
      \foreach \L/\P in{A/below left, B/below right, C/above, D/right, E/below, F/below, G/above left, H/above left}{
        \tkzLabelPoints[\P](\L)
      }
    \end{tikzpicture}
  \end{center}
\end{example}

\begin{proof}[提示]
  解法一,可用燕尾定理,关键是得到$AG:GH:HD$。考虑燕尾$CAHB$的两边,如下图:
  \begin{center}
    \begin{tikzpicture}[scale=.8]
      \begin{scope}
        \coordinate(A)at(0,0);\coordinate(E)at(1,0);\coordinate(F)at(4,0);\coordinate(B)at(6,0);
        \coordinate(C)at(4,4);\coordinate(D)at(5,2);
        \tkzInterLL(A,D)(C,E)\tkzGetPoint{G}
        \tkzInterLL(A,D)(C,F)\tkzGetPoint{H}
        % \fill[color=red!20](E)--(F)--(H)--(G)--cycle;
        \fill[pattern=north west lines, pattern color=red!20](C)--(A)--(H);
        \fill[pattern=north east lines, pattern color=blue!20](C)--(B)--(H);
        \draw(A)--(B)--(C)--(A)--(D) (C)--(F);
        \draw[dashed](H)--(B);
        \tkzDrawPoints(A,B,C,D,E,F,G,H);
        \foreach \L/\P in{A/below left, B/below right, C/above, D/right, E/below, F/below, G/above, H/above left}{
          \tkzLabelPoints[\P](\L)
        }
      \end{scope}
      \begin{scope}[shift={(8,0)}]
        \coordinate(A)at(0,0);\coordinate(E)at(1,0);\coordinate(F)at(4,0);\coordinate(B)at(6,0);
        \coordinate(C)at(4,4);\coordinate(D)at(5,2);
        \tkzInterLL(A,D)(C,E)\tkzGetPoint{G}
        \tkzInterLL(A,D)(C,F)\tkzGetPoint{H}
        % \fill[color=red!20](E)--(F)--(H)--(G)--cycle;
        \fill[pattern=north west lines, pattern color=red!20](C)--(A)--(G);
        \fill[pattern=north east lines, pattern color=blue!20](C)--(B)--(G);
        \draw(A)--(B)--(C)--(A)--(D) (C)--(E);
        \draw[dashed](G)--(B);
        \tkzDrawPoints(A,B,C,D,E,F,G,H);
        \foreach \L/\P in{A/below left, B/below right, C/above, D/right, E/below, F/below, G/above, H/above left}{
          \tkzLabelPoints[\P](\L)
        }
      \end{scope}
    \end{tikzpicture}
  \end{center}
  则有
  \begin{align*}
    S_{\triangle CHA}:S_{\triangle CHB} = AF:FB = (1+3):2=2
  \end{align*}
  由$D$是$BC$中点,可知$S_{\triangle BHD}=S_{\triangle CHD}$,代入有
  \begin{align*}
    S_{\triangle CAH}:(2\times S_{\triangle CHD}) = 2\implies
    S_{\triangle CAH}:S_{\triangle CHD} = 4
  \end{align*}
  从而有$AH:HD=4$。同样的,由燕尾$CAGB$,有
  \begin{align*}
    S_{\triangle CGA}:S_{\triangle CGD} ={}& 2\times( S_{\triangle CGA}:S_{\triangle CGB} )\\
    ={}& 2\times( AE:EB ) = 2\times( 1 : (3+2))=2:5
  \end{align*}
  由此可得$AG:GH:HD$。实际上,$AG$占整个$AD$的$\frac27$,$HD$占整个$AD$的$\frac15$,所以$GH$占整个$AD$的$1-\frac27-\frac15$,由此可得三者的比例为$10:18:7$。从而由共角定理有
  \begin{align*}
    &S_{\triangle AEG} : S_{\triangle AFH} : S_{\triangle ABD}\\
    =&(1\times 10) : ((1+3)\times(10+18)) : ((1+3+2)\times(10+18+7))\\
    =&10 : 112 : 210
  \end{align*}
  由此可得
  \begin{align*}
    S_{\triangle AEG} : S_{EFHG} : S_{FBDH} ={}& 10 : (112-10) : (210-112)\\
    ={}& 10:102:98=5:51:49
  \end{align*}
  现已知$S_{EFHG}=51$,从而有
  \begin{align*}
    S_{\triangle AEG}={}&5\\
    S_{FBDH}         ={}&49\\
    S_{\triangle ABD}={}&5+51+49=105\\
    S_{\triangle ABC}={}&2\times 105=210
  \end{align*}

  类似地也通过求得$CG:GE$及$CH:HF$,来得到原图中各小部分的面积比。

  方法二,可作各种平行线,求得各线段的比例关系。

  方法三可用梅涅劳斯定理。
\end{proof}

\begin{theorem}
  如图,$\triangle ABC$中,$D$和$E$分别是$AB$和$AC$上一点,$CD$和$BE$交于点$F$。若$AD:DB=x:y$,$AE:EC=u:v$,则
  \begin{align*}
    \frac{CF}{FD} = \dfrac{\frac{x}{y}}{1+\frac{u}{v}},\qquad\qquad
    \frac{BF}{FE} = \dfrac{\frac{u}{v}}{1+\frac{x}{y}}
  \end{align*}
  \begin{center}
    \begin{tikzpicture}[scale=1.0]
      \coordinate(A)at(4,0);\coordinate(B)at(3,3);\coordinate(C)at(0,0);
      \coordinate(D)at($0.4*(A)+0.6*(B)$);\coordinate(E)at(2.5,0);
      \tkzInterLL(C,D)(B,E)\tkzGetPoint{F}
      \draw(A)--(D) node[midway,right]{$x$}
              --(B) node[midway,right]{$y$}
              --(C)
              --(E) node[midway,below]{$u$}
              --(A) node[midway,below]{$v$}
           (C)--(D) (B)--(E);
      \tkzDrawPoints(A,B,C,D,E,F)
      \foreach \L/\P in{A/below right, B/above, C/below left, D/right, E/below, F/above left}{
        \tkzLabelPoints[\P](\L)
      }
    \end{tikzpicture}
  \end{center}
\end{theorem}

\begin{proof}
  连接$AF$,考虑燕尾$BCFA$,有$S_{\triangle BAF}:S_{\triangle BCF} = x:y$,把$S_{\triangle AFD}:S_{\triangle BFD}=u:v$代入,有
  \begin{align*}
    \dfrac{(1 + \frac{u}{v})S_{\triangle BFD}}{S_{\triangle BCF}} = \frac{x}{y}
  \end{align*}
  从而
  \begin{align*}
    \frac{CF}{FD}=\dfrac{S_{\triangle BFD}}{S_{\triangle BCF}} = \dfrac{\frac{x}{y}}{1+\frac{u}{v}}&\qedhere
  \end{align*}
\end{proof}

\begin{example}
  记$P$为$\forall\triangle ABC$内任意一点,且$AP$、$BP$及$CP$分别与$\triangle ABC$的三边相交于$D$、$E$及$F$。求
  \begin{align*}
    \frac{AP}{AD} + \frac{BP}{BE} + \frac{CP}{CF} = ?
  \end{align*}
  \centering
  \begin{tikzpicture}[scale=1.2]
    \begin{scope}
      \coordinate[label=above:$A$](A)at(2,2);
      \coordinate[label=below left:$B$](B)at(0,0);
      \coordinate[label=below right:$C$](C)at(3,0);
      \coordinate[label=above:$P$](P)at(1.5,.5);
      \tkzInterLL(A,P)(B,C)\tkzGetPoint{D}
      \tkzInterLL(B,P)(C,A)\tkzGetPoint{E}
      \tkzInterLL(C,P)(A,B)\tkzGetPoint{F}
      \draw(A)--(B)--(C)--cycle;
      \draw(A)--(D) (B)--(E) (C)--(F);
      \tkzDrawPoints(A,B,C,D,E,F,P)
      \tkzLabelPoints[below](D)
      \tkzLabelPoints[right](E)
      \tkzLabelPoints[left](F)
    \end{scope}
    \begin{scope}[shift={(5,0)}]
      \coordinate[label=above:$A$](A)at(2,2);
      \coordinate[label=below left:$B$](B)at(0,0);
      \coordinate[label=below right:$C$](C)at(3,0);
      \coordinate(P)at(1.5,.5);
      \tkzInterLL(A,P)(B,C)\tkzGetPoint{D}
      \tkzInterLL(B,P)(C,A)\tkzGetPoint{E}
      \tkzInterLL(C,P)(A,B)\tkzGetPoint{F}
      \fill[color=red!20](A)--(P)--(B)--cycle;
      \fill[pattern=north west lines,pattern color=blue!20](B)--(P)--(C)--cycle;
      \draw(A)--(B)--(C)--cycle;
      \draw(A)--(P) (B)--(P) (C)--(P);
      \tkzDrawPoints(A,B,C,P)
      % \tkzLabelPoints[below](D)
      % \tkzLabelPoints[right](E)
      % \tkzLabelPoints[left](F)
      \node at($1/3*(A)+1/3*(B)+1/3*(P)$){$S_1$};
      \node at($1/3*(B)+1/3*(C)+1/3*(P)$){$S_2$};
      \node at($1/3*(C)+1/3*(A)+1/3*(P)$){$S_3$};
    \end{scope}
  \end{tikzpicture}
\end{example}
\begin{proof}[提示]
  有时长度的比可以转化为面积体积的比。记$S_1\equiv S_{\triangle APB}$,$S_2\equiv S_{\triangle BPC}$,$S_3\equiv S_{\triangle CPA}$,则利用燕尾定理可得
  \begin{align*}
    \frac{AP}{AD}={}\frac{S_1 + S_3}{S_{\triangle ABC}},\quad
    \frac{BP}{BE}={}\frac{S_2 + S_1}{S_{\triangle ABC}},\quad
    \frac{CP}{CF}={}\frac{S_3 + S_2}{S_{\triangle ABC}}
  \end{align*}
  三式相加,可得
  \begin{align*}
    \frac{AP}{AD} + \frac{BP}{BE} + \frac{CP}{CF} = 2&\qedhere
  \end{align*}
\end{proof}

\subsection{相似模型}
\label{sec:similar-model}

?????