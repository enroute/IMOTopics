
\chapter{算法}
\label{chap:algorithm}

\section{递归}
\label{sec:recursive}

\begin{example}
  给定常数$0<\lambda<1$及三角形$\triangle ABC$,选定任意一顶点出发,在相邻的边上找到比例为$\lambda$的分割点,连接顶点与分割点。再从此分割点出发,在新的三角形上做上述动作。
  \begin{center}
    \begin{tikzpicture}[scale=.5]
      \begin{scope}
        \coordinate (A) at (0,0); \coordinate(B) at (4,0); \coordinate(C) at (3,3);
        \draw(A)--(B)node[below right]{$V_1$}
                --(C)node[above]{$V_2$}
                --(A)node[below left]{$V_3$};
        \setcounter{X}{0}
        \whiledo{\value{X}<1}{%
          \coordinate(D) at ($.8*(B)+.2*(C)$);
          \draw(A)--(D);
          \coordinate(E)at(A);\coordinate(A)at(D);\coordinate(B)at(C);\coordinate(C)at(E);
          \stepcounter{X}
        }
      \end{scope}
      \begin{scope}[shift={(7,0)}]
        \coordinate (A) at (0,0); \coordinate(B) at (4,0); \coordinate(C) at (3,3);
        \draw(A)--(B)--(C)--cycle;
        \setcounter{X}{0}
        \whiledo{\value{X}<2}{%
          \coordinate(D) at ($.8*(B)+.2*(C)$);
          \draw(A)--(D);
          \coordinate(E)at(A);\coordinate(A)at(D);\coordinate(B)at(C);\coordinate(C)at(E);
          \stepcounter{X}
        }
      \end{scope}
      \begin{scope}[shift={(14,0)}]
        \coordinate (A) at (0,0); \coordinate(B) at (4,0); \coordinate(C) at (3,3);
        \draw(A)--(B)--(C)--cycle;
        \setcounter{X}{0}
        \whiledo{\value{X}<18}{%
          \coordinate(D) at ($.8*(B)+.2*(C)$);
          \draw(A)--(D);
          \coordinate(E)at(A);\coordinate(A)at(D);\coordinate(B)at(C);\coordinate(C)at(E);
          \stepcounter{X}
        }
    \end{scope}
    \end{tikzpicture}
  \end{center}
  递归本质上也是一种迭代。以上图为例,记初始顶点为$V_1, V_2, V_3$,则
  \begin{align*}
    V_n = \lambda V_{n-3} + (1-\lambda)V_{n-2},\quad (3<n\in\mathbb{Z})
  \end{align*}
  新加的边为$V_{n-1}V_n$。
\end{example}

\begin{example}
  对长方形划分为若干的三角形,在每个三角形中作上例中的操作,可以得到一个神奇的图形。
  \begin{center}
    \begin{tikzpicture}[scale=1.0]
      \begin{scope}[shift={(-7,0)}]
        \foreach \i/\x in {0/0,1/1.5,2/3,3/4.5,4/6}{
          \foreach \j/\y in{0/0,1/1,2/2,3/3,4/4}{
            \coordinate(N\i\j)at(\x,\y);
          }
        }
        \draw(N00)rectangle(N44);
        \foreach \A/\B in{%
          N01/N41, N02/N42, N03/N43,
          N10/N14, N20/N24, N30/N34,
          % N00/N44, 
          % N10/N43, N20/N42, N30/N41,
          % N01/N34, N02/N24, N03/N14,%
          N10/N01, N20/N02, N30/N03, N40/N04, N41/N14, N42/N24, N43/N34%
        }{%
          \draw(\A)--(\B);
        }
      \end{scope}
      \def\RecursiveTriangle[#1,#2,#3,#4]{ %#1:N1, #2:N2, #3:N3, #4:\lambda
        \coordinate (A) at (#1); \coordinate(B) at (#2); \coordinate(C) at (#3);
        \draw(A)--(B)--(C)--cycle;
        \setcounter{X}{0}
        \whiledo{\value{X}<18}{%
          \coordinate(D) at ($#4*(B)+(C)-#4*(C)$);
          \draw(A)--(D);
          \coordinate(E)at(A);\coordinate(A)at(D);\coordinate(B)at(C);\coordinate(C)at(E);
          \stepcounter{X}
        }
      }
      \draw(0,0)rectangle(6,4);
      \foreach \i/\x in {0/0,1/1.5,2/3,3/4.5,4/6}{
        \foreach \j/\y in{0/0,1/1,2/2,3/3,4/4}{
          \coordinate(N\i\j)at(\x,\y);
        }
      }
      \foreach \A/\B/\C/\lambda in{%
        N00/N10/N01/.8, N10/N20/N11/.8,N20/N30/N21/.8,N30/N40/N31/.8,
        N11/N10/N01/.8, N21/N20/N11/.8,N31/N30/N21/.8,N41/N40/N31/.8,
        % 
        N01/N11/N02/.8, N11/N21/N12/.8,N21/N31/N22/.8,N31/N41/N32/.8,
        N12/N11/N02/.8, N22/N21/N12/.8,N32/N31/N22/.8,N42/N41/N32/.8,
        % 
        N02/N12/N03/.8, N12/N22/N13/.8,N22/N32/N23/.8,N32/N42/N33/.8,
        N13/N12/N03/.8, N23/N22/N13/.8,N33/N32/N23/.8,N43/N42/N33/.8,
        % 
        N03/N13/N04/.8, N13/N23/N14/.8,N23/N33/N24/.8,N33/N43/N34/.8,
        N14/N13/N04/.8, N24/N23/N14/.8,N34/N33/N24/.8,N44/N43/N34/.8        
      }{
        \RecursiveTriangle[\A,\B,\C,\lambda]
      }
    \end{tikzpicture}
  \end{center}
\end{example}

\begin{example}对四边形作类似的操作。
  \begin{center}
    \begin{tikzpicture}
      \begin{scope}
        \def\RecursiveTetragon[#1,#2,#3,#4,#5]{ %#1:N1, #2:N2, #3:N3, #4:\lambda
          \coordinate (A) at (#1); \coordinate(B) at (#2); \coordinate(C) at (#3); \coordinate(D) at(#4);
          %\draw(A)--(B)--(C)--(D)--cycle;
          \setcounter{X}{0}
          \whiledo{\value{X}<50}{%
            \coordinate(E) at ($#5*(B)+(C)-#5*(C)$);
            \draw(A)--(D);
            \coordinate(F)at(A);\coordinate(A)at(E);\coordinate(B)at(C);\coordinate(C)at(D);\coordinate(D)at(F);
            \stepcounter{X}
          }
        }

        \foreach \i/\x in {0/0,1/3,2/6}{
          \foreach \j/\y in{0/0,1/2,2/4}{
            \coordinate(N\i\j)at(\x,\y);
          }
        }

        \foreach \A/\B/\C/\D/\lambda in{%
          N01/N11/N10/N00/.8, N10/N20/N21/N11/.8,
          N01/N11/N12/N02/.8, N21/N11/N12/N22/.8%
        }{
          \RecursiveTetragon[\A,\B,\C,\D,\lambda]
        }
        \draw(N00)--(N20)--(N21)--(N01) (N11)--(N12) (N02)--(N22);
      \end{scope}
    \end{tikzpicture}
  \end{center}

  
\end{example}

\section{二进制}
\label{sec:binary-system}

\begin{example}
  有$1000$瓶水,其中有一瓶有剧毒(假设哪怕一个毒药分子在里面也能致命),其余的都是纯净水。蟑螂碰到剧毒药水在$2$小时内就会死亡。现在有$10$只蟑螂可以用来试验水是否有毒,有什么方法可以将有毒的一瓶找出来?哪种方法需要的时间最短?
\end{example}
\begin{proof}[提示]
  最直观的方法,一瓶一瓶的试,最坏的情况是试到最后一瓶才找出来有剧毒的水。需要的时间是$2\times1000=2000$小时。

  也可以用二分法,每次将范围缩小一半。第一次将1000瓶分两半,前500瓶各取一点混成一份,后500瓶各取一点混成一份,一只蟑螂喝其中一份,可分出剧毒水在哪500瓶里。以此类推,第二只蟑螂将500瓶缩小至250瓶。由于$2^{10}=1024>1000$,即使对最坏的情况,10只蟑螂也是够用的。在不考虑配制药水时间的情况下,这种方法需要的时间不超过最坏情况下的$2\times10=20$小时。

  还可以用二进制的方法。按以下方法配10瓶药水:
  \begin{quotation}
    将1000瓶药水编号,写出其编号的二进制形式,不足10位的在前面补零,使其编号都是十位的二进制,并排成一列,如下:
    \begin{center}
    \begin{tabular}{ccccccccccc}
      \toprule[2pt]
      \multirow{2}{*}{瓶编号} & \multicolumn{10}{c}{十位二进制编号}\\
      \cline{2-11} & 1 & 2 & 3 & 4 & 5 & 6 & 7 & 8 & 9 & 10\\
      \midrule
      1 & 0 & 0 & 0 & 0 & 0 & 0 & 0 & 0 & 0 & 1\\
      2 & 0 & 0 & 0 & 0 & 0 & 0 & 0 & 0 & 1 & 0\\
      3 & 0 & 0 & 0 & 0 & 0 & 0 & 0 & 0 & 1 & 1\\
      4 & 0 & 0 & 0 & 0 & 0 & 0 & 0 & 1 & 0 & 0\\
      5 & 0 & 0 & 0 & 0 & 0 & 0 & 0 & 1 & 0 & 1\\
      \multicolumn{11}{c}{$\cdots\cdots$}\\
      \hline
        & $1^*$ & 0 & $1^*$ & 0 & $1^*$ & 0 & 0 & 0 & 0 & 0\\
      \multicolumn{11}{c}{$^*$示例:喝了1、3和5号混合药水的蟑螂死了}\\  
      \bottomrule[2pt]
    \end{tabular}
    \end{center}

    用十位二进制编号第1位为1的药水配成第1瓶混合药水;

    用十位二进制编号第2位为1的药水配成第2瓶混合药水;

    用十位二进制编号第3位为1的药水配成第3瓶混合药水;
    
    $\cdots\cdots$

    如配制第10瓶药水,需要取第1、3、5、$\cdots$瓶药水混合。
  \end{quotation}

    配好药水后每个蟑螂分别喝一瓶混合药水。若喝了第$i$号混合药水的蟑螂死了,则说明剧毒药水的十位二进制编码在该位上为1,若蟑螂没死,则该位为0。两小时后观察十只蟑螂的存活状态,可得到剧毒药水的十位二进制编号。所需时间为配制药水所需的时间再加上两小时的观察时间。

    举例,如若第1、3、5号蟑螂死了,其余蟑螂没死,那么剧毒药水的十位二进制编号为$1010100000$,对应的十进制数为$2^9 + 2^7 + 2^5=672$。
\end{proof}