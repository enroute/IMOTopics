
\chapter{趣味集锦}
\label{chap:fun}

\begin{example}[户口调查员]
  网格员在敲开一户人家的门,并询问主人有几个小孩,孩子们都有几岁。主人说他有三个女儿,并且她们年龄的乘积是$36$。

  网格员说这些信息还不足以算出他的女儿们的年龄。

  主人又说:“我就是告诉你她们年龄的总和,你还是不能算出他们的年龄。”

  “我希望你能告诉我更多的信息。”

  “好吧,我的大女儿蓝若她喜欢读书。”

  这时,网格员知道这家三个小孩的年龄了。你知道吗?
\end{example}
\begin{proof}[解答]
  三个年龄乘积为$36$的组合只有以下几种(其中第二行为三人年龄之和):

\begin{center}
\begin{tabular}{cccccccc}
  \hline
  (1,1,36) & (1,2,18) & (1,3,12) & (1,4,9) & (1,6,6) & (2,2,9) & (2,3,6) & (3,3,4)\\
  \hline
  38       & 21       & 16       & 14      & 13      & 13      & 11      & 10\\
  \hline
\end{tabular}
\end{center}

其中和相等的只有两种组合,即$(1,6,6)$和$(2,2,9)$,其和均为$13$。由于主人说就算知道年龄之和也算不出来,可知三人年龄组合只能是$(1,6,6)$和$(2,2,9)$两种这一。而由最后一句,主人有一个最大的女儿,而不是两个一样大的大女儿,从而可知三人年龄组合是$(2,2,9)$。
\end{proof}

\begin{question}
  一个经理有三个女儿,三个女儿的年龄加起来等于13,三个女儿的年龄乘起来等于经理自己的年龄。有一个下属已知道经理的年龄,但仍不能确定经理三个女儿的年龄,这时经理说只有一个女儿的头发是黑的,然后这个下属就知道了经理三个女儿的年龄。请问三个女儿的年龄分别是多少?
\end{question}
\begin{proof}[提示]\let\qed\relax%remove qed symbol
  这里说的是中国人。中国小孩特别小的时候头发是黄的,所以有黄毛丫头一说。慢慢长大之后头发才开始变黑。考虑三数为13的组合情况。
  \begin{align*}\setlength\arraycolsep{3pt}\renewcommand*{\arraystretch}{.9}
    \begin{array}{cccccccccccccc}
      1 & \times & 1 & \times & 11 & = & 11, &\quad
      1 & \times & 2 & \times & 10 & = & 20\\
      1 & \times & 3 & \times &  9 & = & 27, &\quad
      1 & \times & 4 & \times &  8 & = & 32\\
      1 & \times & 5 & \times &  7 & = & 35, &\quad
      1 & \times & 6 & \times &  6 & = & \underline{36}\\
      2 & \times & 2 & \times &  9 & = & \underline{36}, &\quad
      2 & \times & 3 & \times &  8 & = & 48\\
      2 & \times & 4 & \times &  7 & = & 56, &\quad
      2 & \times & 5 & \times &  6 & = & 60\\
      3 & \times & 3 & \times &  7 & = & 63, &\quad
      3 & \times & 4 & \times &  6 & = & 72\\
      3 & \times & 5 & \times &  5 & = & 75, &\quad
      4 & \times & 4 & \times &  5 & = & 80
    \end{array}
  \end{align*}
\end{proof}

\begin{example}
  扎西多吉是个藏人,喜欢随身带着一柄$1.5$米长的长刀。有一次他出远门需要坐火车,但列车安全员不允许将长刀作为手提行李带上车。如果托运,火车的托运规定行李箱又不能超过$1$米。扎西多吉怎么才能合法地将他的长刀带上火车呢?
\end{example}
\begin{proof}[解答]
若局限在二维平面内考虑,是无解的。考虑一个$1\times1\times1$的箱子,其对角线长度为$\sqrt3>1.5$米,可以放下扎西多吉的长刀。
\end{proof}

\begin{example}
\renewcommand{\thefootnote}{\fnsymbol{footnote}}%use symbol rather than number in footnote markers
  唐宋明清铸造钱币的机构叫钱监。有一次,钱监里储存了10箱黄金,每箱100块,每块一两。有一个官员,为了捞油水,把某一个箱子里的每块黄金都磨掉了一钱\footnote{一两等于十钱}。如果允许带一个精度及量程都足够用的称重工具回去,如何能称一次就找到被磨了的那一箱黄金?
\end{example}
\begin{proof}[提示]\let\qed\relax%
  给箱子编号。称法的关键在于称出的重量能与箱子编号一一对应。1号取一块,2号取2块,3号取3块,$\cdots\cdots$,10号取10块,放在一起称。若全都是足称的金块,则应该是$1+2+3+\cdots+10=55$两,但某箱有缺陷,故实际称出来会比55两少。如果是1号被磨,则缺1钱;如果是2号被磨,则缺2钱。缺多少钱就是哪号箱子被磨了。
\end{proof}

\begin{question}
  药厂仓库保存了某种药片共8罐。后来由于异常断电仓库温度失控,导致某一罐中药片与空气发生了化学反应从而变质了,其每片的质量也重了1克。问如何只称一次就判断出是哪个罐子里的药片变质了?
\end{question}

\begin{example}[空瓶换酒]\label{ex:beer}
  超市里某种啤酒在作促销,两元一瓶的啤酒,若集齐两个空瓶子则可以免费换一瓶。小胖手里有10元钱,那么在这个超市里他的钱最多可以让他喝到几瓶啤酒?
\end{example}
\begin{proof}[提示]\renewcommand{\thefootnote}{\fnsymbol{footnote}}%use symbol rather than number in footnote markers
  方法一,借空瓶。买一瓶会得一空瓶,再加一空瓶即可换一瓶。因此2元买一瓶,喝完剩一空瓶,再借一空瓶,换一瓶,喝完又剩一空瓶,把空瓶还了。此时2元喝了两瓶,手里无剩余,无外债。即2元最多可喝2瓶。也就是10元最多可喝10瓶。

  方法二,计算啤酒不含瓶的成本。价值上,两个空瓶等于一整瓶,从而空瓶的成本是$2/2=1$元,啤酒(不含瓶)的成本是$2-1=1$元,从而10元能喝到的啤酒(不含瓶)的数量最多为$10/1=10$瓶。\footnote{此方法只给出了上限,没有证明存在某种方法能达到上限}
\end{proof}

\begin{question}
  在例~\ref{ex:beer}中,若集齐3个空瓶才能换一瓶,则又如何?
\end{question}
\begin{proof}[提示]
  借一空瓶,买两瓶,则4元可喝3瓶。8元可喝6瓶。剩两元可喝1瓶,剩余一个空瓶子。一个空瓶子不足以换取一瓶啤酒里的酒水(不含瓶),即最多可喝7瓶。
\end{proof}


\begin{example}
  推推开关是一种按一次就转变接通与断开状态的开关。现有10个推推开关对应10盏灯。开始时所有灯都是灭的。现进行如下操作(顺序不定):
  \begin{enumerate}
  \item 编号为1的倍数的开关都按一次;
  \item 编号为2的倍数的开关都按一次;
  \item 编号为3的倍数的开关都按一次;
  \item $\cdots\cdots$;
  \item 编号为10的倍数的开关都按一次。
  \end{enumerate}
  这样操作之后,问10盏灯的亮灭状态。
\end{example}
\begin{proof}[提示]\let\qed\relax
  首先,操作顺序不影响最终状态,因为某盏灯的最终状态只与对应开关被按的次数(精确地说是奇偶性)相关。开关总共被按了奇数次的灯是亮的,被按了偶数次的灯是灭的。

  而一个开关被按的次数,相当于其因数(包含1与其本身)的个数。

  \centering
  \begin{tabular}{clcccccccc}
    \toprule
    开关编号 & \multicolumn{1}{c}{因数} & 因数个数 & 灯最终状态\\\midrule
    1        & 1    & 1        & 亮\\
    2        & 1,2  & 2        & 灭\\
    3        & 1,3  & 2        & 灭\\
    4        & 1,2,4& 3        & 亮\\
    5        & 1,5  & 2        & 灭\\
    6        & 1,2,3,6 & 4        & 灭\\
    7        & 1,7  & 2        & 灭\\
    8        & 1,2,4,8 & 4        & 灭\\
    9        & 1,3,9   & 3        & 亮\\
    10       & 1,2,5,10& 4        & 灭\\
    \bottomrule
  \end{tabular}
\end{proof}
