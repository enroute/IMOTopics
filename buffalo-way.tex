
\chapter{Buffalo Way}
\label{chap:buffalo-way}

Buffalo Way(BW)是一种方法,通常用于解决对称性问题,对于$x_1\le x_2\le x_3\le \cdots\le x_n$,令
\begin{align*}
  x_1&=y_1\\
  x_2&=y_1+y_2\\
  x_3&=y_1+y_2+y_3\\
  \cdots&\cdots\\
  x_n&=y_1+y_2+y_3+\cdots+y_n
\end{align*}
其中除$y_1$外$y_2,y_3,\cdots y_n$均为非负数。代入原问题换元求解。

\begin{example}
  证明:若$0\le a\le b\le c$,则$(a+b)(a+c)^2\ge 6abc$。
\end{example}
\begin{proof}
  应用Buffalo Way,令$a=x$, $b=x+y$, $c=x+y+z$,其中$x$,$y$,$z$均为非负数。代入,有
  \begin{align*}
    &(a+b)(a+c)^2\ge 6abc\\
    \iff& (2x+y)(2x+y+z)^2\ge 6x(x+y)(x+y+z)
  \end{align*}
  化简后上式等价于
  \begin{align*}
    2x^3+2x^2z+2xyz+2xz^2+y^3+2y^2z+yz^2\ge 0
  \end{align*}
  由于$x,y,z$非负,上式显然成立,且等号成立时当且仅当以下各式同时满足
  \begin{align*}
    2x^3&=0,    &2x^2z&=0,  &2xyz&=0,  &2xz^2&=0\\
    y^3&=0,     &2y^2z&=0,  &yz^2&=0   &     &
  \end{align*}
  即$x=y=0$,$z$任意。即等号成立当且仅当$a=b=0$。
\end{proof}

\begin{example}
  证明:对于正数$x,y,z$,有
  \begin{align*}
    \sum_{\mathrm{cyc}} \frac{x^4}{8x^3+5y^3}\ge\frac{x+y+z}{13}
  \end{align*}
  其中$\sum\limits{\mathrm{cyc}}$表示轮换求和,对上式的$x$,$y$和$z$作轮换,则有
  \begin{align*}
    \sum_{\mathrm{cyc}}\frac{x^4}{8x^3+5y^3} \implies
    \begin{cases}
      \dfrac{x^4}{8x^3+5y^3} \xRightarrow{x\leftarrow x, y\leftarrow y} \dfrac{x^4}{8x^3+5y^3} \\
      \dfrac{x^4}{8x^3+5y^3} \xRightarrow{x\leftarrow y, y\leftarrow z} \dfrac{y^4}{8y^3+5z^3} \\
      \dfrac{x^4}{8x^3+5y^3} \xRightarrow{x\leftarrow z, y\leftarrow x} \dfrac{z^4}{8z^3+5x^3}
    \end{cases}%
  \end{align*}
  上面三式相加,可得:
  \begin{align*}
    \sum_{\mathrm{cyc}} \frac{x^4}{8x^3+5y^3} = \frac{x^4}{8x^3+5y^3}
    + \frac{y^4}{8y^3+5z^3}
    + \frac{z^4}{z^3+5x^3}
  \end{align*}
\end{example}
\begin{proof}
  应用BW,不妨设$0<x\le y\le z$,且$y=x+u$,$z=x+u+v$,代入。{\color{red}非常繁琐!如果想锻炼计算能力,可以一试;如果是竞赛题,基本可放弃此方法,需另辟蹊径。}
\end{proof}

\begin{example}
  证明:对任意正数$x,y,z$,有
  \begin{align*}
    \sum_{\mathrm{cyc}}\frac1{(x-y)^2}\equiv\frac1{(x-y)^2}+\frac1{(y-z)^2}+\frac1{(z-x)^2}\ge\frac4{xy+yz+zx}
  \end{align*}
\end{example}
\begin{proof}
  不失一般性,不妨设$0<x\le y\le z$,且$y=x+u,z=x+u+v$,其中$u,v$非负。代入经过繁琐的计算可得。
\end{proof}


可以看出,应用BW时其过程非常直观,不需要过多的思考,但伴随而来的缺点是通常会涉及比较繁琐的计算。BW方法并不是时时都有效,比如下面这一题。
\begin{example}
  证明:对于正数$x,y,z$,有
  \begin{align*}
    \sum_{\mathrm{cyc}} \frac{x^3}{13x^2+5y^2}\ge\frac{x+y+z}{18}
  \end{align*}
\end{example}
