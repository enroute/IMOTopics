
\chapter{数值方法}
\label{chap:numerical-method}

数值方法通常是指应用计算机程序求解数值问题的近似解的各种方法与工具。

其中常见的是迭代法,即不断地用旧值递推出新值的过程。比如数论中的辗转相除法也是一种迭代方法。

\section{迭代法}
\label{sec:iteration-method}

\subsection{韦达跳跃}
\label{sec:vieta-jumping}

韦达跳跃(Vieta Jumping)的其中一种方法,是利用韦达定理构造新点,从而问题跳转到新获得的点上。

\begin{example}[Putnam 1988]
  证明:对于任意正整数$N$,方程
  \begin{align*}
    x_1^2+x_2^2+x_3^2+x_4^2=x_1x_2x_3+x_2x_3x_4+x_3x_4x_1 + x_4x_1x_2
  \end{align*}
  总有$x_1,x_2,x_3$和$x_4$全大于$N$的整数解。
\end{example}
\begin{proof}[提示]用迭代法,从一个特殊解开始不断地构造更大的解。首先猜一个正整数解$(x_1,x_2,x_3,x_4)=(a_1,a_2,a_3,a_4)$。容易猜出来$(x_1,x_2,x_3,x_4)=(1,1,1,1)$是一个正整数解。

  其次,从$(a_1,a_2,a_3,a_4)$出发,构造另一组更大的整数解。由对称性,不妨假设$a_1\le a_2\le a_3\le a_4$。保持较大的$a_2,a_3,a_4$不动,若将$a_1$换成一个比$a_4$大的正整数$a_5$且$(a_2,a_3,a_4,a_5)$满足方程的话,则$\left(a_2, a_3,a_4, a_5\right)$就是另一组更大的解,且有
  $a_1\le a_2\le a_3\le a_4 < a_5$。用同样的方式,可以找到$a_6, a_7, a_8, \cdots$,使得序列
  \begin{align*}
    a_1\le a_2\le a_3\le a_4 < a_5 < a_6 < a_7 < a_8 < \cdots
  \end{align*}
  中连续的四项为原方程的一组解。

  下面考虑得到另一个解。对于$x_1$,原方程是一个二元方程,$a_1$是其中一个解,则由韦达定理可得到另一个关于$x_1$的解。变换形式如下:
  \begin{align*}
    x_1^2 - (x_2x_3+x_3x_4+x_4x_2)x_1 + x_2^2 + x_3^2 + x_4^2 - x_2x_3x_4=0
  \end{align*}
  关于$x_1$的方程的两个根$u, v$应满足
  \begin{align*}
    u + v = x_2x_3+x_3x_4+x_4x_2
  \end{align*}
  把其中一个根$x_1=a_1$及系数$x_2=a_2, x_3=a_3, x_4=a_4$代入,则有另一个根(记为$a_5$)为
  \begin{align*}
    a_5 = a_2a_3 + a_3a_4 + a_4a_2 - a_1
  \end{align*}
  又显然有$a_5  \ge a_3a_4 + a_4a_2 > a_4$,从而按此公式,可得到序列的下一项。

  总结如下。开始项:$(1,1,1,1)$,下一项的递推公式为
  \begin{align*}
    a_n = %f(a_{n-1},a_{n-2},a_{n-3},a_{n-4}) =
    a_{n-1}a_{n-2} + a_{n-2}a_{n-3} + a_{n-3}a_{n-1} - a_{n-4}
  \end{align*}
  这种苦力,用计算机容易得到下面的结果:
  \begin{align*}
    1,1,1,1,2, 4, 8, 16, 32, 64, 128, 256, 512, 1024, 2048,\cdots
  \end{align*}
  更进一步地,由上面这个结果,还能得到什么更有趣的结果?
\end{proof}

\begin{example}[IMO 1988]
  $a,b$是正整数,且$ab + 1|a^2+b^2$,证明$\frac{a^2+b^2}{ab+1}$是完全平方数。
\end{example}
\begin{proof}[提示]
  这是一道非常有名的题目。即使是当年的金牌获得者,13岁的陶哲轩(Terrence Tao,最年轻的金牌获得者,后面于2006年获得了fields medal)在此题上也只是拿到了7分中的1分。当时只有11名学生完美地解决了这个问题。

  用韦达跳跃及反证,即从最小的$(a,b)$出发,通过韦达跳跃跳到一个比$(a,b)$更小的点上,从而导致矛盾。

  \begin{enumerate}
  \item 记 $S\equiv \{(a,b)|a,b\text{是正整数且}\dfrac{a^2+b^2}{ab+1}\text{是正整数但不是完全平方数}\}$。
  \item 若$S$非空,找出其中一个$(A,B)$使得$A+B$最小,即记$(A,B)\in S$,且有$\forall (a,b)\in S$,有
    \begin{align*}
      A + B \le a + b
    \end{align*}
    由对称性不妨设$A\ge B$。下面由韦达跳跃找到一个更小的来导出矛盾。
  \item 固定$B$,若能找到$C$,使$(B,C)\in S$且$C<A$,则$B+C<A+B$与$(A,B)$的定义矛盾。设$k = \frac{A^2+B^2}{AB+1}$,若同时有$k = \frac{C^2+B^2}{CB+1}$,则$A,C$都是以下方程的解:
    \begin{align*}
      \frac{x^2 + B^2}{Bx + 1} = 0 \implies x^2 - kxB + B^2 - k = 0
    \end{align*}
    即若$A,C$是这个方程的角,则由韦达定理,有
    \begin{align*}
      C = kB - A, \quad\text{且}\quad C = \frac{B^2 - k}{A}
    \end{align*}

  \item 由$C=kB - A$知$C$是整数;由$C = \frac{B^2 - k}{A}$且$k$不是完全平方数知$C\ne 0$;再由$\frac{C^2 + B^2}{CB + 1}=k>0 \implies CB+1>0 \implies C>0$。从而$C$是正整数,且$(C,B)\in S$且$C+B = \frac{B^2-k}{A} + B < A + B$。
  \end{enumerate}
  由此矛盾可知$S$中不存在最小的$(A,B)$,从而$S$只能是空集。
\end{proof}


\section{牛顿法}
\label{sec:Newton-method}

牛顿法是一种迭代方法。迭代方法的思想是通过一个初始点,不断地迭代得到一系列的点,而这些点序列最终会收敛到目标点上。具体来说,牛顿法是通过曲线的切线来构造下一个点,因而牛顿法通常也叫做切线法。

\begin{enumerate}
\item 从一个初始点$(x_0, 0)$开始,找到曲线上的点$(x_0, f(x_0))$;
\item 作切线,找到切线与$x$轴的交点$(x_1,0)$;
\item 从$(x_1,0)$开始重复上述步骤
\end{enumerate}

按上述步骤,可得到一系列的值$x_0, x_1, x_2, \cdots, $。在某些条件下,$x_i$会收敛到曲线与$x$轴的交点。其递推关系为
\begin{align*}
  x_{n+1}=x_0 - \frac{f(x_0)}{f'(x_0)}
\end{align*}
上述迭代的不动点(即迭代完了之后还是它本身),在$f'$不为零的情况下等价于$f(x)$的零点。这是因为
\begin{align*}
  x=x-\frac{f(x)}{f'(x)} \quad \overset{f'\ne0}{\iff}\quad f(x)=0
\end{align*}

\begin{theorem}
  若函数$f(x)$在$x=\alpha$处为零,即$f(\alpha)=0$,且$f(x)$有连续的导数,且$f'(\alpha)\ne0$。则存在$x=\alpha$的一个邻域,在此邻域内任意一点$x_0$出发按牛顿切线法得出的点序列$\{x_i\}$收敛于$\alpha$。
\end{theorem}
由此定理可知,有连续导数的函数的零点如果其导数不会零,则该零点有一个吸引域,在此吸引域内的点按牛顿切线法得到的序列总会收敛于该零点。

\begin{figure}[htbp]
  \centering
  \begin{tikzpicture}[scale=1.0]
    \begin{scope}[shift={(0,0)},decoration={
        markings,
        mark=at position 0.5 with {\arrow{>}}
      }]
      \draw[->](-1,0)--(11,0)node[right]{$x$};
      \draw[->](0,-2)--(0,9)node[above]{$y$};
      \draw[very thick,domain=-.75:9.5,smooth,variable=\x]plot({\x},{.1*(\x-1)*(\x-1)-1});
      \coordinate[label=below:$x_0$] (x0) at (9.3,0);
      \coordinate[label=above left:{$\left( x_0,f(x_0)\right)$}] (f0) at (9.3,5.889);
      \coordinate[label=below:$x_1$] (x1) at (5.7524,0);
      \coordinate (f1) at (5.7524,1.2585);
      \coordinate[label=below:$x_2$] (x2) at (4.4283,0);
      \coordinate (f2) at (4.4283,0.1753);
      \draw[help lines,postaction={decorate}](x0)--(f0);
      \draw[help lines,postaction={decorate}](f0)--(x1);
      \draw[help lines,postaction={decorate}](x1)--(f1);
      \draw[help lines,postaction={decorate}](f1)--(x2);%--(f2);
      \tkzDrawPoints(x0,f0,x1,f1,x2)
      \draw[help lines,|<->|](9.3,-1)--(5.7524,-1)node[midway,fill=white,black]{$\dfrac{f(x_0)}{f'(x_0)}$};
      \draw[help lines, <->|](10,0)--(10,5.889)node[midway,fill=white,black]{$f(x_0)$};
      \draw pic["$\theta$",<->,draw=orange,angle eccentricity=1.6,angle radius=.6cm]{angle=x0--x1--f0};
      \coordinate (f) at ($.7*(x1)+.3*(f0)$);
      \coordinate[label=above:{切线斜率$\tan\theta=f'(x_0)$}] (t) at (5,3);
      \draw[dashed,->,help lines](f)--(t);
    \end{scope}
    \begin{scope}[shift={(0,0)}]
      
    \end{scope}
  \end{tikzpicture}
  \caption{牛顿切线法}
  \label{fig:numeric-newton-method}
\end{figure}

\section{二分迭代}
\label{sec:bi-section-iteration-in-numerical-analysis}

\begin{example}
  用数值方法求$x^3+x^2=1$的一个实根。
\end{example}
\begin{proof}[提示]
  实系数方程若有虚根,则都是成对出现的,从而3次方程至少有一个实根。令$f(x)=x^3+x^2-1$,则$f(1)=1>0$,$f(0)=-1<0$,由$f(x)$的连续性,其曲线在$[0,1]$内必穿过$x$轴,即在$[0,1]$必有一实根。利用二分法,可以不断缩小实根所在的区间。选择新区间的关键点在于函数值在两端点处符号要相反,即一正一负。
  \begin{table}[htbp]
    \centering
    \caption{二分法求实根}
    \label{tab:real-root-by-binary-section}
    \begin{tabular}{ccccccc}
      \hline
      迭代 & $x_1$ & $f(x_1)$ & $x_2$ & $f(x_2)$ & $\dfrac{x_1+x_2}{2}$ & $f\left(\dfrac{x_1+x_2}{2}\right)$\\\hline
      1 & 0 & $<0$ & 1 & $>0$ & 0.5  & $< 0$\\
      2 & 0.5 &  & 1 &   & 0.75 & $<0$\\
      3 & 0.75 & & 1 &   & 0.875 & $>0$\\
      4 & 0.75 & & 0.875&& 0.8125 & $ >0$\\
      5 & 0.75 & & 0.8125&& 0.78125 & $>0$\\
      6 & 0.75 & & 0.78125&& 0.765625 & $>0$\\
      7 & 0.75 & & 0.765625&& 0.7578125 & $>0$\\
      8 & 0.75 & & 0.7578125&& $\cdots$ & $\cdots$\\
      \hline
    \end{tabular}
  \end{table}
\end{proof}