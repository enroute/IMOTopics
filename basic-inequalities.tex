
\chapter{不等式}
\label{chap:inequality}


\section{基本不等式}
\label{sec:basic-inequalities}

\begin{definition}[算术平均数,Arithmetic Means, AM)]
  $\forall x_i$,下式称为其代数平均数
  \begin{align*}
    A_n\equiv\frac{\sum\limits_{i=1}^{n} x_i}{n}
    =\frac{x_1+x_2+x_3+\cdots+x_n}{n}
  \end{align*}
\end{definition}

\begin{definition}[几何平均数, Geometric Means, GM]
  $\forall x_i\ge0$,下式称为其几何平均数
  \begin{align*}
    G_n\equiv\sqrt[n]{\prod_{i=1}^{n}x_i}
    =\sqrt[n]{x_1 x_2 x_3\cdots x_n}
  \end{align*}
\end{definition}

\begin{definition}[调和平均数,Harmonic Means, HM]
  $\forall x_i\ge0$,下式称为其调和平均数
  \begin{align*}
    H_n\equiv\frac{n}{\sum\limits_{i=1}^{n}\dfrac1{x_i}}
    =\frac{n}{\dfrac1{x_1}+\dfrac1{x_2}+\cdots+\dfrac1{x_n}}
    =\frac{1}{\dfrac{\dfrac1{x_1}+\dfrac1{x_2}+\cdots+\dfrac1{x_n}}{n}}
  \end{align*}
\end{definition}

\begin{definition}[平方平均数,Quadratic Mean,QM]
  $\forall x_i$,下式称不其平方平均数
  \begin{align*}
    Q_n\equiv\sqrt{\dfrac{\sum\limits_{i=1}^n x_i^2}{n}}
    =\sqrt{\frac{x_1^2+x_2^2+\cdots+x_n^2}{n}}
  \end{align*}
  平均平方数也称为\term{均方根}(Root Mean Square)。
\end{definition}

若令
\begin{align*}
  \varphi(x_1,x_2,\cdots,x_n;p)\equiv\left(\frac{\sum_{i=1}^{n} x_i^p}{n} \right)^{\frac1p}
  =\left( \frac{x_1^p + x_2^p + \cdots + x_n^p}{n} \right)^{\frac1p}
\end{align*}
则显然有
\begin{align*}
  H_n \equiv \mathrm{HM}(x_1,x_2,\cdots,x_n)&=\varphi(x_1,x_2,\cdots,x_n;-1)\\
  A_n \equiv \mathrm{AM}(x_1,x_2,\cdots,x_n)&=\varphi(x_1,x_2,\cdots,x_n;1)\\
  Q_n \equiv \mathrm{QM}(x_1,x_2,\cdots,x_n)&=\varphi(x_1,x_2,\cdots,x_n;2)
\end{align*}

思考:是否能推出固定$x_1,x_2,\cdots,x_n$,则$\varphi(x_1,x_2,\cdots,x_n;p)$关于$p$是递增函数?若可以,则显然有$H_n\le A_n\le Q_n$。

\begin{theorem}[HM-GM-AM-QM]
  任意正数序列$x_i$,有
  \begin{align*}
    H_n\le G_n\le A_n\le Q_n
  \end{align*}
\end{theorem}
\begin{figure}[htbp]
  \centering
  \begin{tikzpicture}[scale=1.0]
    \draw[help lines](-6,0)--(6,0) arc (0:180:6);
    \draw[help lines,|<->|](-6,-0.6)--(3,-0.6) node[midway,fill=white]{$x_1$};
    \draw[help lines,|<->|](3,-0.6)--(6,-0.6) node[midway,fill=white]{$x_2$};
    \draw[help lines,|<->|](-6,-1.5)--(0,-1.5)node[midway,fill=white]{$\dfrac{x_1+x_2}2$};
    \draw[help lines,|<->|](0,-1.5)--(3,-1.5)node[midway,fill=white]{$\dfrac{x_1-x_2}2$};
    \coordinate (G) at (60:6);
    \coordinate (H) at (60:1.5);
    \tkzDefPoint(3,0){C}
    \tkzDefPoint(0,0){O}
    \tkzMarkRightAngle[draw,fill=white](O,H,C)
    \draw[very thick, blue](G)--(C) node[sloped,midway,fill=white]{GM};
    \draw[very thick, red](0,6)--(0,0) node[midway,sloped,fill=white]{AM};
    \draw[very thick](0,6)--(C) node[sloped,pos=0.3,fill=white]{QM};
    \draw[very thick, violet](H)--(G) node[sloped,pos=0.61,fill=white]{HM};
    \draw[very thick](-6,0)--(6,0);
    \fill (-6,0) circle(3pt);
    \fill (C) circle(3pt);
    \fill (6,0) circle(3pt);
    \fill (G) circle(3pt);
    \fill (H) circle(3pt);
    \fill (0,6) circle(3pt);
    \fill (O) circle(3pt);
    \draw[dashed](0,0)--(H);
    \draw[dashed](3,0)--(H);
  \end{tikzpicture}
  \caption{$n=2$时HM,GM,AM,QM的几何示意}
  \label{fig:HM-GM-AM-QM}
\end{figure}

\begin{lemma}\label{lemma:b-1-a}
  若$b\le1\le a$,则$ab\le a+b-1$。
\end{lemma}
\begin{proof}
  由$0\le (a-1)(1-b)=a+b-ab-1$可得。此引理比较重要,被应用到很多不等式的证明过程中。
\end{proof}

\begin{lemma}\label{lemma:product-ai-ge-n}
  $\forall a_i>0$,若满足$\prod_{i=1}^n a_i=1$,则以下不等式成立
  \begin{align}
    \sum_{i=1}^n a_i\ge n
  \end{align}
  当且仅当$a_1=a_2=\cdots=a_n=1$时上述等号成立。
\end{lemma}
\begin{proof}
  对$n$作数学归纳。当$n=1$时显然。当$n\ge2$时,不妨对$a_i$重新排列使得$a_1\equiv\max(a_1,a_2,\cdots,a_n)\ge1$,$a_2\equiv\min(a_1,a_2,\cdots,a_n)\le1$,那么以下$n-1$个数的序列
  \begin{align*}
    a_1a_2, a_3, a_4, \cdots, a_n
  \end{align*}
  满足$n-1$的情况,从而有
  \begin{align*}
    a_1a_2 + a_3+ a_4+ \cdots + a_n &\ge n-1
  \end{align*}
  当且仅当$a_1a_2=a_3=a_4=\cdots=a_n=1$时等号成立,又由引理\ref{lemma:b-1-a},$a_1+a_2 - 1\ge a_1a_2$,其中等号当且仅当$a_1=1$或者$a_2=1$时成立。代入后有
  \begin{align*}
    (a_1 + a_2 - 1) + a_3+ a_4+ \cdots + a_n &\ge n-1\\
    a_1 + a_2 + \cdots + a_n\ge n
  \end{align*}
  其中等号成立,当且仅当以下两个条件同时满足:
  \begin{enumerate}
  \item $a_1a_2=a_3=a_4=\cdots=a_n=1$
  \item $a_1=1$或者$a_2=1$
  \end{enumerate}
  由此条件可知当且仅当所有的$a_i$都等于1时等号成立。
\end{proof}

下面证明AM-GM不等式。
\begin{proof}
  令$g\equiv\sqrt[n]{\prod\limits_{i=1}^n a_i}$,对序列
  \begin{align*}
    \frac{a_1}{g}, \frac{a_2}{g}, \frac{a_3}{g}, \cdots, \frac{a_n}{g}
  \end{align*}
  应用引理\ref{lemma:product-ai-ge-n},有
  \begin{align*}
    \frac{a_1}{g} + \frac{a_2}{g} + \frac{a_3}{g} + \cdots + \frac{a_n}{g}&\ge n\\
    \frac{a_1 + a_2 + a_3 + \cdots + a_n}{n}&\ge g = \sqrt[n]{a_1a_2a_3\cdots a_n} \qedhere
  \end{align*}
\end{proof}

对于三角形,HM还有以下几何意义:
\begin{example}
  对任意三角形,其内切圆的半径是三角形三条高长度的调和平均值的三分之一,即
  \begin{align*}
    r=\frac13 HM(h_a,h_b,h_c)
  \end{align*}
\end{example}
\begin{proof}
  尝试用面积来解,如图~\ref{fig:incircle}所示设三边边长分别是$a,b,c$,
  其对应的高分别为 $h_a, h_b, h_c$,则三角形面积$S=\frac12
  ah_a = \frac12 bh_b = \frac12 ch_c$,同样有$S=\frac12 r(a+b+c)$,从而
  \begin{align*}
    & S = \frac12 r\left( \frac{2S}{h_c} +\frac{2S}{h_a} + \frac{2S}{h_b} \right)
    \quad \implies\quad  r=\frac{1}{\dfrac1{h_a} + \dfrac1{h_b} + \dfrac1{h_c}} \qedhere
  \end{align*}
\end{proof}

\begin{figure}[htbp]
  \centering
  \begin{tikzpicture}[scale=1]
    \tkzDefPoint[label=below left:$A$](0,0){A}
    \tkzDefPoint[label=below right:$B$](6,0){B}
    \tkzDefPoint[label=above:$C$](5,5){C}
    
    \tkzDefCircle[in](A,B,C)\tkzGetPoint{I}\tkzGetLength{rIN}
    \tkzDrawCircle[R](I,\rIN pt);
    \tkzDrawSegments(A,B B,C C,A)
    \tkzDrawSegments[dashed](I,A I,B I,C)
    \tkzDrawCircle[fillstyle=solid,R](I,2pt)

    \coordinate(IA) at ($(B)!(I)!(C)$);
    \coordinate(IB) at ($(C)!(I)!(A)$);
    \coordinate(IC) at ($(A)!(I)!(B)$);

    \tkzDrawSegments[dashed](I,IA I,IB I,IC)
    \tkzMarkRightAngle[color=blue](B,IC,I)
    \tkzMarkRightAngle[color=blue](C,IA,I)
    \tkzMarkRightAngle[color=blue](A,IB,I)
  \end{tikzpicture}
  \caption{三角形内切圆}
  \label{fig:incircle}
\end{figure}

\begin{example}
  若任意非负数$a,b,c$满足$(a+1)(b+1)(c+1)=8$,则$abc\le 1$。
\end{example}
\begin{proof}
  若$a,b,c$均非负,则由AM-GM不等式,有$a+1\ge 2\sqrt{a}$,从而
  \begin{align*}
    8=(a+1)(b+1)(c+1)\ge 2\sqrt{a}\times 2\sqrt{b} \times 2\sqrt{c}
  \end{align*}
  即$abc\le1$。当且仅当$a=b=c=1$时等号成立。
\end{proof}

上述条件不能扩展到任意实数,$a,b,c$一负两正或者三个都是负数,则显然$abc<0$;但对
于$a,b,c$两负一正,比如$a=-2,b=-2,c=7$,则$abc>1$。



\begin{question}\label{q:1/1+a+b}
  三个正数$a,b,c$的乘积是1,求证
  \begin{align*}
    \frac1{1+a+b} + \frac1{1+b+c} + \frac1{1+c+a} \le 1
  \end{align*}
  当且仅当$a=b=c=1$时等号成立。
\end{question}

先尝试猜测每一项的上界。
\begin{align*}
  1+a+b\ge 3\sqrt[3]{ab},\quad 1+b+c \ge 3\sqrt[3]{bc},\quad 1+c+a\ge 3\sqrt[3]{ca}
\end{align*}
从而
\begin{align*}
  \frac1{1+b+c} + \frac1{1+c+a} + \frac1{1+a+b} \le
  \frac13\left( \frac1{\sqrt[3]{ab}} + \frac1{\sqrt[3]{bc}} + \frac1{\sqrt[3]{ca}} \right)
\end{align*}
令$a\to0+, b\to 0+, c=\frac1{ab}$,则上式$\le$符号右边$\to+\infty$,从而没上界,此估计无用。

% 又任意正数$x,y$,有
% \begin{align*}
%   \sqrt{xy}\ge \frac2{\dfrac1x + \dfrac1y}%
%   % \implies  \frac1{xy}\le \frac2{\dfrac1{x^2} + \dfrac1{y^2}}
% \end{align*}

% 令$a_0=\sqrt[6]a, b_0=\sqrt[6]b, c_0=\sqrt[6]c$,代入,有
% \begin{align*}
%   \frac1{\sqrt[3]{ab}} = \frac1{\sqrt{a_0b_0}}\le \frac2{\dfrac1{a_0} + \dfrac1{b_0}}
%   =\frac{2a_0b_0}{a_0+b_0}
% \end{align*}
% 同理处理$\frac1{\sqrt[3]{bc}}$和$\frac1{\sqrt[3]{ca}}$,从而有
% \begin{align*}
%   \frac1{1+b+c} + \frac1{1+c+a} + \frac1{1+a+b} \le
%   \frac23 \left(
%   \frac{a_0b_0}{a_0+b_0} + \frac{b_0c_0}{b_0+c_0} + \frac{c_0a_0}{c_0+a_0} 
%   \right)
% \end{align*}
% 其中$a_0b_0c_0=1$,且上式当且仅当$a_0=b_0=c_0=1$时等号成立。{\color{red}往下似乎不好走了,尝试换个方法。}


\begin{lemma}
对任意$a,b,c\in\mathcal{R}$,有以下恒等式:
\begin{align*}
  (a+b+c)(ab+bc+ca)&=a^2b+a^2c+b^2c+b^2a+c^2a+c^2b + 3abc\\
                   &=(a+b)(b+c)(c+a)+abc
\end{align*}
\end{lemma}
\begin{proof}
  左右分别展开可得证。
\end{proof}

令$x=b+c, y=c+a, z=a+b$,代入有
\begin{align*}
  & \frac1{1+b+c}+\frac1{1+c+a}+\frac1{1+a+b}\le 1\\
  \iff & \underline{(1+x)(1+y)} + (1+y)(1+z) + (1+z)(1+x) \le \underline{(1+x)(1+y)(1+z)}\\
  \iff & \underline{z(1+x)(1+y)} - \underline{(1+y)(1+z)} - (1+z)(1+x) \ge 0\\
  \iff & \underline{zx(1+y)} - (1+y) - 1-x-z-\underline{zx}\ge 0\\
  \iff & zxy - 2 - (x+y+z)\ge 0\\
  \iff & (a+b)(b+c)(c+a) - 2 - 2(a+b+c)\ge 0\\
  \iff & (a+b+c)(ab+bc+ca) - abc -2(a+b+c)\ge 2 \quad(\text{把}abc=1\text{代入})\\
  \iff & (a+b+c)(ab+bc+ca-2)\ge 3
  % \iff & (1+c+a)(1+a+b) + (1+b+c)(1+a+b) + (1+b+c)(1+c+a) \le (1+b+c)(1+c+a)(1+a+b)\\
  % \iff & (b+c)(1+c+a)(1+a+b) - (1+b+c)(1+a+b) - (1+b+c)(1+c+a) \ge 0\\
  % \iff & (b+c)(c+a)(1+a+b) - (1+a+b) - (1+b+c)(1+c+a) \ge 0\\
  % \iff & (b+c)(c+a)(a+b) - (1+a+b) - (1+c+a) - (b+c) \ge 0\\
  % \iff & (b+c)(c+a)(a+b) - 2(1+a+b+c)\ge 0\\
  % \iff & cab+b^2c+a^2b+ab^2 +c^2a+bc^2+a^2c+abc - 2(1+a+b+c)\ge 0\\
  % \iff & b^2c+a^2b+ab^2 +c^2a+bc^2+a^2c - 2(a+b+c)\ge 0\\
  % \iff & (a+b+c)(ab+bc+ca)-3abc -2(a+b+c)\ge 0\\
  % \iff & (a+b+c)(ab+bc+ca-2)\ge 3
\end{align*}
而由AM-GM不等式,有$a+b+c\ge 3\sqrt[3]{abc}=3, ab+bc+ca\ge 3\sqrt[3]{ab\cdot bc\cdot ca}=3$,可得。


\begin{question}
  上题中能否推广到任意正整数$n$,若正数$a_1,a_2,\cdots,a_n$的乘积是$1$,且令$S$表示其和,即
  \begin{align*}
    S\equiv\sum_{i=1}^n a_i
  \end{align*}
  则有
  \begin{align*}
    \sum_{i=1}^n \frac{1}{1 - a_i + S}\le 1
  \end{align*}
  当且仅当$a_1=a_2=\cdots=a_n=1$时等号成立?
\end{question}

当$n=1$时,$a_1=1$,有
\begin{align*}
  \sum_{i=1}^n \frac{1}{1 - a_i + S} = \frac{1}{1} = 1
\end{align*}

当$n=2$时,由$a_1a_2=1$,代入后恒有
\begin{align*}
  \sum_{i=1}^n \frac{1}{1 - a_i + S} &= \frac{1}{1+a_2} + \frac1{1+a_1}\\
                                     &= \frac{1+a_1 + 1 + a_2}{(1+a_1)(1+a_2)}\\
                                     &=1
\end{align*}

当$n=3$时,由题\ref{q:1/1+a+b}已得证。若用引理\ref{lemma:product-ai-ge-n},有$S\ge 3$,从而
\begin{align*}
  \sum_{i=1}^n \frac{1}{1 - a_i + S} &\le \sum_{i=1}^n \frac{1}{1 - a_i + 3}\\
                                     &=   \sum_{i=1}^n \frac{1}{4 - a_i}\\
\end{align*}
{\color{red}上式可不一定,比如$a_i>4$呢?这方法不一定可行。}

用数学归纳法。设$n\le k$时成立,考虑$n=k+1$的情况。考虑数列$\{a_1,a_2,\cdots, a_{k-1}, a_ka_{k+1}\}$,共$k$个数,且其乘积为$1$,从而有
\begin{align*}
  \sum_{i=1}^{k} \frac{1}{1 - a_i' + S_{k+1}'}
\end{align*}
其中
\begin{align*}
  a_i'=
  \begin{cases}
    a_i &i=1,2,\cdots,k-1\\
    a_ka_{k+1} &i=k
  \end{cases} \quad
  S_{k+1}'= a_1+a_2+\cdots +a_{k-1} + a_ka_{k+1}
\end{align*}

%%%%%%%%%%%%%%%%%%%%%%%%%%%%%%%%%%%%%%%%
%%% Basic inequality examples
%%%%%%%%%%%%%%%%%%%%%%%%%%%%%%%%%%%%%%%%
\begin{example}
  证明对$\forall a,b,c\in\mathcal{R}$,有$a^2+b^2+c^2\ge ab+bc+ca$。
\end{example}
\begin{proof}
  由AM--GM不等式,有
\begin{align*}
  a^2+b^2\ge 2ab,\quad b^2+c^2\ge 2bc,\quad c^2+a^2\ge 2ca
\end{align*}
三式相加并除以2可得。该不等式可以推广:$\forall n>1, x_i\in\mathcal{R}(i=1,2,\cdots,n)$,有
\begin{align*}
  \sum_{i=1}^n x_i^2\ge \sum_{i=1}^n x_ix_{i+1}
\end{align*}
其中$x_{n+1}=x_1$。
\end{proof}

\begin{example}
  对任意非负数$a,b$及正整数$n\ge2$,有 $(n-1)a^n + b^n\ge na^{n-1}b$。
\end{example}
对序列$a^n,a^n,\cdots,a^n,b^n$(其中有$n-1$项$a^n$应用AM-GM不等式,有)
\begin{align*}
  \underbrace{a^n + a^n + \cdots + a^n}_{n-1\text{个}} + b^n \ge
  n\times \sqrt[n]{\left(a^n\right)^{n-1} b^n}
  = na^{n-1}b
\end{align*}
当$n=3,4,5$时,有
\begin{align*}
  2a^3 + b^3&\ge 3a^2b\\
  3a^4 + b^4&\ge 4a^3b\\
  4a^5 + b^5&\ge 5a^4b
\end{align*}

\begin{example}
  若正数$a,b,c$的乘积为$1$,求$ab+bc+ca$的极值。
\end{example}
\begin{proof}[提示]
\begin{align*}
  ab+bc+ca\ge 3\times\sqrt[3]{ab\cdot bc\cdot ca}=3\times\sqrt[3]{(abc)^2}=3
\end{align*}
当且仅当$a=b=c=1$时等号成立。另一方面,令$a=b=n,c=1/n^2$,则
\begin{align*}
  ab+bc+ca>ab=n^2\to+\infty (n\to+\infty)
\end{align*}
即$ab+bc+ca$无上界。
\end{proof}


\begin{example}
  若正数$a,b,c$的乘积为$1$,求$a+b+c$的极值。
\end{example}
\begin{align*}
  a+b+c\ge 3\times\sqrt[3]{abc}=3
\end{align*}
当且仅当$a=b=c=1$时等号成立。另一方面,令$a=b=n,c=1/n^2$,则
\begin{align*}
  a+b+c>a=n\to+\infty(n\to+\infty)
\end{align*}
即$a+b+c$无上界。

\begin{example}
  找出所有满足下列等式的实数$a,b,c,d$
  \begin{align*}
    a^2+b^2+c^2+d^2=a(b+c+d)
  \end{align*}
\end{example}

$\forall a,b,c,d\in\mathcal{R}$,有
\begin{align*}
  \left(\frac a2\right)^2\ge 0,\quad \left(\frac a2\right)^2 + b^2\ge ab,
  \quad \left(\frac a2\right)^2 + c^2\ge ac, \quad \left(\frac a2\right)^2 + d^2\ge ad
\end{align*}
四式相加,有
\begin{align*}
  a^2+b^2+c^2+d^2\ge a(b+c+d)
\end{align*}
当且仅当$a=0, \frac a2=b=c=d$时即$a=b=c=d=0$时成立。即原题只有$a=b=c=d=0$这一个解。

\begin{example}
  若正数$a,b,c$的平方和为1,求下式的最小值
  \begin{align*}
    S=\frac{a^2b^2}{c^2} + \frac{b^2c^2}{a^2} + \frac{c^2a^2}{b^2}
  \end{align*}
\end{example}

由
\begin{align*}
  \frac{a^2b^2}{c^2} + \frac{b^2c^2}{a^2}\ge 2b^2,\quad
  \frac{b^2c^2}{a^2} + \frac{c^2a^2}{b^2}\ge 2c^2,\quad
  \frac{c^2a^2}{b^2} + \frac{a^2b^2}{c^2}\ge 2a^2
\end{align*}
三式相加并除以2,有
\begin{align*}
  S\ge a^2+b^2+c^2=1
\end{align*}
当且仅当$\dfrac{a^2b^2}{c^2} = \dfrac{b^2c^2}{a^2} = \dfrac{c^2a^2}{b^2}$即$a=b=c=\frac1{\sqrt3}$时等号成立。


\begin{example}
  $x,y$是小于1的正数,则
  \begin{align*}
    \frac1{1-x^2} + \frac1{1-y^2} \ge \frac2{1-xy}
  \end{align*}
\end{example}
由$x+y\ge2\sqrt{xy}$,有
\begin{align*}
  \frac1{1-x^2} + \frac1{1-y^2} &\ge \frac2{\sqrt{(1-x^2)(1-y^2)}} &&\text{等号成立}\iff x=\pm y\\
  &=\frac2{\sqrt{1+x^2y^2-x^2-y^2}} \\
  &\ge \frac2{\sqrt{1+x^2y^2-2xy}} &&\text{等号成立}\iff x=y\\
  &=\frac2{\sqrt{(1-xy)^2}} \\
  &=\frac2{1-xy}
\end{align*}
当且仅当$x=y$时等号成立。

\begin{example}
  对任意非负数$a,b$,有
  \begin{align*}
    a^3+b^3\ge a^2b+ab^2
  \end{align*}
  推广:对任意非负数$a,b,c$,有
  \begin{align*}
    a^3+b^3+c^3\ge a^2b+b^2c+c^2a
  \end{align*}
  上述结论对任意$n>1$个非负数$a_1,a_2,\cdots,a_n$是否成立,即
  \begin{align*}
    \sum_{i=1}^n a_i^3\ge a_1^2a_2 + a_2^2a_3 + \cdots + a_{n-1}^2a_n + a_n^2a_1
  \end{align*}
\end{example}
由$a^3+b^3-a^2b-ab^2=a^2(a-b)+b^2(b-a)=(a-b)^2(a+b)\ge0$可得,当且仅当$a=b$时等号成立。

由$2a^3+b^3\ge3a^2b$,轮换$a,b,c$,有
\begin{align*}
  2a^3+b^3\ge3a^2b,\quad 2b^3+c^3\ge3b^2c,\quad 2c^3+a^3\ge3c^2a
\end{align*}
三式相加并除以3可得。

\begin{theorem}[Shapiro不等式,Shapiro's Cyclic Inequalities]
  $n$是正整数,序列$\{x_1,x_2,\cdots,x_n\}$是正数序列,则
  \begin{enumerate}
  \item 若$n$是正奇数且$3\le n\le 23$,则
    \begin{align*}
      \sum_{i=1}^{n} \frac{x_i}{x_{i+1} + x_{i+2}} \ge \frac{n}{2}  
    \end{align*}
    其中$x_{n+1} = x_1, x_{n+2} = x_2$。当且仅当$x_1=x_2=\cdots=x_n$时等号成立;

  \item 若$n$且正偶数且$4\le n\le 12$,则
    \begin{align*}
      \sum_{i=1}^{n} \frac{x_i}{x_{i+1} + x_{i+2}} \ge \frac{n}{2}  
    \end{align*}
    当且仅当$x_1=x_3=x_5=\cdots=x_{n-1}$且$x_2=x_4=x_6=\cdots=x_n$时等号成立;

  \item 若$n$是大于12的偶数或者是大于23的奇数,则存在正数序列$\{x_1,x_2,\cdots,x_n\}$,使得
    \begin{align*}
      \sum_{i=1}^{n} \frac{x_i}{x_{i+1} + x_{i+2}} < \frac{n}{2}  
    \end{align*}
  \end{enumerate}
\end{theorem}
\begin{proof}[说明]
  这里仅说明一下其证明历史。B.~A.~Troesch在1989年证明了(1),P.~J.~Bushell \& J.~B.~McLeod在2002年证明了(2),而(3)则是早在1979年就被 J.~L.~Searcy \& B.~A.~Troesch所证明。
\end{proof}

\begin{example}[Nesbitt不等式]
  对任意正数$a,b,c$,有
  \begin{align*}
    \frac{a}{b+c}+\frac{b}{c+a}+\frac{c}{a+b}\ge\frac32
  \end{align*}
\end{example}
\begin{proof}[提示]
  这是Shapiro不等式在$n=3$时的情形。有多种巧妙的证明方法。下面是其中三种。
  \begin{enumerate}
  \item 利用凸函数的Jensen不等式。记$S=a+b+c$,则
    \begin{align*}
      f(x)=\frac{x}{S-x}
    \end{align*}
    在$x\in[0,S)$上是凸的,应用Jensen不等式,则有
    \begin{align*}
      \frac{f(a) + f(b) + f(c)}{3}\ge f\left(\frac{a+b+c}{3}\right)=f\left(\frac{S}{3}\right)=\frac12
    \end{align*}
  \item 用换元法,消除难处理的分母。令$x=a+b,y=b+c,z=c+a$,则$x,y,z$是正数,且有
    \begin{align*}
      a=\frac{x-y+z}2,\quad b=\frac{x+y-z}2,\quad c=\frac{-x+y+z}2
    \end{align*}
    代入,有
    \begin{align*}
      \frac{a}{b+c}+\frac{b}{c+a}+\frac{c}{a+b}
      &= \frac{x-y+z}{2y} + \frac{x+y-z}{2z} + \frac{-x+y+z}{2x}\\
      &= \frac12\left(\frac{x+z}{y} + \frac{x+y}{z} + \frac{y+z}{x}\right)-\frac32\\
      &= \frac12\left( \underbrace{\left(\frac xy + \frac yx\right)}_{\text{应用AM--GM不等式}}
        +\left(\frac zy + \frac yz\right)
        +\left(\frac xz + \frac zx\right)
        \right)-\frac32\\
      &\ge \frac12\times ( 2 + 2 + 2) - \frac32=\frac32
      % &=\frac32
    \end{align*}
    当且仅当$\dfrac xy=\dfrac yx$, $\dfrac zy=\dfrac yz$, $\dfrac xz=\dfrac zx$即$x=y=z$时等号成立,亦即$a=b=c$时等号成立。

  \item 直接利用AM--HM不等式。由
    \begin{align*}
      & \frac{(a+b)+(b+c)+(c+a)}{3}\ge \frac{3}{\dfrac1{a+b}+\dfrac1{b+c}+\dfrac1{c+a}}\\
      \iff & \big[(a+b)+(b+c)+(c+a)\big]\left(\dfrac1{a+b}+\dfrac1{b+c}+\dfrac1{c+a}\right)\ge 9\\
      \iff & 2(a+b+c)\left(\dfrac1{a+b}+\dfrac1{b+c}+\dfrac1{c+a}\right)\ge 9\\
      \iff & 2\left(1+\dfrac{c}{a+b}+1+\dfrac{a}{b+c}+1+\dfrac{b}{c+a}\right)\ge 9
    \end{align*}
    展开可得。其等号成立的充要条件是$a+b=b+c=c+a$,即$a=b=c$。$\qedhere$
  \end{enumerate}
\end{proof}

\begin{question}
  证明Shapiro不等式在$n=4$时的情况,即对任意正数$a,b,c,d$,有
  \begin{align*}
    \frac{a}{b+c}+\frac{b}{c+d}+\frac{c}{d+a}+\frac{d}{a+b}\ge 2
  \end{align*}
\end{question}
% \begin{proof}[提示]
%   应用\ref{lemma:titu}的T2引理,有
%   \begin{align*}
%          &\frac{a}{b+c}+\frac{b}{c+d}+\frac{c}{d+a}+\frac{d}{a+b}\\
%     =\   &\frac{a^2}{a(b+c)}+\frac{b^2}{b(c+d)}+\frac{c^2}{c(d+a)}+\frac{d^2}{d(a+b)}\\
%     \ge\ &\dfrac{(a+b+c+d)^2}{a(b+c) + b(c+d) + c(d+a) + d(a+b)}
%   \end{align*}

%   \begin{align*}
%        & \frac{a}{b+c}+\frac{b}{c+d}+\frac{c}{d+a}+\frac{d}{a+b}\\
%     =\ & \left(\frac{a}{b+c}+\frac{c}{d+a}\right) + \left(\frac{b}{c+d}+\frac{d}{a+b}\right)\\
%     \ge\ & \dfrac{(a+c)^2}{a(b+c)+c(d+a)} + \dfrac{(b+d)^2}{b(c+d)+d(a+b)}\\
%     =\ & \dfrac{a^2+2ac+c^2}{ab+ac+cd+da}
%   \end{align*}
% \end{proof}

% 同样应用换元法简化分母,令$w=a+b$,$x=b+c$,$y=c+d$,$z=d+a$。先反求用$w$,$x$,$y$和$z$表示$a$:
% \begin{enumerate}
% \item 先看$w$,比$a$多了个$b$;
% \item $x$有$b$,$w-x=a-c$,又多减了个$c$;
% \item $y$里有$c$,$w-x+y=a+d$,则又多加了个$d$;
% \item $z$里有$d$,$w-x+y-z=2a$,多加了个$a$。
% \end{enumerate}
% 从而有$a=(w-x+y-z)/2$。类似地,有
% \begin{align*}
%   a&=\frac{w-x+y-z}2 & b&=\frac{x-y+z-w}2\\
%   c&=\frac{y-z+w-x}2 & d&=\frac{z-w+x-y}2
% \end{align*}
% 代入,有
% \begin{align*}
%      &\frac{a}{b+c}+\frac{b}{c+d}+\frac{c}{d+a}+\frac{d}{a+b}\\
%   =\ &\frac{w-x+y-z}{2x} + \frac{x-y+z-w}{2y} + \frac{y-z+w-x}{2z} + \frac{z-w+x-y}{2w}\\
%   =\ &\frac12\left( \frac{w+y-z}{x} + \frac{x+z-w}{y} + \frac{y+w-x}{z} + \frac{z+x-y}{w} - 4\right)
% \end{align*}
% 有负号,不能直接应用AM--GM不等式。代入继续化简,要证明的不等式等价于
% \begin{align*}
%        & \frac12\left( \frac{w+y-z}{x} + \frac{x+z-w}{y} + \frac{y+w-x}{z} + \frac{z+x-y}{w} - 4\right) \ge 2\\
%   \iff & \frac{w+y-z}{x} + \frac{x+z-w}{y} + \frac{y+w-x}{z} + \frac{z+x-y}{w} \ge 8\\
%   \iff & wyz(w+y-z) + xzw(x+z-w) + ywx(y+w-x) + zxy(z+x-y) \ge 8wxyz\\
%   \iff & w^2(yz-xz+xy) + x^2(wz-wy+yz) + y^2(
% \end{align*}


% {\color{red}似乎没用?}


\begin{question}
  证明:对任意正数$a,b,c$,有
  \begin{align*}
    \frac{a^3}{a^2+ab+b^2}+\frac{b^3}{b^2+bc+c^2}+\frac{c^3}{c^2+ca+a^2}
    \ge
    \frac{a+b+c}{3}
  \end{align*}
\end{question}

\begin{question}
  对任意正数$a_1,a_2,\cdots,a_n,b_1,b_2,\cdots,b_n$,有
  \begin{align*}
    \sum_{i=1}^n\frac{a_ib_i}{a_i+b_i}\le
    \frac{\sum\limits_{i=1}^n a_i \cdot \sum\limits_{i=1}^n b_i}
         {\sum\limits_{i=1}^n a_i + \sum\limits_{i=1}^n b_i}
  \end{align*}
\end{question}
对$n$用数学归纳法。

\begin{question}
  若$a_1,a_2,\cdots,a_n,b_1,b_2,\cdots,b_n$是正数,则
  \begin{align*}
    \sum_{i=1}^n \sqrt{a_ib_i}
    \le
    \sqrt{\sum_{i=1}^n a_i \cdot \sum_{i=1}^n b_i}
  \end{align*}
\end{question}
对$n$用数学归纳法。

\begin{question}
  对正数$a,b,c$,证明
  \begin{align*}
    \frac{5a^3-ab^2}{a+b} & \ge 3a^2-b^2\\
    \frac{5a^3-ab^2}{a+b} + \frac{5b^3-bc^2}{b+c} + \frac{5c^3-ca^2}{c+a} & \ge 2(a^2+b^2+c^2)
  \end{align*}
\end{question}
只需证明第一条即可,而第一条又等价于$2a^3+b^3\ge 3a^2b$。

\begin{question}
  对大于$2$的整数$n$,及非负数$x_1,x_2,\cdots,x_n$,若$x_1=0, x_n=1$,则存在整数$j\in[1,2,\cdots,n-1]$,使得
  \begin{align*}
    \left| x_{j-1} - 2x_j + x_{j+1} \right| \ge \frac{4}{n^2}
  \end{align*}
\end{question}
令$x_0=0,x_{n+1}=1$,且对$k=0,1,2,\cdots,n$,令$y_k=x_{k+1}-x_k$,则$y_0=y_n=0$,且$y_0+y_1+y_2+\cdots+y_n=1$。用反证法。假设对所有的$j=1,2,\cdots,n$,不等式不成立,从而有
\begin{align*}
  y_j - y_{j-1} \le \left| y_j-y_{j-1} \right| = \left| x_{j-1} - 2x_j + x_{j+1} \right| < \frac{4}{n^2}
\end{align*}
对$j=1,2,\cdots,k$加起来,则有
\begin{align*}
  y_k&=y_k-y_0=(y_1-y_0) + (y_2-y_1) + \cdots + (y_k-y_{k-1})\\
  &<\underbrace{\frac{4}{n^2}+\frac{4}{n^2}+\cdots+\frac{4}{n^2}}_{k\text{个}}\\
  &=\frac{4k}{n^2}
\end{align*}
同样$y_{j-1}-y_j<\frac{4}{n^2}$,加起来,有
\begin{align*}
  y_k&=y_{k}-y_n=(y_k-y_{k+1})+(y_{k+1}-y_{k+2})+\cdots+(y_{n-1}-y_{n})\\
  &<\underbrace{ \frac{4}{n^2}+\frac{4}{n^2}+\cdots+\frac{4}{n^2}}_{n-k\text{个}}\\
  &=\frac{4(n-k)}{n^2}
\end{align*}

若$n$是奇数,则
\begin{align*}
  1&=y_0+y_1+y_2+\cdots+y_n\\
  &=\left(y_1+y_1+\cdots+y_{\frac{n-1}2}\right)
  + \left(y_{\frac{n+1}2}+y_{\frac{n+1}2+1}+\cdots+y_n\right)\\
  & <\frac4{n^2}\left(
  \left(1+2+\cdots+\frac{n-1}2\right) +
  \left( \left(n-\frac{n+1}2\right) + \left(n-\frac{n+1}2-1\right) + \cdots + 1\right)
  \right)\\
  &=2\cdot\frac4{n^2}\left(1+2+\cdots+\frac{n-1}2\right)\\
  &=\frac8{n^2}\left( \left(1+\frac{n-1}2\right)\cdot\frac{n-1}2\cdot\frac12 \right)\\
  &=\frac{n^2-1}{n^2}<1
\end{align*}
矛盾。

若$n$是偶数,则
\begin{align*}
  1&=y_0+y_1+y_2+\cdots+y_n\\
  &=\left(y_1+y_1+\cdots+y_{\frac{n}2-1}\right)
  + \left(y_{\frac{n}2+1}+y_{\frac{n}2+2}+\cdots+y_n\right)
\end{align*}
剩余略。

