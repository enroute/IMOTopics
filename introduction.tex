
\chapter{绪论}
\label{chap:introduction}

此书的本意,是增广见闻,拓宽思维,不是计算练习。遇到复杂的计算,除非有特别说明,不建议用纸笔手工验证,纯属浪费时间。若有需要,请使用计算器或自行编写计算机程序解决。

千万不要背!此非编者意图,理解为上,背诵为下。能让人记住的通常都是方法,而不是结论。

\section{关于版权}
\label{sec:copyright}

此书内容,除部分来自编写者的智力成果外,多来源于网络。若其中有侵犯了您的版权的地方,请务必联系\verb|mario.li@foxmail.com|。

\section{关于符号}
\label{sec:symbol}

字体$\vec{v}$通常用于表示向量,$\vec{AB}$通常用于表示由点$A$到点$B$的向量,而$AB$通常用于表示点$A$到点$B$之间线段的长度。

大部分情况下,$i$用于表示虚数单位。

\section{关于证明}
\label{sec:about-proof}

本书中的证明大多是不严格的,只是给人提供一种思路。若需要严格的证明,请自选补充完整,或者参考各种专业书籍。

\section{关于抄小道}

这里选择的一些美妙数学内容及技巧,只是给读者们增广见闻,千万不要在学校系统学习数学知识的时候过份依赖这里的可能属于超纲的内容,否则很有可能得不偿失。下面转一篇Heshawn写的文章\footnote{原文地址:https://www.zhihu.com/question/296821910}。

\begin{example}
  有没有在高考数学中使用洛必达法则而不扣分的方法?
\end{example}
\noindent{\bfseries{Heshawn的回复}}.

  做事爱「走小路」尽管有时看起来是抄了近道,但常走小道有两个坏处:你丧失了学习主流基本功的机会,底子不扎实难成大事;其次,小路沟多、容易翻车。

  \begin{quotation}
    我是全国一卷考生,我想问一个判卷方面的问题:那种像「洛必达法则」和「泰勒公式」这样的高等数学的结论在大题上能直接用吗?因为平时练习时有导数那部分很多题,感觉用洛必达法则很方便呀,老师对这部分学习有什么建议吗?
  \end{quotation}
  当时我写了很详细的回答。恰巧今天看到了这个问题,感觉你也一定需要我对他咨询的回复,所以我把这篇6,200字的回复全文分享给你:

  你提了一个非常直击要害的问题,但回答起来我要非常小心,因为稍不留意就会触犯阅卷的保密协议。不过好在高考的阅卷细则保密期限只有5年,而你说的这种题目在高考试卷上几乎每年都有,所以我不妨给你举一个已经超出保密期限很多年的旧例子。

\noindent{\bfseries \underline{0. 一道超出保密期限的真题}}
  
你是全国卷的考生,那你不妨翻一翻2010年全国课标卷理科的那道导数题目,它长成这个样子:
\begin{example}[2010新课标卷](本小题满分12分)\par
  \noindent 设函数$f(x)=e^x-1-x-ax^2$。
  \begin{enumerate}
  \item 若$a=0$,求$f(x)$的单调区间;
  \item 若当$x\ge 0$时$f(x)\ge0$,求$a$的取值范围。
  \end{enumerate}
\end{example}

这道题的第一问严格来说跟第二问没有直接的联系\footnote{即解决第一问无需用到第二问的结论,从而先解决第一问后,在解决第二问时可采用第一问的结论。},所以我直接把它第二问的「常规解法」放在下面:

\fbox{\begin{minipage}{.9\textwidth}
\begin{proof}[第2问的解法]
  $f'(x)=e^x-1-2ax$。由(1)知,$e^x\ge 1 + x$,当且仅当$x=0$时等号成立,故
  \begin{align*}
    f'(x)\ge x-2ax=(1-2a)x
  \end{align*}
  从而当$1-2a\ge0$,即$a\le\frac12$时,$f'(x)\ge0(x\ge0)$。而$f(0)=0$,于是$x\ge0$时,$f(x)\ge0$。由$e^x>1+x(x\ne0)$可得
  \begin{align*}
    e^{-x}>1-x(x\ne0)
  \end{align*}
  从而当$a>\frac12$时,有
  \begin{align*}
    f'(x)<e^x-1+2a(e^{-x}-1)=e^{-x}\left(e^x-1\right)\left(e^x-2a\right)
  \end{align*}
  故当$x\in(0,\ln2a)$时,$f'(x)<0$,而$f(0)=0$,于是当$x\in(0,\ln2a)$时,$f(x)<0$。综上得$a$的取值范围为$(-\infty,\frac12]$。
\end{proof}
\end{minipage}
}

其实这道题就是非常典型的能直接分离参量的类型,比如你可以把$f(x)=e^x-1-x-ax^2\ge0$这个式子的参量$a$给分离出来,得到的是下面这个式子:
\begin{align*}
  a\le\frac{e^x-1-x}{x^2}
\end{align*}

你看,分母除的还是$x^2$,都不用考虑变不变号,多方便,然后当$x$趋向于0时,用洛必达法则直接可以得到一个$a\leq\frac{1}{2}$,这不就是有些老师嘴里所谓的「秒杀算法」吗?

但是你且慢,让我先抛开这个题目,给你掰开揉碎讲一点更高级的东西。

\noindent{\bfseries \underline{1. 总抄小道的人会错过主流信息}}

罗琳阿姨当年写《哈利波特与火焰杯》时,在正邪两派势力即将进行对决的前夜,魔法学院的校长邓布利多对哈利说过一句话:

\begin{quotation}
  “到时候每个人都要做出一个选择:我们应该做正确的事?还是简单的事?”
\end{quotation}

我第一次读到这句话时曾经因为它的逻辑错位而迷惑,因为一般而言我们以为「正确」的反面应该是「错误」,可邓布利多校长的话无疑是向我们暗示「正确」的反面是「简单」,这和我们的常识相违背。

那是因为当时我还太小。实际上当你长大之后,会多多少少发现一点规律:每次你做事试图偷工减料时,往往会出错;即使暂时看上去安然无恙、你也会为以后埋下隐患。

我进入大学一年级时刚刚开始学《数学分析》这门课,上来第一章讲的就是极限理论,定义非常的细,有一种特别标准的规范,叫做「$\varepsilon- \delta$语言」——每一节课后都有特别多的题目,让你根据极限的定义计算一大堆看上去特别简单的题目,比如$\frac{x^2}{sinx}$在$x$趋向于0的时候等于多少。

我们当时还没讲到洛必达法则,但是我写作业时就直接用,结果当时教我这门课的教授把我叫到了办公室跟我说:你看书很快,预习做的也不做,这题目我也不能算你错,但是这节课后的题目主要目的是想让你练习使用「 $\varepsilon- \delta$语言」这种规范定义来理解极限的本质,你要是用洛必达法则把题目都做了,在这儿不把极限语言理解清楚,以后你学级数、反常积分时就会吃力,在数学上,欠下的债以后都是要还的。

——这个教授告诉我的话是这么一条道理:很多时候我们明知有大路却喜欢抄小道,自以为走了近路快人一步,其实你不知道走小路还有潜在的风险:比如你会错过学习主流经验、夯实自己基本功的机会,这样一次两次看不出什么,但长此以往你的底子不行,就难成大器。

我们就说回最开始的那道题,你可能都把答案忘了,我再给你贴一遍:
\begin{quote}
  这里不再帖了,请翻回去看吧。
\end{quote}

——你仔细观察,这个解法其实没有你想象的那么麻烦,它的核心有3个:
\begin{enumerate}
\item 你要看出来整个函数求导后是一部分是指数函数、一部分是一次函数,这两者之间的图像走势、以及两者图像之间的位置关系你要清楚;

\item 然后你要了解导函数的正负和原函数单调性之间的关系,进而:

\item 在最后通过分离变量法,根据「导函数是否小于0」确定关于参量a的分类讨论标准。
\end{enumerate}

我在之前的文章\footnote{https://www.zhihu.com/question/36099747/answer/547283165}中特别提到过:

题目不重要,但答案中每一步是「怎么想到的」的你要弄清楚,因为下次考试你很难碰到原题,可是再多的题目也不过是有限的步骤和知识点进行排列组合的表象。

而针对这道题,我为你分析了3个最主要的知识点——它们有些是你早就学过的:比如第1点是「简单函数图像的走势与不同函数图像之间的对比」,这是你高一就要讲的东西;还有一些是特别重要的思考原则:比如第3点,我们都知道要分类讨论,但这个类应该怎么分,从哪儿下手?对于这道题目,你要考虑到的分割是导函数和0的关系,进而就是$a$和$\frac12$之间的关系。

这些都是非常关键的知识点,而你如果在平时训练时反手就是一个洛必达法则,好像看起来很精明,而事实上你丧失了练习以前的知识点和学习「参量分类标准确立原则」的机会,长久以往、你就会有知识盲点,其实得不偿失。

所以,作为一个数学老师,我仅仅从你学知识的这个层面来讲,也不建议你用洛必达法则,因为抄近道是有代价的,首先的代价就是你会错失学习主流知识点和夯实基础能力的机会。

当然、其次还有第二点。

\noindent{\bfseries \underline{2. 永远要避免灾难性失败}}

当时我大学时用洛必达法则绕过极限定义直接做题,教授把我叫到办公室说「这道题目我不能算你错」,那是因为大学的教育气氛没那么严格。

可是就这道2010年的题目来说,如果你在高考试卷上用洛必达法则算这道题,即便最终得到了和标准答案一样的结果,在当年你只能得2分——因为2010年已经过去快要9年了,所以这道题的评分细则是可以说的,我不知道当年参加考试的学生下来之后有没有人感觉自己数学的实际分数比估分少了那么五六分——大概就是这道题的缘故。

当然这还不是我要说的重点,我想跟你说的是另外一个你不仔细思考就难以体察的东西:你在考场上有没有想过,这道题自己可能做不对呢?如果你用超纲的方法做题,结果还玩儿砸了,你猜猜最后会是什么结果?

这就还需要讲另外一个阅卷场上的真事儿了。

你一定知道理科数学卷背后有一道立体几何的大题,第二问一般需要建立空间直角坐标系计算一个平面的法向量——常规做法是:假设这个平面的法向量坐标为$(x,y,z)$,然后根据垂直向量内积为0联立两个方程,求一个未定式方程组,比如一道典型的题目长这样:

{\color{red} https://pic3.zhimg.com/80/v2-21bc3721dd27d16046282fc53753aaf1\_hd.jpg}

但是如果有人跟你讲的话,你可能会听说:平面的法向量是可以用「向量外积」来求的,实际上我在知乎上还看到有老师专门写这种方法,这里我就不点名了。

请注意,我下面要讲一个【{\bfseries 翻车的故事}】:

我在阅卷场上就碰到过这种情况:它用外积算平面的法向量,结果运算出错了——后面所有的步骤都是规范的,但因为它之前的数据错了,所以就成了车祸现场。

但很抱歉,这件事儿距离现在太近,所以这道题最后的判分结果我不能讲,但我要提醒你的是:遇到这种大题,你老老实实用常规方法做,即便从一开始数据就算错了,但是每一步老师都能给你步骤分,你要是能算到最后,至少还能保住一多半的卷面分;可假如你一开始就用超纲方法算,一旦失手,后果基本上就是灾难性的。

这就是我想说的第二点:你想使用超常规的做法,就得承担超常规做法失败后自己输的一干二净的风险——从这个角度上讲,这个世界上很多人循规蹈矩,坚决不走小路,不抄近道,那不是因为他们傻、更不是因为他们怂,而是因为他们比你更清楚地意识到:抄小路有一个潜在的风险,就是路窄了车容易翻到沟里。

换用如今比较时兴的一个说法:永远不要冒有可能毁灭你的风险。

我不知道零零后现在有没有到对商业和金融感兴趣的时候,但也许你去问稍微比你大几岁的哥哥姐姐,他们一定知道有个靠炒股和金融投资成为世界首富的人,叫做巴菲特。我对金融没有特别的研究,所以他的投资理念是我无法评价的,但是巴菲特谈论自己的投资的方法时有几句话我们都值得听一听,其中一句就是:他说他自己从来不在投资时加杠杆,也不要做空股票,因为这两种行为会带来毁灭性的风险。

这里有两个术语,分别是「做空」和「加杠杆」,这两个词是什么意思我就不在这里写了,有兴趣的话你可以搜一搜,简单来说这是两种在资本市场上放大收益率以及风险的的放大器——他们可以让你在能赚钱的时候赚得更多,但同样当你亏欠的时候这两种方法能让你倾家荡产。

巴菲特说,你做投资时永远不要玩这种游戏,因为失败并不可怕,可怕的是你面临灾难性失败,失败的时候也要给自己留有缓冲的余地,毁灭了、就没有重生的机会了。

这也就是我今天要跟你说的第二句话:不要用超纲的方法答题,采用常规解法、错了还有步骤分;而你一旦用超纲解法,一旦出错,这道题尸骨无存。

\noindent{\bfseries \underline{3. 总结}}

最后总结一下我想说的几个要点:

第一,平时练习时总想抄近路,会让你失去练习基本知识与方法的机会,造成知识体系上的盲区,所以我不建议你平时使用洛必达法则;

第二、考场上使用任何超纲解法,有极大风险,因为你不能保证自己每一步的运算都对,常规方法算错了还有步骤分,而超纲解法稍有闪失、失分严重;

第三,即便你算对了,也只能拿到微不足道的一点结果分。

特别提醒的是,最后的第三点是我针对2010年那道题目讲的,这不算泄漏评分细则,警察叔叔就不要给我寄快递了。

希望这些内容能够对你有所帮助。

你一定还想看下面这篇文章:
\begin{quote}
  有哪些高中教材不要求但高考解题时非常好用的知识?\footnote{https://www.zhihu.com/question/36099747/answer/547283165}
\end{quote}

