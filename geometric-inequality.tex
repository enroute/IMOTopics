
\section{几何不等式}
\label{sec:geometric-inequality}

\subsection{基本不等式}
\label{sec:basic-geometric-inequality}

\begin{theorem}
  平面上两点间直线段长度最短。
\end{theorem}

\begin{example}
  平面上凸四边形$ABCD$,$M,N$分别是$AD$和$BC$的中点,则
  \begin{align*}
    MN\le\frac{AB+CD}{2}
  \end{align*}
  当且仅当$AB\parallel CD$时等号成立。

  \centering
  \begin{tikzpicture}[scale=1.0,line join=round]
    \begin{scope}[shift={(0,0)}]
      \coordinate[label=below left:$A$]  (A) at (0,0);
      \coordinate[label=below right:$B$] (B) at (5,0);
      \coordinate[label=above right:$C$] (C) at (4,4);
      \coordinate[label=above left:$D$]  (D) at (2,3);
      \coordinate[label=left:$M$]        (M) at ($.5*(A)+.5*(D)$);
      \coordinate[label=right:$N$]       (N) at ($.5*(B)+.5*(C)$);
      % \coordinate[label=below:$P$] (P) at ($.5*(D)+.5*(B)$);
      \draw(A)--(B)--(C)--(D)--cycle;
      \draw(M)--(N);%--(P)--cycle;
      %\draw(D)--(B);
      \tkzDrawPoints(A,B,C,D,M,N);
    \end{scope}
    \begin{scope}[shift={(7,0)}]
      \coordinate[label=below left:$A$]  (A) at (0,0);
      \coordinate[label=below right:$B$] (B) at (5,0);
      \coordinate[label=above right:$C$] (C) at (4,4);
      \coordinate[label=above left:$D$]  (D) at (2,3);
      \coordinate[label=left:$M$]        (M) at ($.5*(A)+.5*(D)$);
      \coordinate[label=right:$N$]       (N) at ($.5*(B)+.5*(C)$);
      \coordinate[label=below right:$P$] (P) at ($.5*(D)+.5*(B)$);
      \draw(A)--(B)--(C)--(D)--cycle;
      \draw(M)--(N);
      \draw[help lines, dashed](M)--(P)--(N);
      \draw[help lines, dashed](D)--(B);
      \tkzDrawPoints(A,B,C,D,M,N,P);
    \end{scope}
  \end{tikzpicture}
\end{example}
\begin{proof}
  如图取$BD$的点$P$并作辅助线,则
  \begin{align*}
    MN\le MP+NP=\frac{AB+CD}{2}
  \end{align*}
  当且仅当点$P$在$MN$上时等号成立,而$P$在$MN$上时等价于$AB\parallel CD$。
\end{proof}

\begin{theorem}[Pythagorean不等式]
  记三角形的三边分别为$a\le b\le c$,则
  \begin{enumerate}
  \item 三角形是直角三角形$\iff a^2+b^2=c^2$;
  \item 三角形是锐角三角形$\iff a^2+b^2>c^2$;
  \item 三角形是钝角三角形$\iff a^2+b^2<c^2$。
  \end{enumerate}
\end{theorem}
\begin{proof}
  由余弦定理$c^2=a^2+b^2-2ab\cos\mathrm{C}$可得。
\end{proof}


\begin{theorem}[等周不等式,Isoperimetric Inequality]
  若一个平面图形的面积与周长分别为$A$和$P$,则$4\pi A\le P^2$。也就是说在平面上用长度为$P$的线段能围成的最大面积是半径为$\dfrac{P}{2\pi}$的圆,其面积为$\dfrac{P^2}{4\pi}$。
\end{theorem}
\begin{proof}[提示]
  该定理的证明并不平凡,
\end{proof}

\begin{theorem}[三角不等式,Trigonometric Inequality]
  记三角形的三个角分别为$A,B,C$,则
  \begin{align*}
    \sin A +\sin B + \sin C&\le\frac{3\sqrt3}{2}\\
    \cos A +\cos B + \cos C&\le\phantom{3}\,\frac{3}{2}
  \end{align*}
\end{theorem}
\begin{proof}
  由$\sin(x)$及$\cos(x)$在$[0,\pi]$上是凹函数,利用Jensen不等式可得。
\end{proof}

\begin{theorem}[相交弦定理,Intersecting Chords Theorem]
  圆内的两条相交弦,被交点分成的两条线段长的积相等。
\end{theorem}
\begin{proof}
  利用相似三角形可得。
\end{proof}

\begin{theorem}[海伦公式,海伦-秦九韶公式,Heron's Formula]
  记三角形的三边边长分别为$a,b,c$,其半周长$s=\dfrac{a+b+c}2$,则三角形的面积为
  \begin{align}
    S=\sqrt{s(s-a)(s-b)(s-c)}
  \end{align}
\end{theorem}
\begin{proof}
  用余弦公式可证。或者用勾股定理证明$c$边对应的高
  \begin{align*}
    h=\frac{4s(s-a)(s-b)(s-c)}{c^2} &\qedhere
  \end{align*}
\end{proof}

\begin{theorem}
  如图~\ref{fig:r-of-incircle}所示,三角形内切圆将各边分别分割成$x,y,z$的长度,则可得到内切圆的半径公式
  \begin{align}
    r=\sqrt{\frac{xyz}{x+y+z}}=\frac{S}{s}
  \end{align}
\end{theorem}
\begin{figure}[htbp]
  \centering
  \begin{tikzpicture}[scale=1]
    \tkzDefPoint[label=below left:$A$](0,0){A}
    \tkzDefPoint[label=below right:$B$](6,0){B}
    \tkzDefPoint[label=above:$C$](5,5){C}
    
    \tkzDefCircle[in](A,B,C)\tkzGetPoint{I}\tkzGetLength{rIN}
    \tkzDrawCircle[R](I,\rIN pt);
    \tkzDrawSegments(A,B B,C C,A)
    % \tkzDrawSegments[dashed](I,A I,B I,C)

    \coordinate(IA) at ($(B)!(I)!(C)$);
    \coordinate(IB) at ($(C)!(I)!(A)$);
    \coordinate(IC) at ($(A)!(I)!(B)$);

    %\tkzDrawSegments[dashed](I,IA I,IB I,IC)
    \tkzMarkRightAngle[color=blue](B,IC,I)
    \tkzMarkRightAngle[color=blue](C,IA,I)
    \tkzMarkRightAngle[color=blue](A,IB,I)

    \foreach \p in{A,B,C,I,IA,IB,IC}{
      \tkzDrawPoint(\p)
    }
    \tkzLabelPoints[below left](I)

    \draw(A)--(IC) node[below,sloped,midway]{$x$};
    \draw(A)--(IB) node[above left,sloped,midway]{$x$};
    \draw(B)--(IC) node[below,sloped,midway]{$y$};
    \draw(B)--(IA) node[above right,sloped,midway]{$y$};
    \draw(C)--(IB) node[above left,sloped,midway]{$z$};
    \draw(C)--(IA) node[above right,sloped,midway]{$z$};
    \draw[dashed](I)--(IC);
    \draw[dashed](I)--(IA) node[above,sloped,midway]{$r$};
    \draw[dashed](I)--(IB);% node[below left,sloped,midway]{$r$};
  \end{tikzpicture}
  \caption{三角形内切圆的半径}
  \label{fig:r-of-incircle}
\end{figure}
\begin{proof}
  令$s=\dfrac{a+b+c}2$是半周长,则显然有$x=s-a, y=s-b, z=s-c$。由海伦公式可得
  \begin{align*}
    S=\sqrt{xyz(x+y+z)}
  \end{align*}
  另一方面,$S=\dfrac12r(a+b+c)=r(x+y+z)$,从而可得。
\end{proof}

\begin{theorem}[Euler定理]
  记$R,r$分别是三角形的外接圆和内切圆半径,$d$是外接圆与内切圆圆心间的距离,则$d^2=R(R-2r)$,或等价的
  \begin{align*}
    \frac1{R-d}+\frac1{R+d}=\frac1r
  \end{align*}
\end{theorem}
\begin{proof}
  记$G,I$分别是外接圆和内切圆的圆心,在过$G$和$I$的直线上找到长度分别为$R+d$和$R-d$的线段,比如直线$GI$与外接圆的交点$P$、$Q$,则线段$IP$和$IQ$分别为$R\pm d$。

  如图\ref{fig:in/circumcircle},延长$CI$得$L$,延长$LG$得$M$,延长$GI$得$P$和$Q$,则$\triangle CID\sim\triangle MLA$,且$LA=LI$,从而
  \begin{align*}
    (R+d)(R-d)&=IP\cdot IQ=IC\cdot IL=IC\cdot LA=ID\cdot LM=2Rr&&\qedhere
  \end{align*}
\end{proof}

\begin{figure}[htbp]
  \centering
%\fbox{%Use \fbox to check the boundary of the tikz picture
  \begin{tikzpicture}[scale=1]
    \tkzDefPoint[label=below left:$A$](0,0){A}
    \tkzDefPoint[label=below right:$B$](6,0){B}
    \tkzDefPoint[label=above:$C$](5,5){C}
    
    \tkzDefCircle[in](A,B,C)\tkzGetPoint{I}\tkzGetLength{rIN}
    \tkzDrawCircle[R](I,\rIN pt);
    \tkzDrawSegments(A,B B,C C,A)
    % \tkzDrawSegments[dashed](I,A I,B I,C)

    % \coordinate(IA) at ($(B)!(I)!(C)$);
    \coordinate[label=above left:$D$](IB) at ($(C)!(I)!(A)$);
    % \coordinate(IC) at ($(A)!(I)!(B)$);

    % \tkzDrawSegments[dashed](I,IA I,IB I,IC)
    \tkzDrawSegments[dashed](I,IB)
    % \tkzMarkRightAngle[color=blue](B,IC,I)
    % \tkzMarkRightAngle[color=blue](C,IA,I)
    \tkzMarkRightAngle[color=blue](A,IB,I)

    % circumcircle
    \tkzCircumCenter(A,B,C)\tkzGetPoint{G}
    \tkzDrawCircle(G,A)
    \tkzInterLC(G,I)(G,A)\tkzGetPoints{P}{Q}

    \draw[dashed](A)--(I);
    \draw[dashed](I)--(P);
    \draw[dashed](I)--(Q);
    \draw[thick](G)--(I);

    \tkzInterLC(C,I)(G,A)\tkzGetPoints{}{L}
    \tkzInterLC(L,G)(G,A)\tkzGetPoints{}{M}

    \tkzLabelPoints[left](P)
    \tkzLabelPoints[right](Q,I)
    \tkzLabelPoints[below](L)
    \tkzLabelPoints[below left](G)
    \tkzLabelPoints[above](M)

    \draw[dashed](C)--(L)--(M)--(A)--(L);

    \foreach \p in{A,B,C,I,G,P,Q,IB,L,M}{
      \tkzDrawPoint(\p)
    }

    % emphasize similar triangles
    \draw[very thick](C)--(I)--(IB)--cycle;
    \draw[very thick](M)--(A)--(L)--cycle;
  \end{tikzpicture}
%}
  % Maybe due to BUG of pgfplots, there're wide space between figure
  % and caption, -3cm is measured by eye
  \vspace*{-3cm}
  \caption{三角形的外接圆与内切圆}
  \label{fig:in/circumcircle}
\end{figure}


\begin{theorem}[Euler不等式]
  记$R,r$分别是三角形$\triangle ABC$的外接圆和内切圆半径,则有$R\ge2r$。
\end{theorem}
\begin{proof}
  由Euler定理立得。
\end{proof}

\begin{theorem}\label{th:R-of-circumcircle}
  任意$\triangle ABC$,其三边边长分别为$a,b,c$,面积为$S$,则其外接圆半径为
  \begin{align}
    R=\frac{abc}{4S}
  \end{align}
\end{theorem}
\begin{figure}[htbp]
  \centering
  \begin{tikzpicture}[scale=1]
    \tkzDefPoint[label=below left:$A$](0,0){A}
    \tkzDefPoint[label=below right:$B$](6,0){B}
    \tkzDefPoint[label=above:$C$](5,5){C}
    
    % circumcircle
    \tkzCircumCenter(A,B,C)\tkzGetPoint{G}
    \tkzDrawCircle(G,A)
    \tkzLabelPoints[below left](G)

    \coordinate[label=above right:$D$](D) at ($(B)!(G)!(C)$);

    \draw[dashed](B)--(G) node[midway,below left]{$R$};
    \draw[dashed](C)--(G);
    \draw[dashed](D)--(G);
    \tkzMarkRightAngle[color=blue](G,D,B)
    \draw(A)--(B)--(C)
        node[pos=0.20,above right]{$\dfrac{a}2$} 
        node[pos=0.65,above right]{$\dfrac{a}2$} --cycle;

    \draw pic["$\alpha$",draw=orange,<->,angle eccentricity=1.6,angle radius=.6cm] {angle=B--A--C};
    \draw pic["$\alpha$",draw=orange,<->,angle eccentricity=1.6,angle radius=.6cm] {angle=B--G--D};
    \draw pic["$\alpha$",draw=orange,<->,angle eccentricity=1.6,angle radius=.6cm] {angle=D--G--C};

    \foreach \p in{A,B,C,D,G}{
      \tkzDrawPoint(\p)
    }
  \end{tikzpicture}    
  % Maybe due to BUG of pgfplots, there're wide space between figure
  % and caption, -3cm is measured by eye
  \vspace*{-3.5cm}
  \caption{三角形外接圆半径}
  \label{fig:R-of-circumcircle}
\end{figure}
\begin{proof}
  由图~\ref{fig:R-of-circumcircle}容易看出$\angle BAC = \angle CGD = \angle BGD$,从而有
  \begin{align*}
    S_{\triangle ABC} = \frac12 bc\cdot\sin\angle BAC
    = \frac12 bc \cdot \frac{a/2}{R} = \frac{abc}{4R}
  \end{align*}
  从而可算出$R$。
\end{proof}

\begin{theorem}
  $\triangle ABC$中三个顶点分别为$A,B,C$,则
  \begin{align*}
    \sin A \sin B \sin C \le{}& \left(\frac{\sin A + \sin B + \sin C}3\right)^3\\
                         \le{}& \sin^3\frac{A+B+C}3\\
                             =& \left(\frac{\sqrt3}{2}\right)^3 = \frac{3\sqrt3}{8}
  \end{align*}
  当且仅当$\triangle ABC$是正三角形时等号成立。
\end{theorem}
\begin{proof}
  三角形中任一顶点的正弦值是正的,从而由AM--GM不等式,有
  \begin{align*}
    \sqrt[3]{\sin A \sin B \sin C} \le \frac{\sin A + \sin B + \sin C}{3} 
  \end{align*}
  当且仅当$\sin A=\sin B=\sin C$时等号成立。而对于三个角的正弦值的和,考虑$\sin$在$(0,\pi)$上是凸的,利用Jensen不等式,有
  \begin{align*}
         && \frac{\sin A + \sin B + \sin C}{3} \le \sin \frac{A + B + C}{3} = \sin \frac{\pi}{3}
  \end{align*}
  当且仅当$A = B = C$时等号成立。代入,有
  \begin{align*}
    \sin A \sin B \sin C \le \left( \frac{\sin A + \sin B + \sin C}{3} \right)^3 \le \left( \frac{\sqrt3}{2} \right)^3
  \end{align*}
  当且仅当$A=B=C$即三角形是正三角形时等号成立。

  关于$\sin A + \sin B + \sin C \le 3 \sin \frac\pi3$,也可由下面的图直观的看出来。
  \begin{center}
    \begin{tikzpicture}[scale=1.0]
      \begin{axis}[anchor=origin,
        scale only axis,
        width=0.8\textwidth,
        height=5cm,
        xmin = -.5,
        xmax = 4,
        ymin = -.5,
        ymax = 1.5, 
        axis lines = middle,
        enlargelimits = true,
        xlabel = {$\mathbf{x}$},
        ylabel = {$\mathbf{y}$},
        yticklabels={,,},
        xticklabels={,,}
        ]
        % \coordinate (A) at (axis cs:40,6);
        \addplot[domain=0:3.1415926,smooth] (\x, {sin(deg(\x))});
        \coordinate (A) at (.2, 0.198669330795061);  % sin(.2) = 0.198669330795061
        \coordinate (B) at (1.0, 0.841470984807897); % sin(1 ) = 0.841470984807897
        \coordinate (C) at (3.1415926-1.2, 0.932039105385902); % sin(1.91415926) = 0.932039105385902
        % \tkzDrawPoints(A,B,C)
        % \draw[dashed] (-3.1415926,0)--(-3.1415926,3)node[pos=.8,left]{$x=-\pi$};
        \draw[help lines] (3.1415926,0)--(3.1415926,.1)node[pos=0,below]{$\pi$};
        \draw[help lines] (3.1415926/3,0)--(3.1415926/3,.1)node[pos=0,below]{$\frac\pi3$};
        \draw(A)--(B)--(C)--(A);
        \coordinate(G)at($1/3*(A) + 1/3*(B) + 1/3*(C)$);
        % \tkzDrawPoints(A,B,C);
        \draw(A)circle(1pt);
        \draw(B)circle(1pt);
        \draw(C)circle(1pt);
        \draw(G)circle(1pt);
      \end{axis}
    \end{tikzpicture}
  \end{center}
  在直角坐标系下作出$\sin$在$(0,\pi)$区间的函数图像,同时以曲线上的三点$(A, \sin A)$,$(B, \sin B)$,$(C, \sin C)$为顶点做三角形,则三角形在曲线下,其重心坐标为
  \begin{align*}
    \left( \frac{A+B+C}3, \frac{\sin A+\sin B+\sin C}3 \right) = \left( \frac\pi3, \frac{\sin A+\sin B+\sin C}3 \right)
  \end{align*}
  重心坐标在三角形内从而也在曲线之下,从而有
  \begin{align*}
    \frac{\sin A+\sin B+\sin C}3 \le \sin\frac\pi3
  \end{align*}
  上式左边是重心的$y$坐标,右边是曲线在$x=\frac\pi3$(也是重心的$x$坐标)上的$y$坐标。本质在于$\sin$是凸函数。
\end{proof}


\begin{theorem}
  单位圆的所有内接三角形中,正三角形的面积最大,周长最大。
\end{theorem}
\begin{proof}[提示]
  % 此命题中面积最大与周长最大是等价的。记$p\equiv a+b+c$是三角形的周长,根据
  % \begin{align*}
  %   S=\frac12r(a+b+c), \quad r = \frac{R^2 - d^2}{2R}
  % \end{align*}
  % 有
  % \begin{align*}
  %   S = \frac{(R^2-d^2)(a+b+c)}{4R} = \frac{(R^2-d^2)p}{4R}
  % \end{align*}
  % 其中$r$是三角形内接圆的半径,$d$是三角形外接圆与内接圆圆心的距离。
  
  % \begin{enumerate}
  % \item 若正三角形的面积最大,则由于正三角形中$d=0$
  %   \begin{align*}
  %     p = \frac{4RS}{R^2-d^2} = \frac{4S}{R}
  %   \end{align*}
  %   也最得最大值。
    
  % \item 若正三角形的周长最大,则正三角形的面积
  %   \begin{align*}
  %     S = \frac{(R^2-d^2)p}{4R} = \frac{Rp}{4}
  %   \end{align*}
  %   也取得最大值。
  % \end{enumerate}

  可以由下面的图看出圆内接三角形中正三角形面积最大的直观说明。先固定两个顶点$A,B$,第三个顶点$C$在圆上游动,要使$\triangle ABC$的面积最大则$C$必在$AB$的中垂线上。同样$A$必在$BC$的中垂线上,$B$在$CA$的中垂线上,从而面积最大的三角形必是正三角形。

  \begin{center}
    \begin{tikzpicture}[scale=1.0]
      \begin{scope}
        \coordinate(O) at (0,0); \coordinate[label=below left:$A$](A) at (210:2); \coordinate[label=below right:$B$](B) at (340:2);
        \coordinate[label=above:$C$](C) at (95:2); \coordinate[label=above:$C'$](C')at(130:2);
        \coordinate(D)at($.5*(A)+.5*(B)$);
        \tkzMarkRightAngle(C,D,A)
        \tkzDrawPoints(A,B,C,C',O)
        \draw(O)circle(2);
        \draw(A)--(B)--(C)--(A);
        \draw[help lines, dashed](C)--(D) (A)--(C')--(B);
      \end{scope}
      \begin{scope}[shift={(6,0)}]
        \coordinate(O) at (0,0); \coordinate[label=below left:$A$](A) at (210:2); \coordinate[label=below right:$B$](B) at (340:2);
        \coordinate[label=above:$C$](C) at (95:2); \coordinate(D)at(275:2);
        \tkzDrawPoints(A,B,C,D,O)
        \tkzMarkRightAngle(C,B,D)
        \draw pic["",draw=orange,angle eccentricity=1.6,angle radius=.6cm,fill=blue!20] {angle=B--A--C};
        \draw pic["",draw=orange,angle eccentricity=1.6,angle radius=.6cm,fill=blue!20] {angle=B--D--C};
        \draw(O)circle(2);
        \draw(A)--(B)--(C)node[midway,sloped,above]{$a$}--(A);
        \draw[help lines, dashed](C)--(D)--(B);
      \end{scope}
    \end{tikzpicture}
  \end{center}

  下面利用三角函数严格证明。由上面右图,容易知道$a = 2R\sin A$。记三角形的面积为$S$,周长为$p$,则
  \begin{align*}
    2S ={}& ab \sin C = (R\sin A)(R\sin B)\sin C\\
       ={}& R^2 \sin A\sin B\sin C \le R^2 \cdot \frac{3\sqrt3}{8}     
  \end{align*}
  当且仅当$A=B=C$时等号成立,从而当且仅当三角形是正三角形时其面积$S$取得最大值。同样,对于周长$p$,有
  \begin{align*}
    p ={}& a + b + c = 2R\sin A + 2R\sin B + 2R\sin C\\
      ={}& 2R(\sin A + \sin B + \sin C) \le 3\sqrt3 R
  \end{align*}
  当且仅当$A=B=C$时等号成立,从而当且仅当三角形是正三角形时其周长$p$取得最大值。
\end{proof}

\begin{theorem}
  任意$\triangle ABC$,记$a,b,c$是其三边边长,$r,R$分别是其内切圆与外接圆半径,则
  \begin{align}
    rR=\frac{abc}{2(a+b+c)}
  \end{align}
\end{theorem}
\begin{proof}
  在定理~\ref{th:R-of-circumcircle},将$r=\dfrac{S}{s}$及$a+b+c=2s$代入可得。
\end{proof}

\begin{theorem}
  记凸四边形的四边分别为$a,b,c,d$,四个顶角依次为$A,B,C,D$,两条对角线为$m,n$,则
  \begin{align*}
    (mn)^2 ={}& (ac)^2 + (bd)^2 - 2abcd\cos(A+C)\\
           ={}& (ac)^2 + (bd)^2 - 2abcd\cos(B+D)
  \end{align*}
  \begin{center}
    \begin{tikzpicture}[scale=1.0]
      \coordinate(A) at (0,0); \coordinate(B) at (3,0);
      \coordinate(C) at (4,2); \coordinate(D) at (1,3);
      \draw(A)--(B)node[midway,below]{$a$}
              --(C)node[midway,sloped,below]{$b$}
              --(D)node[midway,sloped,above]{$c$}
              --(A)node[midway,sloped,above]{$d$};
      \draw(A)--(C)node[pos=.4,sloped,below]{$m$};
      \draw(B)--(D)node[pos=.6,sloped,below]{$n$};
      \tkzDrawPoints(A,B,C,D)
      \draw pic["$A$",draw=orange,<->,angle eccentricity=1.6,angle radius=.6cm] {angle=B--A--D};
      \draw pic["$C$",draw=orange,<->,angle eccentricity=1.6,angle radius=.6cm] {angle=D--C--B};
    \end{tikzpicture}
  \end{center}
\end{theorem}
\begin{proof}[提示]
  由于$A+C = 2\pi - (B+D)$,所以后两式是等价的。考虑左边的等式,其形式与余弦定理相似,可以考虑应用余弦定理。然而等式中的项的量纲是长度的4次方,而余弦定理中每项的量纲是长度的2次方,不能直接应用。将其变化为如下的等价形式:
  \begin{align*}
    n^2 = \left(\frac{ac}{m}\right)^2 + \left(\frac{bd}{m}\right)^2 - 2\cdot\frac{ac}{m}\cdot\frac{bd}{m}\cdot\cos(A+C)
  \end{align*}
  从量纲而言,上式有望应用余弦定理,为此需要构造角度$A+C$及长度$\nicefrac{ac}{m}$和$\nicefrac{bd}{m}$,这种形式的长度,可以考虑相似三角形。

  \begin{center}
    \begin{tikzpicture}[scale=1.5]
      \coordinate(A) at (0,0); \coordinate(B) at (3,0);
      \coordinate(C) at (5,2); \coordinate(D) at (1,2.5);
      % \draw pic["$A$",draw=orange,<->,angle eccentricity=1.6,angle radius=.6cm] {angle=B--A--D};
      % \draw pic["$C$",draw=orange,<->,angle eccentricity=1.6,angle radius=.6cm] {angle=D--C--B};

      % from tkz-euclide packge, to get angle
      \tkzFindAngle(D,C,A) \tkzGetAngle{angleDCA}
      \tkzFindAngle(C,A,D) \tkzGetAngle{angleCAD}
      \coordinate(E1)at($(A)!.5!-\angleDCA:(B)$);\coordinate(E2)at($(B)!.5!\angleCAD:(A)$);
      \tkzInterLL(A,E1)(B,E2)\tkzGetPoint{E}

      \tkzFindAngle(A,C,B) \tkzGetAngle{angleACB}
      \tkzFindAngle(B,A,C) \tkzGetAngle{angleBAC}
      \coordinate(F1)at($(A)!.5!\angleACB:(D)$);\coordinate(F2)at($(D)!.5!-\angleBAC:(A)$);
      \tkzInterLL(A,F1)(D,F2)\tkzGetPoint{F}

      \draw pic["",draw=orange,angle eccentricity=1.6,angle radius=.3cm,fill=blue!20] {angle=E--A--B};
      \draw pic["",draw=orange,angle eccentricity=1.6,angle radius=.3cm,fill=blue!20] {angle=D--C--A};

      \draw pic["",draw=orange,angle eccentricity=1.6,angle radius=.4cm,pattern color=red!20,pattern=north west lines] {angle=A--C--B};
      \draw pic["",draw=orange,angle eccentricity=1.6,angle radius=.4cm,pattern color=red!20,pattern=north west lines] {angle=D--A--F};

      \draw pic["",draw=orange,angle eccentricity=1.6,angle radius=.7cm,pattern color=orange!20,pattern=crosshatch] {angle=B--A--C};
      \draw pic["",draw=orange,angle eccentricity=1.6,angle radius=.7cm,pattern color=orange!20,pattern=crosshatch] {angle=F--D--A};

      \draw pic["",draw=orange,angle eccentricity=1.6,angle radius=.7cm,pattern color=red!20,pattern=crosshatch] {angle=B--A--C};
      \draw pic["",draw=orange,angle eccentricity=1.6,angle radius=.7cm,pattern color=red!20,pattern=crosshatch] {angle=F--D--A};
      
      \draw pic["",draw=orange,angle eccentricity=1.6,angle radius=.5cm,fill=green!20] {angle=C--A--D};
      \draw pic["",draw=orange,angle eccentricity=1.6,angle radius=.5cm,fill=green!20] {angle=A--B--E};
      \draw pic["",draw=orange,angle eccentricity=1.6,angle radius=.5cm,pattern=north east lines] {angle=C--A--D};
      \draw pic["",draw=orange,angle eccentricity=1.6,angle radius=.5cm,pattern=north east lines] {angle=A--B--E};

      \draw[help lines,line width=2pt](E)--(F)
                                      (A)--(E)node[midway,sloped,below]{$\nicefrac{ac}{m}$}
                                      (A)--(F)node[midway,sloped,below]{$\nicefrac{bd}{m}$};
      \draw[dashed,help lines](E)--(B)node[midway,sloped,below]{$\nicefrac{ad}{m}$}
                              (D)--(F)node[midway,sloped,above]{$\nicefrac{ad}{m}$};

      \draw(A)--(B)node[midway,below]{$a$}
              --(C)node[midway,sloped,below]{$b$}
              --(D)node[midway,sloped,above]{$c$}
              --(A)node[midway,sloped,above]{$d$};
      \draw(A)--(C)node[pos=.3,sloped,below]{$m$};
      \draw(B)--(D)node[pos=.6,sloped,below]{$n$};
      \tkzDrawPoints(A,B,C,D,E,F)
    \end{tikzpicture}
  \end{center}
  如上图,沿$a$边和$d$边向外作出两个辅助的三角形,然后连接两个三角形的另外两个顶点,由两条虚线边平行(同旁内角和等于$\pi$)且相等,可得一个平行四边形,再对浅粗线三角形应用余弦定理可得。

  上图是$A+C\le\pi$的情况。当$\pi<A+C<2\pi$的情况类似。
\end{proof}

由上面的例子,有
\begin{align*}
  (mn)^2 \le (ac)^2 + (bd)^2 + 2abcd = (ac + bd)^2 \iff mn \le ac + bd
\end{align*}
当且仅当$\cos(A+C) = -1$即$A+C=\pi$时等号成立。从而可得下面的托勒密定理。其实由图中的浅粗线三角形及“一边小于两边的和”也得可得到
\begin{align*}
  n\le \frac{ac}{m} + \frac{bd}{m} \iff mn\le ac + bd
\end{align*}
当且仅当三角形退化成线段且$n$边为最长边时等号成立。这就是下面的托勒密定理。

% 同样的,也可得到以下的不等式:
% \begin{align*}
%   \left| ac - bd \right| \le mn
% \end{align*}
% 当且仅当$\cos(A+C) = 1$即$A+C=2k\pi$时等号成立,此时四边形退化为线段。

\begin{theorem}[托勒密定理,托勒密不等式,Ptolemy's Inequality]
  任意凸四边形$ABCD$,有
  \begin{align}
    AB\cdot CD + AD\cdot BC\ge AC\cdot BD
  \end{align}
  当且仅当$ABCD$是圆内接四边形等号成立。
\end{theorem}

\begin{lemma}\label{lem:ax>bn+cm}
  如图,$P$是三角形内任意一点,三角形的三边分别为$a,b,c$,$P$到三角形三边的距离为$l,m,n$,到三个顶点的距离分别是$x,y,z$,则
  \begin{align*}
    ax \ge{} & bn + cm  &  by \ge{} & an + cl &  cz \ge{} & am + bl
  \end{align*}
  当且仅当$P$是三角形的外心时等号成立。

  类似的,有
  \begin{align*}
    ax \ge{}& bm + cn  &  by \ge{}& al + cn &  cz \ge{}& al + bm
  \end{align*}
  当且仅当$P$是三角形的垂心时等号成立。

  \begin{center}
    \begin{tikzpicture}[scale=.8]
      \begin{scope}
        \coordinate(B)at(0,0);\coordinate(C)at(6,0);\coordinate(A)at(5,5);\coordinate(P)at(3,1);
        \coordinate(P3)at($(A)!(P)!(B)$);\coordinate(P1)at($(B)!(P)!(C)$);\coordinate(P2)at($(C)!(P)!(A)$);

        \draw pic["$\alpha_1$",draw=orange,angle eccentricity=1.6,angle radius=.7cm,fill=blue!20] {angle=B--A--P};
        \draw pic["$\alpha_2$",draw=orange,angle eccentricity=1.6,angle radius=.5cm,pattern=crosshatch] {angle=P--A--C};

        \foreach \u/\v/\w in{A/P3/P,C/P1/P,C/P2/P}{
          \tkzMarkRightAngle(\u,\v,\w)
        }
        \draw(A)--(B)node[midway,sloped,above]{$c$}
             (A)--(C)node[midway,sloped,above]{$b$}
                --(B)node[midway,below]{$a$};
        \foreach \u/\p/\l in{A/above/x,B/above/y,C/above/z,P3/below/n,P2/above/m}{
          \draw(P)--(\u)node[midway,sloped,\p]{$\l$};
        }
        \draw(P)--(P1)node[midway,left]{$l$};
        \tkzDrawPoints(A,B,C,P,P1,P2,P3)
        \node[above]at(A){$A$};
        \node[below left]at(B){$B$};
        \node[below right]at(C){$C$};
        \path(P)++(-.025,.1)node[above]{$P$};
        \node[above right]at(P2){$M$};
        \node[above left]at(P3){$N$};
      \end{scope}
    \end{tikzpicture}
  \end{center}
\end{lemma}
\begin{proof}[提示]
  对于第一个不等式,作出$ax,bn,cm$这三个长度。但按量纲来看,这些项都是长度的平方,先变换为
  \begin{align*}
    a \ge \frac{bn}{x} + \frac{cm}{x}
  \end{align*}
  或者更简单的,把$x,n,m$看成是不带量纲的常数。若是使用$\nicefrac{bn}{x}$的形式,则可以画到原图上。若是使用放大形式,则需要新画一图,把原三角形的边长放大$x$倍得出一新三角形,则三角形的三边分别是$ax,bx,cx$,然后根据下面的图作出$bn,cm$。

    \begin{center}
    \begin{tikzpicture}[scale=.75]
      \begin{scope}[shift={(0,0)}]
        \coordinate(B)at(0,0);\coordinate(C)at(6,0);\coordinate(A)at(5,5);\coordinate(P)at(3,1);
        \tkzFindAngle(B,A,P) \tkzGetAngle{angleBAP}
        \tkzFindAngle(P,A,C) \tkzGetAngle{anglePAC}
        \coordinate(D1)at($(B)!.5!\anglePAC:(A)$);\coordinate(D)at($(B)!(A)!(D1)$);
        \coordinate(E1)at($(C)!.5!-\angleBAP:(A)$);\coordinate(E)at($(C)!(A)!(E1)$);

        \draw pic["$\alpha_2$",draw=orange,angle eccentricity=1.6,angle radius=.7cm,pattern=crosshatch] {angle=A--B--D};
        \draw pic["$\alpha_1$",draw=orange,angle eccentricity=2.0,angle radius=.7cm,fill=blue!20] {angle=E--C--A};
        \tkzMarkRightAngle(A,D,B)\tkzMarkRightAngle(A,E,C)

        \coordinate(P3)at($(A)!(P)!(B)$);\coordinate(P1)at($(B)!(P)!(C)$);\coordinate(P2)at($(C)!(P)!(A)$);

        % \foreach \u/\v/\w in{A/P3/P,C/P1/P,C/P2/P}{
        %   \tkzMarkRightAngle(\u,\v,\w)
        % }
        \draw(A)--(B)node[midway,sloped,above]{$cx$}
             (A)--(C)node[pos=.4,sloped,above]{$bx$}
                --(B)node[midway,below]{$ax$};
        % \foreach \u/\p/\l in{A/above/x,B/above/y,C/above/z,P3/below/r,P2/above/q}{
        %   \draw(P)--(\u)node[midway,sloped,\p]{$\l$};
        % }
        % \draw(P)--(P1)node[midway,left]{$p$};
        \draw[help lines](A)--(P) (B)--(P) (C)--(P);
        \tkzDrawPoints(A,B,C,P)%,P1,P2,P3)
        \node[above left]at(A){$A'$};
        \node[below left]at(B){$B'$};
        \node[below right]at(C){$C'$};
        \path(P)++(-.025,.1)node[above]{$P'$};

        \draw[dashed,help lines](B)--(D)
                                   --(A)node[midway,above]{$cm$}
                                   --(E)node[midway,above]{$bn$}
                                   --(C);
        \node[below]at(3,-.7){$ax\ge bn + cm$};
      \end{scope}
      \begin{scope}[shift={(8,0)}]
        \coordinate(B)at(0,0);\coordinate(C)at(6,0);\coordinate(A)at(5,5);\coordinate(P)at(3,1);
        \tkzFindAngle(B,A,P) \tkzGetAngle{angleBAP}
        \tkzFindAngle(P,A,C) \tkzGetAngle{anglePAC}
        \coordinate(D1)at($(B)!.5!\angleBAP:(A)$);\coordinate(D)at($(B)!(A)!(D1)$);
        \coordinate(E1)at($(C)!.5!-\anglePAC:(A)$);\coordinate(E)at($(C)!(A)!(E1)$);

        \draw pic["$\alpha_1$",draw=orange,angle eccentricity=1.6,angle radius=.7cm,fill=blue!20] {angle=A--B--D};
        \draw pic["$\alpha_2$",draw=orange,angle eccentricity=1.6,angle radius=.5cm,pattern=crosshatch] {angle=E--C--A};
        \tkzMarkRightAngle(A,D,B)\tkzMarkRightAngle(A,E,C)
        \coordinate(P3)at($(A)!(P)!(B)$);\coordinate(P1)at($(B)!(P)!(C)$);\coordinate(P2)at($(C)!(P)!(A)$);
        % \foreach \u/\v/\w in{A/P3/P,C/P1/P,C/P2/P}{
        %   \tkzMarkRightAngle(\u,\v,\w)
        % }
        \draw(A)--(B)node[midway,sloped,above]{$cx$}
             (A)--(C)node[midway,sloped,above]{$bx$}
                --(B)node[midway,below]{$ax$};
        % \foreach \u/\p/\l in{A/above/x,B/above/y,C/above/z,P3/below/r,P2/above/q}{
        %   \draw(P)--(\u)node[midway,sloped,\p]{$\l$};
        % }
        % \draw(P)--(P1)node[midway,left]{$p$};
        \draw[help lines](A)--(P) (B)--(P) (C)--(P);
        \tkzDrawPoints(A,B,C,P)%,P1,P2,P3)
        \node[above right]at(A){$A'$};
        \node[below left]at(B){$B'$};
        \node[below right]at(C){$C'$};
        \path(P)++(-.025,.1)node[above]{$P'$};

        \draw[dashed,help lines](B)--(D)
                                   --(A)node[midway,above]{$cn$}
                                   --(E)node[midway,above]{$bm$}
                                   --(C);
        \node[below]at(3,-.7){$ax\ge bm + cn$};
      \end{scope}
    \end{tikzpicture}
  \end{center}


  如上面左图,先做一个放大的三角形$\triangle A'B'C'$,然后在$B',C'$上向$A'$一侧按图中的角度作射线,由同旁内角和为$\pi$可知两线平行。再由$A'$作两线的垂线可得两三角形分别与题中原图两个含角度$\alpha_1$和$\alpha_2$的直角三角形相似,由此可算得左图中$A'$到两虚线的距离。类似地可作出右图。

  由左图的直角梯形,可知$ax\ge bn+cm$,当且仅当梯形为矩形时成立。$P$是$\triangle ABC$的外心$\iff P'$是$\triangle A'B'C'$的外心。若$P'$是外心,则$P'$到$A', B', C'$的距离相等,容易知道$B'C'$与虚线的夹角是直角。反之,若梯形是矩形,则
  \begin{align*}
    \angle B = \frac\pi2 - \alpha_2 = \angle APM,\quad
    \angle C = \frac\pi2 - \alpha_1 = \angle APN
  \end{align*}
  同样的,若在顶点$C'$的两边向外做三角形,同样可以得到$\angle B = \angle CPM$,从而$\triangle APC$是等腰三角形,$x=z$。类似的就可以推出
  \begin{align*}
    x = y = z
  \end{align*}
  即$P$为三角形$\triangle ABC$的外心。
  
  由右图的直角梯形,可知$ax \ge bm + cn$,又$A'P'$与两虚线平行,因此当且仅当$\alpha_1 + B = 90^\circ$时等号成立,即当且仅当$AP\perp BC$时等号成立。从而三式的等号都成立等价于$P$是垂心。也可用面积法得到此结论。作$BC$边上的高$AD$,记其长度为$h_a$,则$x\ge h_a - l$,当且仅当$P$在$AD$上时等号成立。

  \begin{center}
    \begin{tikzpicture}[scale=.8]
      \begin{scope}
        \coordinate(B)at(0,0);\coordinate(C)at(6,0);\coordinate(A)at(5,5);\coordinate(P)at(3,1);
        \coordinate(P3)at($(A)!(P)!(B)$);\coordinate(P1)at($(B)!(P)!(C)$);\coordinate(P2)at($(C)!(P)!(A)$);

        \fill[color=red!10](A)--(B)--(P)--(C)--(A);

        \draw pic["$\alpha_1$",draw=orange,angle eccentricity=1.6,angle radius=.7cm,fill=blue!20] {angle=B--A--P};
        \draw pic["$\alpha_2$",draw=orange,angle eccentricity=1.6,angle radius=.5cm,pattern=crosshatch] {angle=P--A--C};

        \foreach \u/\v/\w in{A/P3/P,C/P1/P,C/P2/P}{
          \tkzMarkRightAngle(\u,\v,\w)
        }
        \draw(A)--(B)node[midway,sloped,above]{$c$}
             (A)--(C)node[midway,sloped,above]{$b$}
                --(B)node[midway,below]{$a$};
        \foreach \u/\p/\l in{A/above/x,B/above/y,C/above/z,P3/below/n,P2/above/m}{
          \draw(P)--(\u)node[midway,sloped,\p]{$\l$};
        }
        \draw(P)--(P1)node[midway,left]{$l$};
        \tkzDrawPoints(A,B,C,P,P1,P2,P3)
        \node[above]at(A){$A$};
        \node[below left]at(B){$B$};
        \node[below right]at(C){$C$};
        \path(P)++(-.025,.1)node[above]{$P$};
        \node[above right]at(P2){$M$};
        \node[above left]at(P3){$N$};

        \coordinate[label=below:$D$](D)at($(B)!(A)!(C)$);
        \draw[help lines,dashed](A)--(D)node[midway,left]{$h_a$};
        \tkzMarkRightAngle(A,D,C)
      \end{scope}
    \end{tikzpicture}
  \end{center}
  记燕尾$ABPC$的面积为$S_{ABPC}$,则有
  \begin{align*}
    \begin{cases}
      S_{ABPC} = \frac12 bm + \frac12 cn\\
      S_{ABPC} = \frac12 ah_a - \frac12 al = \frac12 a(h_a- l) \le \frac12 ax
    \end{cases}
    \implies  ax \ge bm + cn &\qedhere
  \end{align*}
\end{proof}

\begin{lemma}
  在上面的图中,有
  \begin{align*}
    xyz\ge 8 lmn
  \end{align*}
\end{lemma}
\begin{proof}
  对每项使A—G不等式然后三式相乘,有
  \begin{align*}
    ax\cdot by\cdot cz \ge 2\sqrt{bqcr} \cdot 2\sqrt{crap} \cdot 2\sqrt{apbq}
    = 8abc \cdot lmn
  \end{align*}
  两边同除$abc$可得。
\end{proof}


\begin{theorem}[鄂尔多斯—门德尔不等式,Erdos-Mordell Inequality,E-M不等式]
  $P$是三角形$ABC$内任意一点,$P$到三角形三边的距离分别为$l,m,n$,到三顶点的距离分别是$x,y,z$,则
  \begin{align}
    x+y+z\ge 2(l+m+n)
  \end{align}
\end{theorem}
\begin{proof}[提示]
  利用引理~\ref{lem:ax>bn+cm},有
  \begin{align*}
    x \ge \frac{bn + cm}{a}, \quad y \ge \frac{an + cl}{b}, \quad z \ge \frac{am + bl}{c}
  \end{align*}
  三式相加并由AM--GM不等式可得
  \begin{align*}
    x + y + z ={}& l\left(\frac bc + \frac cb\right) + m\left(\frac ac + \frac ca\right) + n\left(\frac ab + \frac ba\right)\\
    \ge{}& 2(l + m + n)
  \end{align*}
  当且仅当$a=b=c$且$P$为三角形的外心时等号成立。
\end{proof}

\begin{theorem}
  记三角形的外接圆与内切圆的半径分别为$R,r$,则
  \begin{align*}
    l + m + n = R + r
  \end{align*}
\end{theorem}
\begin{proof}定理~\ref{ex:vivianis-theorem}是此定理的特例。

  \color{red}等补充,还没有思路。
\end{proof}